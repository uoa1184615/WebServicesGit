%!TEX root = ../diverseExamples.tex
\subsection{\texttt{randWalkIn2D}: advection-diffusion of random walk in 2D} 
\label{randWalkIn2D}

A heat exchanger is simply two pipes with `fluid' flowing in opposite directions, and exchanging heat between them.  Let \(u_1(x,t)\C u_2(x,t)\) be the temperatures in the two pipes as a function of space-time.   Advecting at the same but opposite velocities, noon-dimensional \pde{}s are
\begin{align*}
&\D t{u_1}=-\D x{u_1}+u_2-u_1\,,
&&\D t{u_2}=+\D x{u_1}+u_1-u_2\,.
\end{align*}

The linearisation for \emph{gradual} variations in space is to then neglect the spatial derivatives: \(\D t{u_1}=u_2-u_1\C \D t{u_2}=u_1-u_2\)\,.
This has eigenvalues \(\lambda=0,-2\) with respective eigenvectors \((1,1)\C (1,-1)\).
We model effective dispersion of heat in these two pipes in space-time over long times and large space scales.

Start by loading the procedure.
\begin{reduce}
in_tex "../invariantManifold.tex"$
\end{reduce}
In the printed output, group terms depending upon order of spatial derivatives (which are assumed `small').
\begin{reduce}
factor small;
\end{reduce}
The following procedure call constructs the slow manifold for this system to errors~\Ord{\grad^5}.
\begin{reduce}
invariantmanifold( {x_1,x_2},
    mat((-diff(u1,x_1)-diff(u1,x_2)+(u2-u1)
        ,+diff(u2,x_1)           +(u1-2*u2+u3)
        ,-diff(u3,x_1)+diff(u3,x_2)+(u2-u3)     
        )),
    mat(( 0 )),
    mat( (1/3,1/3,1/3) ),
    mat( (1,1,1) ),
    4 )$
end;
\end{reduce}


The procedure then actually analyses the parametrised system
\begin{align*}&
\D t{u_{1}}=\eps \big(-\frac{{\rm d}\,u_1}{{\rm d}\,x_1}-\frac{{\rm d}
\,u_1}{{\rm d}\,x_2}\big)-u_1+u_2\,,
\\&
\D t{u_{2}}=\eps \frac{{\rm d}\,u_2}{{\rm d}\,x_1}+u_1-2 u_2+u_3\,.
\\&
\D t{u_{3}}=\eps \big(-\frac{{\rm d}\,u_3}{{\rm d}\,x_1}+\frac{{\rm d}
\,u_3}{{\rm d}\,x_2}\big)+u_2-u_3\,.
\end{align*}
Consequently the procedure's artificial parameter~\(\eps\)\ counts the number of spatial derivatives in each term.


\paragraph{The invariant manifold} 
Five iterations constructs the slow manifold model.
The slow manifold is expressed in terms of a series in space derivatives.
\begin{align*}&
u_{1}=\ParMath{ \eps^{2} \big(8/27 \frac{{\rm d}^{2}s_{1}}{
{\rm d}\,x\_1{\rm d}\,x\_2}-4/243 \frac{{\rm d}^{2}s_{1}}{{\rm d}x\_1^{2
}}+1/27 \frac{{\rm d}^{2}s_{1}}{{\rm d}x\_2^{2}}\big)+\eps \big(-2/27 
\frac{{\rm d}\,s_{1}}{{\rm d}\,x\_1}-1/3 \frac{{\rm d}\,s_{1}}{{\rm d}\,
x\_2}\big)+O\big(\varepsilon ^{3}\big)+1/3 s_{1}
}\\&
u_{2}=\ParMath{ \eps^{2}
 \big(8/243 \frac{{\rm d}^{2}s_{1}}{{\rm d}x\_1^{2}}-2/27 \frac{{\rm d}
^{2}s_{1}}{{\rm d}x\_2^{2}}\big)+4/27 \eps \frac{{\rm d}\,s_{1}}{{\rm d}
\,x\_1}+O\big(\varepsilon ^{3}\big)+1/3 s_{1}
}\\&
u_{3}=\ParMath{ \eps^{2} \big(-8/27 \frac{{\rm d}^{2}s_{1}}{
{\rm d}\,x\_1{\rm d}\,x\_2}-4/243 \frac{{\rm d}^{2}s_{1}}{{\rm d}x\_1^{2
}}+1/27 \frac{{\rm d}^{2}s_{1}}{{\rm d}x\_2^{2}}\big)+\eps \big(-2/27 
\frac{{\rm d}\,s_{1}}{{\rm d}\,x\_1}+1/3 \frac{{\rm d}\,s_{1}}{{\rm d}\,
x\_2}\big)+O\big(\varepsilon ^{3}\big)+1/3 s_{1}
}\end{align*}

 
\paragraph{Invariant manifold PDEs} 
The system evolves according to this \pde\ that describes the effective movement of the random walker: an advection-diffusion \pde, with anisotropic diffusion, and third-order dispersive effects included:
\begin{align*}&
\D t{s_{1}}=\ParMath{ \eps^{3} \big(-20/27 
\frac{{\rm d}^{3}s_{1}}{{\rm d}\,x\_1{\rm d}x\_2^{2}}+16/243 \frac{
{\rm d}^{3}s_{1}}{{\rm d}x\_1^{3}}\big)+\eps^{2} \big(8/27 \frac{{\rm d}
^{2}s_{1}}{{\rm d}x\_1^{2}}+2/3 \frac{{\rm d}^{2}s_{1}}{{\rm d}x\_2^{2}}
\big)-1/3 \eps \frac{{\rm d}\,s_{1}}{{\rm d}\,x\_1}+O\big(\varepsilon ^{
4}\big)
}\end{align*}


\paragraph{Project initial conditions et al.}
To project initial conditions
onto the slow manifold, or non-autonomous
forcing, or modifications of the original system, or to quantify uncertainty \cite[Ch.12]{Roberts89b, Roberts97b, Roberts2014a}, use the projection defined by the following derived vector.
\begin{align*}&
\zv_1=\begin{bmatrix}z_{11}&z_{12}&z_{13}\end{bmatrix}^T
\\&
=\begin{bmatrix}
\ParMath{ \eps^{3} \big(-8/729 \diff_{x\_1}^{3}-4/27 \diff_{x\_1}^{2} \diff
_{x\_2}+38/27 \diff_{x\_1} \diff_{x\_2}^{2}+11/9 \diff_{x\_2}^{3}\big)+
\eps^{2} \big(-4/27 \diff_{x\_1}^{2}+8/9 \diff_{x\_1} \diff_{x\_2}-5/9 
\diff_{x\_2}^{2}\big)+\eps \big(-2/9 \diff_{x\_1}-\diff_{x\_2}\big)+O
\big(\varepsilon ^{4}\big)+1
}\\[4ex]
\ParMath{ \eps^{3} \big(-80/729 \diff_{x\_1}^{3}+28/27 \diff_{x\_1} \diff_{
x\_2}^{2}\big)-8/9 \eps^{2} \diff_{x\_2}^{2}+4/9 \eps \diff_{x\_1}+O
\big(\varepsilon ^{4}\big)+1
}\\[4ex]
\ParMath{ \eps^{3} \big(-8/729 \diff_{x\_1}^{3}+4/27 \diff_{x\_1}^{2} \diff
_{x\_2}+38/27 \diff_{x\_1} \diff_{x\_2}^{2}-11/9 \diff_{x\_2}^{3}\big)+
\eps^{2} \big(-4/27 \diff_{x\_1}^{2}-8/9 \diff_{x\_1} \diff_{x\_2}-5/9 
\diff_{x\_2}^{2}\big)+\eps \big(-2/9 \diff_{x\_1}+\diff_{x\_2}\big)+O
\big(\varepsilon ^{4}\big)+1
}
\end{bmatrix}.
\end{align*}

