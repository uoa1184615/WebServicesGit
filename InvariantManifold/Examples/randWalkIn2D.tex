%!TEX root = ../diverseExamples.tex
\subsection{\texttt{randWalkIn2D}: advection-diffusion of random walk in 2D} 
\label{randWalkIn2D}
\localtableofcontents

A `drunk' walker stumbles around in a 2D meadow. Let
position of the walker at any time~\(t\) be \((x_1,x_2)\).
The walker: \begin{itemize}
\item sometimes heads North-East, direction~\((1,1)\), but
may decide to turn West;
\item sometimes West, direction~\((-1,0)\), but may turn to
the North-East or South-East; and 
\item sometimes South-East, direction~\((1,-1)\), but may
turn back to the West.
\end{itemize}
  
Where can we expect the drunk walker to be as time varies?

Let \(u_j(x_1,x_2,t)\) be the probability of the walker
being at position~\((x_1,x_2)\) and walking in the
\(j\)th~of the mentioned directions. Then the
non-dimensional \pde{}s for these probabilities may be
\begin{align*}
&\D t{u_1}=-\D {x_1}{u_1}-\D {x_2}{u_1}-u_1+u_2\,,
\\&\D t{u_2}=+\D {x_1}{u_2}+u_1-2u_2+u_3\,,
\\&\D t{u_3}=-\D {x_1}{u_3}+\D{x_2}{u3}+u_2-u_3\,.
\end{align*}

The linearisation for \emph{gradual} variations in space is
to then neglect the spatial derivatives: \(\D
t{u_1}=-u_1+u_2\C \D t{u_2}=+u_1-2u_2+u_3\C \D
t{u_3}=u_2-u_3\)\,. This has eigenvalues \(\lambda=0,-1,-3\)
with respective eigenvectors \((1,1,1)\C (1,0,-1)\C
(1,-2,1)\). We use this information to model the probability
distribution of the dispersion of the drunk walker in
space-time over long times and large space scales.

Start by loading the procedure.
\begin{reduce}
in_tex "../invariantManifold.tex"$
\end{reduce}
In the printed output, group terms depending upon order of
spatial derivatives (which are assumed `small').
\begin{reduce}
factor small;
\end{reduce}
The following procedure call constructs the slow manifold
for this system to errors~\Ord{\grad^5}.
\begin{reduce}
invariantmanifold( {x_1,x_2},
    mat((-pdf(u1,x_1)-pdf(u1,x_2)+(u2-u1)
        ,+pdf(u2,x_1)           +(u1-2*u2+u3)
        ,-pdf(u3,x_1)+pdf(u3,x_2)+(u2-u3)     
        )),
    mat(( 0 )),
    mat( (1/3,1/3,1/3) ),
    mat( (1,1,1) ),
    4 )$
end;
\end{reduce}


The procedure then actually analyses the parametrised system
\begin{align*}&
\D t{u_{1}}=\eps \big(-\frac{\partial\,u_1}{\partial\,x_1}-\frac{\partial
\,u_1}{\partial\,x_2}\big)-u_1+u_2\,,
\\&
\D t{u_{2}}=\eps \frac{\partial\,u_2}{\partial\,x_1}+u_1-2 u_2+u_3\,.
\\&
\D t{u_{3}}=\eps \big(-\frac{\partial\,u_3}{\partial\,x_1}+\frac{\partial
\,u_3}{\partial\,x_2}\big)+u_2-u_3\,.
\end{align*}
Consequently the procedure's artificial parameter~\(\eps\)\
counts the number of spatial derivatives in each term.


\paragraph{The invariant manifold} 
Five iterations constructs the slow manifold model. The slow
manifold is expressed in terms of a series in space
derivatives.
\begin{align*}&
u_{1}=\ParMath{ \eps^{2} \big(8/27 \frac{\partial^{2}s_{1}}{
\partial\,x_1\partial\,x_2}-4/243 \frac{\partial^{2}s_{1}}{\partial x_1^{2
}}+1/27 \frac{\partial^{2}s_{1}}{\partial x_2^{2}}\big)+\eps \big(-2/27 
\frac{\partial\,s_{1}}{\partial\,x_1}-1/3 \frac{\partial\,s_{1}}{\partial\,
x_2}\big)+O\big(\varepsilon ^{3}\big)+1/3 s_{1}
}\\&
u_{2}=\ParMath{ \eps^{2}
 \big(8/243 \frac{\partial^{2}s_{1}}{\partial x_1^{2}}-2/27 \frac{\partial
^{2}s_{1}}{\partial x_2^{2}}\big)+4/27 \eps \frac{\partial\,s_{1}}{\partial
\,x_1}+O\big(\varepsilon ^{3}\big)+1/3 s_{1}
}\\&
u_{3}=\ParMath{ \eps^{2} \big(-8/27 \frac{\partial^{2}s_{1}}{
\partial\,x_1\partial\,x_2}-4/243 \frac{\partial^{2}s_{1}}{\partial x_1^{2
}}+1/27 \frac{\partial^{2}s_{1}}{\partial x_2^{2}}\big)+\eps \big(-2/27 
\frac{\partial\,s_{1}}{\partial\,x_1}+1/3 \frac{\partial\,s_{1}}{\partial\,
x_2}\big)+O\big(\varepsilon ^{3}\big)+1/3 s_{1}
}\end{align*}

 
\paragraph{Invariant manifold PDEs} 
The system evolves according to this \pde\ that describes
the effective movement of the random walker: an
advection-diffusion \pde, with anisotropic diffusion, and
third-order dispersive effects included:
\begin{align*}&
\D t{s_{1}}=\ParMath{ \eps^{3} \big(-20/27 
\frac{\partial^{3}s_{1}}{\partial\,x_1\partial x_2^{2}}+16/243 \frac{
\partial^{3}s_{1}}{\partial x_1^{3}}\big)+\eps^{2} \big(8/27 \frac{\partial
^{2}s_{1}}{\partial x_1^{2}}+2/3 \frac{\partial^{2}s_{1}}{\partial x_2^{2}}
\big)-1/3 \eps \frac{\partial\,s_{1}}{\partial\,x_1}+O\big(\varepsilon ^{
4}\big)
}\end{align*}
So, on average, the walker will drift in the
\(+x_1\)-direction, but with significant and growing spread
in the \(x_1x_2\)-meadow.


\paragraph{Project initial conditions et al.}
To project initial conditions onto the slow manifold, or
non-autonomous forcing, or modifications of the original
system, or to quantify uncertainty \cite[Ch.12]{Roberts89b,
Roberts97b, Roberts2014a}, use the projection defined by the
following derived vector.
\begin{align*}&
\zv_1=\begin{bmatrix}z_{11}&z_{12}&z_{13}\end{bmatrix}^T
\\&
=\begin{bmatrix}
\ParMath{ \eps^{3} \big(-8/729 \diff_{x_1}^{3}-4/27 \diff_{x_1}^{2} \diff
_{x_2}+38/27 \diff_{x_1} \diff_{x_2}^{2}+11/9 \diff_{x_2}^{3}\big)+
\eps^{2} \big(-4/27 \diff_{x_1}^{2}+8/9 \diff_{x_1} \diff_{x_2}-5/9 
\diff_{x_2}^{2}\big)+\eps \big(-2/9 \diff_{x_1}-\diff_{x_2}\big)+O
\big(\varepsilon ^{4}\big)+1
}\\[4ex]
\ParMath{ \eps^{3} \big(-80/729 \diff_{x_1}^{3}+28/27 \diff_{x_1} \diff_{
x_2}^{2}\big)-8/9 \eps^{2} \diff_{x_2}^{2}+4/9 \eps \diff_{x_1}+O
\big(\varepsilon ^{4}\big)+1
}\\[4ex]
\ParMath{ \eps^{3} \big(-8/729 \diff_{x_1}^{3}+4/27 \diff_{x_1}^{2} \diff
_{x_2}+38/27 \diff_{x_1} \diff_{x_2}^{2}-11/9 \diff_{x_2}^{3}\big)+
\eps^{2} \big(-4/27 \diff_{x_1}^{2}-8/9 \diff_{x_1} \diff_{x_2}-5/9 
\diff_{x_2}^{2}\big)+\eps \big(-2/9 \diff_{x_1}+\diff_{x_2}\big)+O
\big(\varepsilon ^{4}\big)+1
}
\end{bmatrix}.
\end{align*}

