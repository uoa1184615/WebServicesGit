%!TEX root = ../diverseExamples.tex
\subsection{\texttt{doubleHopfDDE}: Double Hopf interaction in a 2D DDE} 
\label{doubleHopfDDE}

\cite{Erneux2009} [\S7.2] explored an example of a laser
subject to optoelectronic feedback, coded as a delay
differential equation. For certain parameter values it has a
two frequency Hopf bifurcation. Near Erneux's parameters
$(\eta,\theta)=(3/5,2)$, the system may be represented as
\begin{align*}&
\dot u_{1}=-4(1+\delta)^2\big[\tfrac58u_2+\tfrac38u_2(t-\pi)\big]
\\&
\dot u_{2}= u_1(1+ u_2).
\end{align*}
for small parameter~\(\delta\). Due to the
delay,~\(u_2(t-\pi)\), this system is effectively an
infinite-dimensional dynamical system. Here we describe the
emergent dynamics on its four-dimensional centre manifold.

The linearisation of this system at the origin has modes
with frequencies \(\omega=1,2\), and corresponding
eigenvectors \((1,\mp i/\omega)e^{\pm i\omega t}\). 
Corresponding  eigenvectors of the adjoint are \((1,\mp
i\omega)e^{\pm i\omega t}\). We model the nonlinear
interaction of these four modes over long times.


Start by loading the procedure.
\begin{reduce}
in_tex "../invariantManifold.tex"$
\end{reduce}
But turn off \verb|gcd| as it wrecks this code for some unknown reason.
\begin{reduce}
off gcd,ezgcd; 
\end{reduce}
In the printed output, group terms with like powers of
amplitudes~\(s_j\), the complex exponential, and the
parameter~\(\delta\).
\begin{reduce}
factor s,delta,exp;
\end{reduce}
Execute the construction of the slow manifold for this
system, where \verb|u2(pi)| denotes the delayed
variable~\(u_2(t-\pi)\), and where \verb|1+small*delta|
reflects that we wish to use the `small'
parameter~\(\delta\) to explore regimes where this factor is
near the value~\(1\).
\begin{reduce}
invariantmanifold(
    mat(( -4*(1+small*delta)^2*(5/8*u2 +3/8*u2(pi)),
          +u1*(1+u2) )),
    mat(( i,-i,2*i,-2*i )),
    mat( (1,-i), (1,+i), (1,-i/2), (1,+i/2) ),
    mat( (1,-i), (1,+i), (1,-2*i), (1,+2*i) ),
    3 )$
end;
\end{reduce}
The code works for errors of order higher than cubic, but is
much slower: takes several minutes per iteration.

The procedure actually analyses the embedding system
\begin{align*}&
\dot u_{1}=-4(1+2\eps^2\delta+\eps^3\delta^2)\big[\tfrac58u_2+\tfrac38u_2(t-\pi)\big]
\\&
\dot u_{2}= u_1(1+ \eps u_2).
\end{align*}



\paragraph{The centre manifold} 
These give the location of the invariant manifold in terms
of parameters~\(s\sb j\). Here, \(u_1\approx
s_1e^{it}+s_2e^{-it}+s_3e^{i2t}+s_4e^{-i2t}\) so that (for
real solutions) \(s_1,s_2\) are complex conjugate amplitudes
that modulate the oscillations of frequency \(\omega=1\)\,,
whereas \(s_3,s_4\) are complex conjugate amplitudes that
modulate the oscillations of frequency \(\omega=2\)\,.
\begin{align*}&
u_{1}=\ParMath{\exp \big(-i t\big) s_{4} s_{1} \eps \big(0.2309 i-0.04495
\big)+\exp \big(-i t\big) s_{2}+0.1667 \exp \big(-4 i t\big) s_{4}^{2}
 \eps i+0.1875 \exp \big(-3 i t\big) s_{4} s_{2} \eps i+\exp \big(-2 i t
\big) s_{4}+\exp \big(-2 i t\big) s_{2}^{2} \eps \big(-0.3953 i-
0.1233\big)+\exp \big(i t\big) s_{3} s_{2} \eps \big(-0.2309 i-
0.04495\big)+\exp \big(i t\big) s_{1}-0.1667 \exp \big(4 i t\big) s_
{3}^{2} \eps i-0.1875 \exp \big(3 i t\big) s_{3} s_{1} \eps i+\exp \big(
2 i t\big) s_{3}+\exp \big(2 i t\big) s_{1}^{2} \eps \big(0.3953 i-
0.1233\big)}
\\&
u_{2}=\ParMath{\exp \big(-i t\big) s_{4} s_{1} \eps \big(0.04495 i+0.2309
\big)+\exp \big(-i t\big) s_{2} i-0.1667 \exp \big(-4 i t\big) s_{4}^{
2} \eps-0.5625 \exp \big(-3 i t\big) s_{4} s_{2} \eps+0.5 \exp \big(-2 i
 t\big) s_{4} i+\exp \big(-2 i t\big) s_{2}^{2} \eps \big(0.06167 i-
0.1977\big)+\exp \big(i t\big) s_{3} s_{2} \eps \big(-0.04495 i+
0.2309\big)-\exp \big(i t\big) s_{1} i-0.1667 \exp \big(4 i t\big) s
_{3}^{2} \eps-0.5625 \exp \big(3 i t\big) s_{3} s_{1} \eps-0.5 \exp 
\big(2 i t\big) s_{3} i+\exp \big(2 i t\big) s_{1}^{2} \eps \big(-
0.06167 i-0.1977\big)}
\end{align*}

 
\paragraph{Centre manifold ODEs} 
The system evolves on the invariant manifold such that the
parameters evolve according to these ODEs that characterise
how the modulation of the oscillations evolve due to their
nonlinear interaction.
\begin{align*}&
\dot s_{1}=\ParMath{s_{4} s_{3} s_{1} \eps^{2} \big(-0.03089 i+0.05032\big)+s
_{3} s_{2} \eps \big(-0.08991 i-0.03816\big)+s_{2} s_{1}^{2} \eps^{2
} \big(-0.01837 i-0.1095\big)+s_{1} \delta  \eps^{2} \big(0.1526 i
-0.3596\big)}
\\&
\dot s_{2}=\ParMath{s_{4} s_{3} s_{2} \eps^{2} \big(0.03089 i+0.05032\big)+s_
{4} s_{1} \eps \big(0.08991 i-0.03816\big)+s_{2}^{2} s_{1} \eps^{2} 
\big(0.01837 i-0.1095\big)+s_{2} \delta  \eps^{2} \big(-0.1526 i-
0.3596\big)}
\\&
\dot s_{3}=\ParMath{s_{4} s_{3}^{2} \eps^{2} \big(-0.0349 i-0.04111\big)+s_{3
} s_{2} s_{1} \eps^{2} \big(-0.2499 i-0.2153\big)+s_{3} \delta  \eps
^{2} \big(0.8376 i+0.9867\big)+s_{1}^{2} \eps \big(-0.4934 i+
0.4188\big)}
\\&
\dot s_{4}=\ParMath{s_{4}^{2} s_{3} \eps^{2} \big(0.0349 i-0.04111\big)+s_{4}
 s_{2} s_{1} \eps^{2} \big(0.2499 i-0.2153\big)+s_{4} \delta  \eps^{
2} \big(-0.8376 i+0.9867\big)+s_{2}^{2} \eps \big(0.4934 i+
0.4188\big)}
\end{align*}


