\documentclass[11pt,a5paper]{article}
\usepackage[a5paper,margin=13mm]{geometry}
\usepackage{parskip,time} \raggedright
\def\eps{\varepsilon} \def\_{_}
\title{Invariant manifold of your dynamical system}
\author{A. J. Roberts, University of Adelaide\\
\texttt{http://orcid.org/0000-0001-8930-1552}}
\date{\now, \today}
\begin{document}
\maketitle
Throughout and generally: the lowest order, most
important, terms are near the end of each expression.

\(\)
\paragraph{The specified dynamical system}
\(
\)\par

\(\dot u_{1}=-\alpha  \eps^{2} u\sb3-\eps^{2} u\sb1^{3}-2 \eps u\sb1^{2}-4
 u\sb3
\)\par

\(\dot u_{2}=2 u\sb1-2 u\sb2
\)\par

\(\dot u_{3}=2 u\sb2-2 u\sb3
\)\par

\(\)
\paragraph{Invariant subspace basis vectors}
\(
\)\par

\(\vec e_{1}=\left\{
\left\{
1 , -1/2 i+1/2 , -1/2 i
\right\} , \cis\big(2 t\big)
\right\}
\)\par

\(\vec e_{2}=\left\{
\left\{
1 , 1/2 i+1/2 , 1/2 i
\right\} , \cis\big(-2 t\big)
\right\}
\)\par

\(\vec z_{1}=\left\{
\left\{
1/5 i+2/5 , -2/5 i+1/5 , -3/5 i-1/5
\right\} , \cis\big(2 t\big)
\right\}
\)\par

\(\vec z_{2}=\left\{
\left\{
-1/5 i+2/5 , 2/5 i+1/5 , 3/5 i-1/5
\right\} , \cis\big(-2 t\big)
\right\}
\)\par


off echo;

\(
\)
\paragraph{The invariant manifold}
These give the location of the invariant manifold in
terms of parameters~\(s\_ j\).
\(
\)\par
\(k_0=O\big(\varepsilon ^{3}\big)+k_0
\)\par
\(k_1=O\big(\varepsilon ^{3}\big)+k_1
\)\par
\(x=x+O\big(\varepsilon ^{3}\big)
\)\par
\(u_{4}=x \eps^{2} \big(-k_1^{2} k\_2 k\_5^{-3}+k_1^{2} k\_5^{-2}
\big)+x \eps k_1 k\_5^{-1}-\eps^{2} k_1 k_0 k\_5^{-2}+O\big(
\varepsilon ^{3}\big)
\)\par
\(
\)
\paragraph{Invariant manifold ODEs}
The system evolves on the invariant manifold such
that the parameters evolve according to these ODEs.
\(
\)\par
\(\dot k_0=O\big(\varepsilon ^{4}\big)
\)\par
\(\dot k_1=O\big(\varepsilon ^{4}\big)
\)\par
\(\dot x=x \eps^{3} \big(2 k_1^{3} k\_2^{3} k\_5^{-5}-3 k_1^{3
} k\_2^{2} k\_5^{-4}+k_1^{3} k\_2 k\_5^{-3}\big)+x \eps^{2} \big(-
k_1^{2} k\_2^{2} k\_5^{-3}+k_1^{2} k\_2 k\_5^{-2}\big)+x \eps 
\big(k_1 k\_2 k\_5^{-1}-k_1\big)+\eps^{3} \big(2 k_1^{2} k_0 
k\_2^{2} k\_5^{-4}-k_1^{2} k_0 k\_2 k\_5^{-3}\big)-\eps^{2} k_1 k_0 k\_2 k\_5^{-2}+\eps k_0+O\big(\varepsilon ^{4}\big)
\)\par
\(
\)
\paragraph{Normals to isochrons at the slow manifold}
Use these vectors: to project initial conditionsonto the slow manifold; to project non-autonomousforcing onto the slow evolution; to predict the
consequences of modifying the original system; in
uncertainty quantification to quantify effects on
the model of uncertainties in the original system.
The normal vector \(\vec z\_ j:=(z\_{j1},\ldots,z\_{jn})\)
\(
\)\par
\(z_{11}=O\big(\varepsilon ^{4}\big)+1
\)\par
\(z_{12}=O\big(\varepsilon ^{4}\big)
\)\par
\(z_{13}=O\big(\varepsilon ^{4}\big)
\)\par
\(z_{14}=O\big(\varepsilon ^{4}\big)
\)\par
\(z_{21}=O\big(\varepsilon ^{4}\big)
\)\par
\(z_{22}=O\big(\varepsilon ^{4}\big)+1
\)\par
\(z_{23}=O\big(\varepsilon ^{4}\big)
\)\par
\(z_{24}=O\big(\varepsilon ^{4}\big)
\)\par
\(z_{31}=\eps^{3} \big(-4 k_1^{2} k\_2^{2} k\_5^{-5}+2 k_1^{2} k\_2 
k\_5^{-4}\big)+\eps^{2} k_1 k\_2 k\_5^{-3}+O\big(\varepsilon ^{4}\big)
\)\par
\(z_{32}=x \eps^{3} \big(-16 k_1^{2} k\_2^{3} k\_5^{-6}+21 k_1^{2}
 k\_2^{2} k\_5^{-5}-6 k_1^{2} k\_2 k\_5^{-4}\big)+x \eps^{2} \big(
4 k_1 k\_2^{2} k\_5^{-4}-3 k_1 k\_2 k\_5^{-3}\big)-x \eps k\_2 
k\_5^{-2}+\eps^{3} \big(-6 k_1 k_0 k\_2^{2} k\_5^{-5}+3 k_1 k_0 
k\_2 k\_5^{-4}\big)+\eps^{2} k_0 k\_2 k\_5^{-3}+O\big(\varepsilon ^{4}
\big)
\)\par
\(z_{33}=\eps^{3} \big(-10 k_1^{3} k\_2^{3} k\_5^{-6}+12 k_1^{3} k\_2
^{2} k\_5^{-5}-3 k_1^{3} k\_2 k\_5^{-4}\big)+\eps^{2} \big(3 k_1^{2}
 k\_2^{2} k\_5^{-4}-2 k_1^{2} k\_2 k\_5^{-3}\big)-\eps k_1 k\_2 k\_5
^{-2}+O\big(\varepsilon ^{4}\big)+1
\)\par
\(z_{34}=\eps^{3} \big(-20 k_1^{3} k\_2^{4} k\_5^{-7}+30 k_1^{3} k\_2
^{3} k\_5^{-6}-12 k_1^{3} k\_2^{2} k\_5^{-5}+k_1^{3} k\_2 k\_5^{-4}
\big)+\eps^{2} \big(6 k_1^{2} k\_2^{3} k\_5^{-5}-6 k_1^{2} k\_2^{2} 
k\_5^{-4}+k_1^{2} k\_2 k\_5^{-3}\big)+\eps \big(-2 k_1 k\_2^{2} k\_5
^{-3}+k_1 k\_2 k\_5^{-2}\big)+O\big(\varepsilon ^{4}\big)+k\_2 k\_5^{
-1}
\)\par
\(
\)\par
\end{document}
