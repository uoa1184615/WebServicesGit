\documentclass[11pt,a5paper]{article}
\usepackage[a5paper,margin=13mm]{geometry}
\usepackage{parskip,time} \raggedright
\def\exp\big(#1\big){\,{\rm e}^{#1}}
\def\eps{\varepsilon}
\title{Invariant manifold of your dynamical system}
\author{A. J. Roberts, University of Adelaide\\
\texttt{http://orcid.org/0000-0001-8930-1552}}
\date{\now, \today}
\begin{document}
\maketitle
Throughout and generally: the lowest order, most
important, terms are near the end of each expression.

\(\)
\paragraph{The specified dynamical system}
\(
\)\par

\(\dot u_{1}=\eps^{2} \big(-3/8 \exp \big(-2 i t\big) i u\sb2+3/8 \exp 
\big(-2 i t\big) u\sb1+3/8 \exp \big(2 i t\big) i u\sb2+3/8 \exp \big(2 
i t\big) u\sb1+3/4 \exp \big(0\big) u\sb1\big)-u\sb1+u\sb2
\)\par

\(\dot u_{2}=\eps^{2} \big(-3/8 \exp \big(-2 i t\big) i u\sb1-3/8 \exp 
\big(-2 i t\big) u\sb2+3/8 \exp \big(2 i t\big) i u\sb1-3/8 \exp \big(2 
i t\big) u\sb2+3/4 \exp \big(0\big) u\sb2\big)-u\sb1-u\sb2
\)\par

\(\)
\paragraph{Invariant subspace basis vectors}
\(
\)\par

\(\vec e_{1}=\left\{
\left\{
1 , i
\right\} , \exp \big(i t-t\big)
\right\}
\)\par

\(\vec e_{2}=\left\{
\left\{
1 , -i
\right\} , \exp \big(-i t-t\big)
\right\}
\)\par

\(\vec z_{1}=\left\{
\left\{
1/2 , 1/2 i
\right\} , \exp \big(i t-t\big)
\right\}
\)\par

\(\vec z_{2}=\left\{
\left\{
1/2 , -1/2 i
\right\} , \exp \big(-i t-t\big)
\right\}
\)\par


off echo;


\(
\)
\paragraph{The invariant manifold}
These give the location of the invariant manifold in
terms of parameters~\(s\sb j\).
\(
\)\par

\(u_{1}=\eps^{4} \big(-s_{2}+2 s_{1}\big)-\eps^{2} s_{1}+s_{1}
\)\par

\(u_{2}=\eps^{4} \big(s_{2}-2 s_{1}\big)+\eps^{2} s_{1}
\)\par

\(u_{3}=\eps^{4} \big(-2 s_{2}+s_{1}\big)+\eps^{2} s_{2}
\)\par

\(u_{4}=\eps^{4} \big(2 s_{2}-s_{1}\big)-\eps^{2} s_{2}+s_{2}
\)\par

\(
\)
\paragraph{Invariant manifold ODEs}
The system evolves on the invariant manifold such
that the parameters evolve according to these ODEs.
\(
\)\par

\(\dot s_{1}=\eps^{4} \big(s_{2}-s_{1}\big)
\)\par

\(\dot s_{2}=\eps^{4} \big(-s_{2}+s_{1}\big)
\)\par

\(
\)
\paragraph{Normals to isochrons at the slow manifold}
Use these vectors: to project initial conditions
onto the slow manifold; to project non-autonomous
forcing onto the slow evolution; to predict the
consequences of modifying the original system; in
uncertainty quantification to quantify effects on
the model of uncertainties in the original system.
The normal vector \(\vec z\sb j:=(z\sb{j1},\ldots,z\sb{jn})\)
\(
\)\par

\(z_{11}=\eps^{4}+1
\)\par

\(z_{12}=4 \eps^{4}-\eps^{2}+1
\)\par

\(z_{13}=-4 \eps^{4}+\eps^{2}
\)\par

\(z_{14}=-\eps^{4}
\)\par

\(z_{21}=-\eps^{4}
\)\par

\(z_{22}=-4 \eps^{4}+\eps^{2}
\)\par

\(z_{23}=4 \eps^{4}-\eps^{2}+1
\)\par

\(z_{24}=\eps^{4}+1
\)\par
\end{document}
