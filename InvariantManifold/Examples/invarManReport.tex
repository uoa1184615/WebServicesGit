\documentclass[11pt,a5paper]{article}
\usepackage[a5paper,margin=13mm]{geometry}
\usepackage{parskip,time} \raggedright
\def\eps{\varepsilon} \def\_{_}
\title{Invariant manifold of your dynamical system}
\author{A. J. Roberts, University of Adelaide\\
\texttt{http://orcid.org/0000-0001-8930-1552}}
\date{\now, \today}
\begin{document}
\maketitle
Throughout and generally: the lowest order, most
important, terms are near the end of each expression.

\(\)
\paragraph{The specified dynamical system}
\(
\)\par

\(\dot u_{1}=-\alpha  \eps^{2} u\sb3-\eps^{2} u\sb1^{3}-2 \eps u\sb1^{2}-4
 u\sb3
\)\par

\(\dot u_{2}=2 u\sb1-2 u\sb2
\)\par

\(\dot u_{3}=2 u\sb2-2 u\sb3
\)\par

\(\)
\paragraph{Invariant subspace basis vectors}
\(
\)\par

\(\vec e_{1}=\left\{
\left\{
1 , -1/2 i+1/2 , -1/2 i
\right\} , \cis\big(2 t\big)
\right\}
\)\par

\(\vec e_{2}=\left\{
\left\{
1 , 1/2 i+1/2 , 1/2 i
\right\} , \cis\big(-2 t\big)
\right\}
\)\par

\(\vec z_{1}=\left\{
\left\{
1/5 i+2/5 , -2/5 i+1/5 , -3/5 i-1/5
\right\} , \cis\big(2 t\big)
\right\}
\)\par

\(\vec z_{2}=\left\{
\left\{
-1/5 i+2/5 , 2/5 i+1/5 , 3/5 i-1/5
\right\} , \cis\big(-2 t\big)
\right\}
\)\par


off echo;


\(
\)
\paragraph{The invariant manifold}
These give the location of the invariant manifold in
terms of parameters~\(s\sb j\).
\(
\)\par

\(u_{1}=\exp \big(-i t\big) s_{2}+\exp \big(i t\big) s_{1}+O\big(
\varepsilon \big)
\)\par

\(u_{2}=i \big(-\exp \big(-i t\big) s_{2}+\exp \big(i t\big) s_{1}\big)+O
\big(\varepsilon \big)
\)\par

\(u_{3}=O\big(\varepsilon \big)
\)\par

\(u_{4}=O\big(\varepsilon \big)
\)\par

\(u_{5}=O\big(\varepsilon \big)
\)\par

\(u_{6}=O\big(\varepsilon \big)
\)\par

\(
\)
\paragraph{Invariant manifold ODEs}
The system evolves on the invariant manifold such
that the parameters evolve according to these ODEs.
\(
\)\par

\(\dot s_{1}=i \eps \big(3/10 s_{2} s_{1}^{2}-1/4 a\big)+\eps \big(-1/2 s_
{1} c\_1+1/4 b\big)+O\big(\varepsilon ^{2}\big)
\)\par

\(\dot s_{2}=i \eps \big(-3/10 s_{2}^{2} s_{1}+1/4 a\big)+\eps \big(-1/2 s
_{2} c\_1+1/4 b\big)+O\big(\varepsilon ^{2}\big)
\)\par

\(
\)\par
\end{document}
