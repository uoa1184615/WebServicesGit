\documentclass[11pt,a5paper]{article}
\usepackage[a5paper,margin=13mm]{geometry}
\usepackage{parskip,time,verbatim} \def\_{_}
\raggedright \def\eps{\varepsilon}
\def\diff{\partial\makebox[0pt]{\ $\cdot$}}
\title{Invariant manifold of your dynamical system}
\author{A. J. Roberts, University of Adelaide\\
\texttt{http://orcid.org/0000-0001-8930-1552}}
\date{\now, \today}
\begin{document}
\maketitle
Throughout and generally: the lowest order, most
important, terms are near the end of each expression.
\par\leftskip=2em  \parindent=-2em

\(\)
\paragraph{The specified dynamical system}
\(
\)\par

\(\dot u_{1}=\eps^{2} \big(-3/8 \exp \big(-2 i t\big) i u\sb2+3/8 \exp 
\big(-2 i t\big) u\sb1+3/8 \exp \big(2 i t\big) i u\sb2+3/8 \exp \big(2 
i t\big) u\sb1+3/4 \exp \big(0\big) u\sb1\big)-u\sb1+u\sb2
\)\par

\(\dot u_{2}=\eps^{2} \big(-3/8 \exp \big(-2 i t\big) i u\sb1-3/8 \exp 
\big(-2 i t\big) u\sb2+3/8 \exp \big(2 i t\big) i u\sb1-3/8 \exp \big(2 
i t\big) u\sb2+3/4 \exp \big(0\big) u\sb2\big)-u\sb1-u\sb2
\)\par

\(\)
\paragraph{Invariant subspace basis vectors}
\(
\)\par

\(\vec e_{1}=\left\{
\left\{
1 , i
\right\} , \exp \big(i t-t\big)
\right\}
\)\par

\(\vec e_{2}=\left\{
\left\{
1 , -i
\right\} , \exp \big(-i t-t\big)
\right\}
\)\par

\(\vec z_{1}=\left\{
\left\{
1/2 , 1/2 i
\right\} , \exp \big(i t-t\big)
\right\}
\)\par

\(\vec z_{2}=\left\{
\left\{
1/2 , -1/2 i
\right\} , \exp \big(-i t-t\big)
\right\}
\)\par


off echo;

\(
\)
\paragraph{The invariant manifold}
These give the location of the invariant manifold in
terms of parameters~\(s\_ j\).
\(
\)\par
\(u_{1}=O\big(\varepsilon ^{3}\big)+s_{1}
\)\par
\(u_{2}=\frac{\partial \,s_{1}}{\partial \,x} \eps+O\big(\varepsilon ^{3}
\big)
\)\par
\(
\)
\paragraph{Invariant manifold ODEs}
The system evolves on the invariant manifold such
that the parameters evolve according to these ODEs.
\(
\)\par
\(\dot s_{1}=\frac{\partial ^{2}s_{1}}{\partial x^{2}} \eps^{2}+O\big(
\varepsilon ^{4}\big)
\)\par
\(
\)
\paragraph{Normals to isochrons at the slow manifold}
Use these vectors: to project initial conditions
onto the slow manifold; to project non-autonomous
forcing onto the slow evolution; to predict the
consequences of modifying the original system; in
uncertainty quantification to quantify effects on
the model of uncertainties in the original system.
The normal vector \(\vec z\_j:=(z\_{j1},\ldots,z\_{jn})\)
\(
\)\par
\(z_{11}=-\diff_{x}^{2} \eps^{2}+O\big(\varepsilon ^{4}\big)+1
\)\par
\(z_{12}=-2 \diff_{x}^{3} \eps^{3}+\diff_{x} \eps+O\big(\varepsilon ^{4}
\big)
\)\par
\(
\)\par
\end{document}
