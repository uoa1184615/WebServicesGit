%!TEX root = ../diverseExamples.tex
\subsection{\texttt{gradsSystem}: spatiotemporal long-waves in Grad's system} 
\label{gradsSystem}
\localtableofcontents

This is Example~B from ``Dynamically Optimal Projection onto
Slow Spectral Manifolds for Linear Systems'' by Kogelbauer
and Karlin (2025). The \pde{}s are apparently a classical
system in kinetic theory. Variables are the
pressure~\(u_1\), velocity~\(u_2\), and stress~\(u_3\). What
are the \pde{}s for the long-wave pressure-velocity waves?
\begin{align*}
&\D t{u_1}=-\tfrac53\D x{u_1}\,,
\\&\D t{u_2}=-\D x{u_1}-\D x{u_3}\,,
\\&\D t{u_3}=-\tfrac43\D x{u_2}-u_3/\epsilon\,.
\end{align*}

The linearisation for \emph{gradual} variations in space is
to then neglect the spatial derivatives: \(\D t{u_1}=\D
t{u_2}=0\) and \(\D t{u_3}=-u_3/\epsilon\) which decays to
quasi-equilibria quite `rapidly'. Simply, the pressure and
velocity are slow variables.

Start by loading the procedure.
\begin{reduce}
in_tex "../invariantManifold.tex"$
\end{reduce}
In the printed output, group terms depending upon order of
spatial derivatives (which are assumed `small').
\begin{reduce}
factor small;
\end{reduce}
The following procedure call constructs the slow manifold for this system to errors~\Ord{\partial_x^5}.
\begin{reduce}
invariantmanifold( {x},
    mat((-5/3*diff(u2,x)
        ,-diff(u1,x)-diff(u3,x)
        ,-4/3*diff(u2,x)-u3/epsilon
        )),
    mat(( 0,0 )),
    mat( (1,0,0),(0,1,0) ),
    mat( (1,0,0),(0,1,0) ),
    5 )$
end;
\end{reduce}


The procedure then actually analyses the parametrised system
\begin{align*}&
\D t{u_{1}}=-5/3 \eps \frac{{\rm d}\,u\_2}{{\rm d}\,x}\,,
\\&
\D t{u_{2}}=\eps \big(-\frac{{\rm d}\,u\_1}{{\rm d}\,x}-\frac{{\rm d}\,
u\_3}{{\rm d}\,x}\big) ,
\\&
\D t{u_{3}}=-4/3 \eps \frac{{\rm d}\,u\_2}{{\rm d}\,x}-\epsilon ^{-1} 
u\_3 \,.
\end{align*}
Be wary of the distinction between the system
parameter~\(\epsilon\), and the procedure's artificial
parameter~\(\eps\)\ that counts the number of spatial
derivatives in each term.


\paragraph{The invariant manifold} 
The slow manifold is expressed in terms of a series in space
derivatives of parameters~\(s\_ j:=u_j\)---the pressure and
velocity.
\begin{align*}&
u_{1}={ O\big(\varepsilon ^{4}\big)+s_{1}
}\\&
u_{2}={ O\big(\varepsilon ^{4}\big)+s_{2}
}\\&
u_{3}={ -4/9 \eps^{3} \frac{{\rm d}^{3}s_{2}}{{\rm d}x^{3}} \epsilon ^{3}-
4/3 \eps^{2} \frac{{\rm d}^{2}s_{1}}{{\rm d}x^{2}} \epsilon ^{2}-4/3 
\eps \frac{{\rm d}\,s_{2}}{{\rm d}\,x} \epsilon +O\big(\varepsilon ^{4}
\big)
}\end{align*}

 
\paragraph{Invariant manifold PDEs} 
The system evolves according to these two \pde\ that
describes the effective long-wave dynamics with higher-order
wave dispersion:
\begin{align*}&
\D t{s_{1}}={ -5/3 \eps \frac{{\rm d}\,s_{2}}{{\rm d}\,x}+O\big(
\varepsilon ^{5}\big),
}\\&
\D t{s_{1}}={ 4/9 \eps^{4} \frac{{\rm d}^{4}s_{2}}{{\rm d}x^{4}} \epsilon 
^{3}+4/3 \eps^{3} \frac{{\rm d}^{3}s_{1}}{{\rm d}x^{3}} \epsilon ^{2}+4/
3 \eps^{2} \frac{{\rm d}^{2}s_{2}}{{\rm d}x^{2}} \epsilon -\eps \frac{
{\rm d}\,s_{1}}{{\rm d}\,x}+O\big(\varepsilon ^{5}\big)
}\end{align*}



\paragraph{Project initial conditions et al.}
To project initial conditions onto the slow manifold, or
non-autonomous forcing, or modifications of the original
system, or to quantify uncertainty \cite[Ch.12]{Roberts89b,
Roberts97b, Roberts2014a}, use the projection defined by the
following two derived vectors.
\begin{align*}&
\zv_1=\begin{bmatrix}z_{11}\\z_{12}\\z_{13}\end{bmatrix}
=\begin{bmatrix}
{ -20/9 \eps^{4} \diff_{x}^{4} \epsilon ^{4}+O\big(\varepsilon ^{5}
\big)+1
}\\[1ex]
{ -20/9 \eps^{3} \diff_{x}^{3} \epsilon ^{3}+O\big(\varepsilon ^{5}
\big)
}\\[1ex]
{ 35/9 \eps^{4} \diff_{x}^{4} \epsilon ^{4}-5/3 \eps^{2} \diff_{x}
^{2} \epsilon ^{2}+O\big(\varepsilon ^{5}\big)
}
\end{bmatrix},
\\&
\zv_2=\begin{bmatrix}z_{21}\\z_{22}\\z_{23}\end{bmatrix}
=\begin{bmatrix}
{ -4/3 \eps^{3} \diff_{x}^{3} \epsilon ^{3}+O\big(\varepsilon ^{5}
\big)
}\\[1ex]
{ 8/9 \eps^{4} \diff_{x}^{4} \epsilon ^{4}-4/3 \eps^{2} \diff_{x}^{
2} \epsilon ^{2}+O\big(\varepsilon ^{5}\big)+1
}\\[1ex]
{ \eps^{3} \diff_{x}^{3} \epsilon ^{3}-\eps \diff_{x} \epsilon +O
\big(\varepsilon ^{5}\big)
}
\end{bmatrix}.
\end{align*}

