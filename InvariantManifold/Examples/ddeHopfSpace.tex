%!TEX root = ../diverseExamples.tex
\subsection{\texttt{ddeHopfSpace}: DDE with Hopf bifurcation and spatial structure} 
\label{ddeHopfSpace}
\localtableofcontents

Model a delayed `logistic' advection-diffusion \pde\ system in one variable with
\begin{equation*}
\de tu=-[1+u(t)]u(t-\pi/2)-A\D x{u(t)}+D\DD x{u(t)}\,,
\end{equation*}
for slow variations in space~\(x\).

Start by loading the procedure.
\begin{reduce}
in_tex "../invariantManifold.tex"$
\end{reduce}
In the printed output, group terms with like powers of amplitudes~\(s_j\), the complex exponential, and the parameter~\(a\).
\begin{reduce}
factor s,exp,A,D,df,small;
\end{reduce}
Execute the construction of the slow manifold for this system (ignore the warning messages about \verb|u1| declared, and then already defined, as an operator).
\begin{reduce}
invariantmanifold({x},
    mat(( -(1+u1)*u1(pi/2) -A*pdf(u1,x) +D*pdf(u1,x,x) )),
    mat((i,-i)),
    mat((1),(1)),
    mat((1),(1)),
    3)$
end;
\end{reduce}
The marginal modes are~\(e^{\pm it}\) so nominate the frequencies~\(\pm 1\).
The eigenvectors are just~\(1\cdot e^{\pm it}\). 
Because for delay differential equations the time dependence~\(e^{\pm i\omega t}\) is an integral part of the definition of the eigenvector; hence the coded eigenvectors can be the same, as here, because they are differentiated through the time dependence~\(e^{\pm i\omega t}\).

The code works for orders higher than three, but is slow: takes about a minute per iteration.

The procedure actually analyses the embedding system
\begin{equation*}
\de tu=-[1+\eps u(t)]u(t-\pi/2)-A\D xu+D\DD xu\,.
\end{equation*}

\paragraph{The centre manifold} 
These give the location of the invariant manifold in
terms of parameters~\(s\sb j\).
\begin{align*}&
u_{1}=\ParMath{
\exp \big(-i t\big) s_{2}+\exp \big(-2 i t\big) s_{2}^{2} \eps 
\big(1/5 i+2/5\big)+\exp \big(i t\big) s_{1}+\exp \big(2 i t\big) s_{1}
^{2} \eps \big(-1/5 i+2/5\big)+O\big(\varepsilon ^{2}\big)
}
\end{align*}
 
\paragraph{Centre manifold ODEs} 
The system evolves on the invariant manifold such
that the parameters evolve according to these ODEs.
\begin{align*}&
\dot s_{1}=\ParMath{
\frac{\partial ^{2}s_{1}}{\partial x^{2}} A^{2} \eps^{2} 
\big(-6 i \pi ^{4}+8 i \pi ^{2}-\pi ^{5}+12 \pi ^{3}\big)/\big(\pi ^{6}+
12 \pi ^{4}+48 \pi ^{2}+64\big)+\frac{\partial ^{2}s_{1}}{\partial x^{2}
} D \eps^{2} \big(-2 i \pi +4\big)/\big(\pi ^{2}+4\big)+\frac{\partial 
\,s_{1}}{\partial \,x} A \eps \big(2 i \pi -4\big)/\big(\pi ^{2}+4\big)+
s_{2} s_{1}^{2} \eps^{2} \big(-2/5 i \pi ^{5}-12/5 i \pi ^{4}-16/5 i 
\pi ^{3}-96/5 i \pi ^{2}-32/5 i \pi -192/5 i-6/5 \pi ^{5}+4/5 \pi ^{4}-
48/5 \pi ^{3}+32/5 \pi ^{2}-96/5 \pi +64/5\big)/\big(\pi ^{6}+12 \pi ^{4
}+48 \pi ^{2}+64\big)+O\big(\varepsilon ^{3}\big)
}
\\&
\dot s_{2}=\ParMath{
\frac{\partial ^{2}s_{2}}{\partial x^{2}} A^{2} \eps^{2} 
\big(6 i \pi ^{4}-8 i \pi ^{2}-\pi ^{5}+12 \pi ^{3}\big)/\big(\pi ^{6}+
12 \pi ^{4}+48 \pi ^{2}+64\big)+\frac{\partial ^{2}s_{2}}{\partial x^{2}
} D \eps^{2} \big(2 i \pi +4\big)/\big(\pi ^{2}+4\big)+\frac{\partial \,
s_{2}}{\partial \,x} A \eps \big(-2 i \pi -4\big)/\big(\pi ^{2}+4\big)+s
_{2}^{2} s_{1} \eps^{2} \big(2/5 i \pi ^{5}+12/5 i \pi ^{4}+16/5 i \pi 
^{3}+96/5 i \pi ^{2}+32/5 i \pi +192/5 i-6/5 \pi ^{5}+4/5 \pi ^{4}-48/5 
\pi ^{3}+32/5 \pi ^{2}-96/5 \pi +64/5\big)/\big(\pi ^{6}+12 \pi ^{4}+48 
\pi ^{2}+64\big)+O\big(\varepsilon ^{3}\big)
}
\\[2ex]&
\dot s_{1}=\ParMath{
\frac{\partial ^{2}s_{1}}{\partial x^{2}} A^{2} \eps^{2} 
\big(-0.1895 i+0.02476\big)+\frac{\partial ^{2}s_{1}}{\partial x^{2}
} D \eps^{2} \big(-0.453 i+0.2884\big)+\frac{\partial \,s_{1}}{
\partial \,x} A \eps \big(0.453 i-0.2884\big)+s_{2} s_{1}^{2} \eps^{2
} \big(-0.2636 i-0.2141\big)+O\big(\varepsilon ^{3}\big)
}
\\&
\dot s_{2}=\ParMath{
\frac{\partial ^{2}s_{2}}{\partial x^{2}} A^{2} \eps^{2} 
\big(0.1895 i+0.02476\big)+\frac{\partial ^{2}s_{2}}{\partial x^{2}}
 D \eps^{2} \big(0.453 i+0.2884\big)+\frac{\partial \,s_{2}}{
\partial \,x} A \eps \big(-0.453 i-0.2884\big)+s_{2}^{2} s_{1} \eps^{
2} \big(0.2636 i-0.2141\big)+O\big(\varepsilon ^{3}\big)
}
\end{align*}


