%!TEX root = ../diverseExamples.tex
\subsection{\texttt{timeVaryingForcing}: slowly varying in
time forcing of a spectral sub-centre manifold} 
\label{timeVaryingForcing}
\localtableofcontents

\cref{forcedNonlinNormMode} constructed a spectral
submanifold of a system forced by sinusoidal factors~\(\cos
t\) and~\(\sin(t/2)\). Our modelling code can also model
arbitrary forcing provided the forcing is varying slowly
enough in time.

Thus let's modify the system of \cref{forcedNonlinNormMode} to
\begin{align*}
&\dot x_1=x_3\,,
&&\dot x_3=-2x_1+x_2-\frac12x_1^3+\frac3{10}(-x_3+x_4)+F_1(t)\,,
\\&\dot x_2=x_4\,,
&&\dot x_4=x_1-2x_2+\frac3{10}(x_3-2x_4)F_2(t).
\end{align*}
where \(F_1(t)\) is the strength of a direct forcing, and
\(F_2(t)~\)is the strength of a parametric variation. A
linearisation of this system at the origin has modes with
frequencies \(\omega=1,\sqrt3\), corresponding eigenvalues
\(\lambda=\pm i,\pm i\sqrt3\), and corresponding
eigenvectors \((1,1,\pm i\omega,\pm i\omega)\). 
Corresponding  eigenvectors of the adjoint are \((1,1,\pm
i,\pm i)\) and \((\mp i\omega,\pm i\omega,1,-1)\).   We
construct the spectral sub-centre manifold
\cite[e.g.,][\S7]{Sijbrand85} of the nonlinear forced
dynamics of the frequency one mode.


Start by loading the procedure.
\begin{reduce}
in_tex "../invariantManifold.tex"$
\end{reduce}
In the printed output, group terms with like powers of
amplitudes~\(s_j\), and with like effects in the forcing and
its time-derivatives.
\begin{reduce}
factor df,F_1,F_2,small;
depend F_1,t; depend F_2,t;
\end{reduce}
To encode that the time-variation of the forcing is somehow
slow, we need to truncate the analysis to some order of
time-derivatives of the forcing~\(F_j\).  Here \emph{choose}
to truncate by neglecting third and higher order
derivatives.
\begin{reduce}
let { df(F_1,t,3)=>0, df(F_2,t,3)=>0 };
\end{reduce}
These are truncations additional to that of the procedure
call that specifies also truncating to errors~\Ord{\eps^5}.

The following code makes the linear damping to be
effectively small (which then makes it \verb|small|
squared); consequently, also scale the smallness of the
cubic nonlinearity to match.
\begin{reduce}
invariantmanifold({},
    mat(( u3, u4,
      -2*u1+u2-small*u1^3/2+small*3/10*(-u3+u4)+small*F_1,
      u1-2*u2+small*3/10*(u3-2*u4)*F_2 )),
    mat(( i,-i )),
    mat( (1,1,+i,+i), (1,1,-i,-i) ),
    mat( (1,1,+i,+i), (1,1,-i,-i) ),
    5 )$
end;
\end{reduce}

The procedure reports that it actually analyses the
embedding system
\begin{align*}&
\dot u_{1}=u\_3\,,
\qquad
\dot u_{2}=u\_4\,,
\\&
\dot u_{3}=F\_1 \eps^{2} +\eps^{2} \big(-1/2 u\_1^{3}-3/
10 u\_3+3/10 u\_4\big)-2 u\_1+u\_2\,,
\\&
\dot u_{4}=F\_2 \eps^{2} \big(3/10 u\_3-3/5 u\_4\big)+u\_1-2 u\_2\,.
\end{align*}



\paragraph{The invariant manifold} 
Here these give the reparametrisation of the state
space~\uv\ in terms of parameters~\(s_1,s_2\), via rotating
basis vectors. Here, the coordinate transform is
complicated.   Interesting effects are the those on the
shape of the spectral sub-centre manifold of the
forcings~\(F_j\) and their `slow' time-derivatives.
Quadratic effects would be seen here at order~\(\eps^4\).
\begin{align*}&
u_{1}=\ParMath{
-5/9 \frac{\partial ^{2}F\_1}{\partial t^{2}} \eps^{2}+\frac{
\partial ^{2}F\_2}{\partial t^{2}} \eps^{2} \big(33/320 \exp \big(-i t
\big) s_{2} i-33/320 \exp \big(i t\big) s_{1} i\big)+\frac{\partial \,
F\_2}{\partial \,t} \eps^{2} \big(3/32 \exp \big(-i t\big) s_{2}+3/32 
\exp \big(i t\big) s_{1}\big)+2/3 F\_1 \eps^{2}+F\_2 \eps^{2} \big(-3/80
 \exp \big(-i t\big) s_{2} i+3/80 \exp \big(i t\big) s_{1} i\big)+\eps^{
2} \big(-9/16 \exp \big(-i t\big) s_{2}^{2} s_{1}+7/96 \exp \big(-3 i t
\big) s_{2}^{3}-9/16 \exp \big(i t\big) s_{2} s_{1}^{2}+7/96 \exp \big(3
 i t\big) s_{1}^{3}\big)+\exp \big(-i t\big) s_{2}+\exp \big(i t\big) s_
{1}+O\big(\varepsilon ^{4}\big)
}\\&
u_{2}=\ParMath{
-4/9 \frac{\partial ^{2}F\_1}{\partial t^{2}} \eps^{2}+\frac{
\partial ^{2}F\_2}{\partial t^{2}} \eps^{2} \big(-39/320 \exp \big(-i t
\big) s_{2} i+39/320 \exp \big(i t\big) s_{1} i\big)+\frac{\partial \,
F\_2}{\partial \,t} \eps^{2} \big(-9/160 \exp \big(-i t\big) s_{2}-9/160
 \exp \big(i t\big) s_{1}\big)+1/3 F\_1 \eps^{2}+F\_2 \eps^{2} \big(9/80
 \exp \big(-i t\big) s_{2} i-9/80 \exp \big(i t\big) s_{1} i\big)+\eps^{
2} \big(3/16 \exp \big(-i t\big) s_{2}^{2} s_{1}-1/96 \exp \big(-3 i t
\big) s_{2}^{3}+3/16 \exp \big(i t\big) s_{2} s_{1}^{2}-1/96 \exp \big(3
 i t\big) s_{1}^{3}\big)+\exp \big(-i t\big) s_{2}+\exp \big(i t\big) s_
{1}+O\big(\varepsilon ^{4}\big)
}\\&
u_{3}=\ParMath{
2/3 \frac{\partial \,F\_1}{\partial \,t} \eps^{2}+\frac{\partial 
^{2}F\_2}{\partial t^{2}} \eps^{2} \big(63/320 \exp \big(-i t\big) s_{2}
+63/320 \exp \big(i t\big) s_{1}\big)+\frac{\partial \,F\_2}{\partial \,
t} \eps^{2} \big(-21/160 \exp \big(-i t\big) s_{2} i+21/160 \exp \big(i 
t\big) s_{1} i\big)+F\_2 \eps^{2} \big(-9/80 \exp \big(-i t\big) s_{2}-9
/80 \exp \big(i t\big) s_{1}\big)+\eps^{2} \big(3/16 \exp \big(-i t\big)
 s_{2}^{2} s_{1} i-7/32 \exp \big(-3 i t\big) s_{2}^{3} i-3/16 \exp 
\big(i t\big) s_{2} s_{1}^{2} i+7/32 \exp \big(3 i t\big) s_{1}^{3} i
\big)-\exp \big(-i t\big) s_{2} i+\exp \big(i t\big) s_{1} i+O\big(
\varepsilon ^{4}\big)
}\\&
u_{4}=\ParMath{
1/3 \frac{\partial \,F\_1}{\partial \,t} \eps^{2}+\frac{\partial 
^{2}F\_2}{\partial t^{2}} \eps^{2} \big(-57/320 \exp \big(-i t\big) s_{2
}-57/320 \exp \big(i t\big) s_{1}\big)+\frac{\partial \,F\_2}{\partial 
\,t} \eps^{2} \big(27/160 \exp \big(-i t\big) s_{2} i-27/160 \exp \big(i
 t\big) s_{1} i\big)+F\_2 \eps^{2} \big(3/80 \exp \big(-i t\big) s_{2}+3
/80 \exp \big(i t\big) s_{1}\big)+\eps^{2} \big(-9/16 \exp \big(-i t
\big) s_{2}^{2} s_{1} i+1/32 \exp \big(-3 i t\big) s_{2}^{3} i+9/16 \exp
 \big(i t\big) s_{2} s_{1}^{2} i-1/32 \exp \big(3 i t\big) s_{1}^{3} i
\big)-\exp \big(-i t\big) s_{2} i+\exp \big(i t\big) s_{1} i+O\big(
\varepsilon ^{4}\big)
}
\end{align*}

 
\paragraph{Invariant manifold ODEs} 
The system evolves according to these ODEs that characterise
how the modulation of the oscillations evolve in state space
due to nonlinearity and the forcing.   Both linear and
quadratic effects in the forcing and their time-derivatives
are found.
\begin{align*}&
\dot s_{1}=\ParMath{
-531/12800 \frac{\partial ^{2}F\_2}{\partial t^{2}} F\_2 \eps
^{4} s_{1} i+\frac{\partial ^{2}F\_2}{\partial t^{2}} \eps^{4} \big(99/
2560 s_{2} s_{1}^{2}+9/320 s_{1} i\big)+9/256 \frac{\partial \,F\_2}{
\partial \,t} F\_2 \eps^{4} s_{1}+\frac{\partial \,F\_2}{\partial \,t} 
\eps^{4} \big(27/256 s_{2} s_{1}^{2} i-9/400 s_{1}\big)+9/640 F\_2^{2} 
\eps^{4} s_{1} i+F\_2 \eps^{4} \big(-9/80 s_{2} s_{1}^{2}-9/800 s_{1} i
\big)-3/40 F\_2 \eps^{2} s_{1}+\eps^{4} \big(-155/256 s_{2}^{2} s_{1}^{3
} i+9/160 s_{2} s_{1}^{2}\big)+3/8 \eps^{2} s_{2} s_{1}^{2} i+O\big(
\varepsilon ^{5}\big)
}\\&
\dot s_{2}=\ParMath{
531/12800 \frac{\partial ^{2}F\_2}{\partial t^{2}} F\_2 \eps
^{4} s_{2} i+\frac{\partial ^{2}F\_2}{\partial t^{2}} \eps^{4} \big(99/
2560 s_{2}^{2} s_{1}-9/320 s_{2} i\big)+9/256 \frac{\partial \,F\_2}{
\partial \,t} F\_2 \eps^{4} s_{2}+\frac{\partial \,F\_2}{\partial \,t} 
\eps^{4} \big(-27/256 s_{2}^{2} s_{1} i-9/400 s_{2}\big)-9/640 F\_2^{2} 
\eps^{4} s_{2} i+F\_2 \eps^{4} \big(-9/80 s_{2}^{2} s_{1}+9/800 s_{2} i
\big)-3/40 F\_2 \eps^{2} s_{2}+\eps^{4} \big(155/256 s_{2}^{3} s_{1}^{2}
 i+9/160 s_{2}^{2} s_{1}\big)-3/8 \eps^{2} s_{2}^{2} s_{1} i+O\big(
\varepsilon ^{5}\big)
}
\end{align*}


