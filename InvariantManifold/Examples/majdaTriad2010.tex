%!TEX root = ../diverseExamples.tex
% inspired by Falasca (2025) 2506.22552v1, section III.A and
% appendix B.2.  AJR, 8 Jul 2025 -- 25 Feb 2026
\subsection{\texttt{majdaTriad2010}: project stochastic
forcing in Majda's triad model} 
\label{majdaTriad2010}
\localtableofcontents


\cite{Falasca2025}  investigated averaging in the 3D \sde\
system of \cite{Majda2010}. Let's see how the deterministic
slow manifold projection matches their heuristic stochastic
normal form: (B2) by \cite{Falasca2025} and/or (47) by 
\cite{Majda2010}.

Start by loading the procedure.
\begin{reduce}
in_tex "../invariantManifold.tex"$
\end{reduce}
The system uses parameters~\(L_{1j}\) and \(\gamma_j\) so
define the following: \verb|clear gamma| because by default
it is the gamma function.
\begin{reduce}
off raise,lower;
operator L; 	defindex L(down);
clear gamma;
operator gamma; defindex gamma(down);
\end{reduce}

The additive triad model of \cite{Falasca2025,Majda2010}
consists of three modes, $x_1$, $x_2$~and~$x_3$, evolving in
time according to the stochastic \ode{}s
\begin{align*}&
	\frac{dx_1}{dt'}=L_1x_1+L_2x_2+L_3x_3+Ix_2x_3\,,
	\\&
	\frac{dx_2}{dt'}=-\gamma_2\epsilon^{-1}x_2-L_2x_1-Ix_1^2
	+\sigma_2\epsilon^{-1/2}\dot W_2\,,
	\\&
	\frac{dx_3}{dt'}=-\gamma_3\epsilon^{-1}x_3-L_3x_1
	+\sigma_3\epsilon^{-1/2}\dot W_3\,,
\end{align*}
where $L_j$~and~$\sigma_j$ are some constants, and there is
independent stochastic forcing of the second and third
modes.

With its divisions by~\(\epsilon\), the system is written in
singular perturbation form. Unfortunately, singular
perturbations often `sweep-under-the-carpet' key physics
that occur at finite~\(\epsilon\). So first rescale time to
the fast time \(t=t'/\epsilon\) and define \(\verb|small| =
\varepsilon := \sqrt{\epsilon}\) for convenience. In terms
of new dependent variables \(u_j:=x_j\) the above triad
system becomes
\begin{align*}&
	\frac{du_1}{dt} = \varepsilon^2( L_1u_1+L_2u_2+L_3u_3+Iu_2u_3 ),
	\\&
	\frac{du_2}{dt} = -\gamma_2u_2 +\varepsilon^2( -L_2u_1-Iu_1^2 )
	+\sigma_2\dot W_2\,,
	\\&
	\frac{du_3}{dt} = -\gamma_3u_3 -\varepsilon^2L_3u_1
	+\sigma_3\dot W_3\,,
\end{align*}


Because of their decay rates~\(\gamma_2,\gamma_3\), over
long time scales we expect the second and third modes to be
essentially negligible and the system to be modelled by the
relatively slow evolution of the first mode. However, the
stochastic forcing excites these modes and we turn to the
projection vectors to determine how they then affect the
slow manifold model.


The procedure \verb|invariantmanifold| is previously loaded.
Write a message saying we are now analysing the next system.
\begin{reduce}
write "**** Additive Triad system of Majda (2010) ****";
\end{reduce}
Construct a slow manifold for the deterministic system (no
stochastic noise): terms in \(\verb|small|=\varepsilon\) get
multiplied by another factor of \(\verb|small|=\varepsilon\)
before analysis, so these explicit \verb|small| factors
become equivalent to~\(\epsilon\) factors;
\begin{reduce}
factor small;
invariantmanifold({},
    mat(( small*( L(1)*u1+L(2)*u2+L(3)*u3 +I*u1*u2 ),
     -gamma(2)*u2 +small*( -L(2)*u1-I*u1^2 ),
     -gamma(3)*u3 -small*L(3)*u1 )),
    mat(( 0 )),
    mat( (1,0,0) ),
    mat( (1,0,0) ),
    6 )$
end;
\end{reduce}

The procedure reports that it analyses the following family 
\begin{align*}&
\dot u_{1}=\eps^{2} \big(L_{3} u\_3+L_{2} u\_2+L_{1} u\_1+I u\_1 u\_2\big)
\\&
\dot u_{2}=\eps^{2} \big(-L_{2} u\_1-I u\_1^{2}\big)-\gamma _{2} u\_2
\\&
\dot u_{3}=-\eps^{2} L_{3} u\_1-\gamma _{3} u\_3
\end{align*}
Here, \(\eps^2\)\ counts the order of perturbation
in~\(\epsilon\).

\paragraph{The invariant manifold}
These give the location of the invariant manifold in terms
of parameters~\(s\_ 1\) and \(\epsilon=\eps^2\).
\begin{align*}&
u_{1}=s_{1}+O\big(\varepsilon ^{5}\big)
\\&
u_{2}=\ParMath{ \eps^{2} \big(-L_{2} \gamma _{2}^{-1} s_{1}
-\gamma _{2}^{-1} s_{1}^{2} I\big) +\eps^{4} \big(L_{2}
L_{1} \gamma _{2}^{-2} s_{1} +2 L_{1} \gamma _{2}^{-2}
s_{1}^{2} I\big) +O\big(\varepsilon ^{5}\big)
}\\&
u_{3}=\ParMath{ -\eps^{2} L_{3} \gamma _{3}^{-1} s_{1}
+\eps^{4} L_{3} L_{1} \gamma _{3}^{-2} s_{1}
+O\big(\varepsilon ^{5}\big)
}
\end{align*}
\paragraph{Invariant manifold ODEs}
The system evolves on the invariant manifold such that the
parameters evolve according to this ODE.
\begin{equation*}
\dot s_{1}=\ParMath{ \eps^{2} L_{1} s_{1} +\eps^{4}
\big(-L_{3}^{2} \gamma _{3}^{-1} s_{1}-L_{2}^{2} \gamma
_{2}^{-1} s_{1}-2 L_{2} \gamma _{2}^{-1} s_{1}^{2} I-\gamma
_{2}^{-1} s_{1}^{3} I^{2}\big) +O\big(\varepsilon ^{6}\big)
}
\end{equation*}
These match the deterministic part of \cite{Falasca2025}'s
(B2) and \cite{Majda2010}'s (47)---except of course for 
the quadratic noise-noise induced drifts in his It\^o~form.

The stochastic parts of (B2) and/or (47) now come from the 
projection of forcing onto the above deterministic
slow manifold.

\paragraph{Normals to isochrons at the slow manifold}
Use these vectors: to project initial conditions onto the
slow manifold; to project non-autonomous forcing onto the
slow evolution; to predict the consequences of modifying the
original system; in uncertainty quantification to quantify
effects on the model of uncertainties in the original
system. The normal vector 
\begin{align*}&
\vec z :=
\begin{bmatrix} 
1 +\eps^{4} \big(L_{3}^{2} \gamma _{3}^{-2} +L_{2}^{2}
\gamma _{2}^{-2} +3 L_{2} \gamma _{2}^{-2} s_{1} I +2 \gamma
_{2}^{-2} s_{1}^{2} I^{2}\big) +O\big(\varepsilon ^{6}\big)
\\[1ex]
+\eps^{2} \big(L_{2} \gamma _{2}^{-1}+\gamma _{2}^{-1} s_{1}
I\big) -\eps^{4} L_{2} L_{1} \gamma _{2}^{-2}
+O\big(\varepsilon ^{6} \big)
\\[1ex]
+\eps^{2} L_{3} \gamma _{3}^{-1} -\eps^{4} L_{3} L_{1}
\gamma _{3}^{-2} +O\big(\varepsilon ^{6}\big)
\end{bmatrix}
\\&\quad{}=
\begin{bmatrix} 
1+O\big(\varepsilon ^{4}\big)
\\[1ex]
+\eps^{2} \big(L_{2} \gamma _{2}^{-1}+\gamma _{2}^{-1} s_{1}
I\big) +O\big(\varepsilon ^{4} \big)
\\[1ex]
+\eps^{2} L_{3} \gamma _{3}^{-1} +O\big(\varepsilon
^{4}\big)
\end{bmatrix}.
\end{align*}
Consequently, the leading effect of the stochastic forcing
\(\vec f:=(0, \sigma_2\dot W_2, \sigma_3\dot W_3)\) is to
add the following stochastic terms to the \(s_1\)~\ode: 
namely, and agreeing with \cite{Falasca2025}'s (B2),
\begin{equation*}
+\eps^{2} \big(L_{2} \gamma _{2}^{-1}+\gamma _{2}^{-1} s_{1}
I\big)\sigma_2\dot W_2 +\eps^{2} L_{3} \gamma
_{3}^{-1}\sigma_3\dot W_3\,.
\end{equation*}

\paragraph{Summary}  The derived projection vectors also
determine how stochastic forcing effects are mapped onto a
slow manifold, to linear effects in the noise.
