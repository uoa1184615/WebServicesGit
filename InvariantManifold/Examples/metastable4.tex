%!TEX root = ../allExamples.tex
\subsection{\texttt{metastable4}: Metastability in a four state Markov chain} 
\label{ss:metastable4}

Variable \(\epsilon\) characterises the rate of exchange between metastable states~\(u_1\) and~\(u_4\) in this system.
\begin{align*}
&\dot u_{1}=+u_{2} -\epsilon  u_{1} \,,
\\&\dot u_{2}=-u_{2}+\epsilon  (u_{3}-u_{2}+u_{1}),
\\&\dot u_{3}=-u_{3}+\epsilon  (u_{4}-u_{3}+u_{2}),
\\&\dot u_{4}=+u_{3}-\epsilon  u_{4}\,.
\end{align*}

Start by loading the procedure.
\begin{reduce}
in_tex "../invariantManifold.tex"$
\end{reduce}
Execute the construction of the slow manifold for this system.
The explicit parameter~\verb|small|, math-name~\(\eps\), gets replaced by~\verb|small^2| in the code, so in effect \(\eps^2=\epsilon\)\,.
\begin{reduce}
invariantmanifold(
    mat(( u2-small*u1,
         -u2+small*(u1-u2+u3),
         -u3+small*(u2-u3+u4),
          u3-small*u4 )),
    mat((0,0)),
    mat((1,0,0,0),(0,0,0,1)),
    mat((1,1,0,0),(0,0,1,1)),
    6 )$
end;
\end{reduce}
The matrix~\(\begin{bmat} 0&1&0&0 \\ 0&-1&0&0 \\ 0&0&-1&0 \\ 0&0&1&0 \end{bmat}\), of the linearisation about \(\eps=0\), has eigenvalues~\(0\) and~\(-1\) (both multiplicity two). 
We seek the slow manifold so specify the two zero eigenvalues in the second parameter to the procedure.
Corresponding eigenvectors are \(\ev_1=(1,0,0,0)\) and \(\ev_2=(0,0,0,1)\).   
Choosing corresponding left-vector (here not an eigenvector) is \(\zv_1=(1,1,0,0)\) and \(\zv_2=(0,0,1,1)\) means that the slow manifold parameters~\(s_1,s_2\) have the physical meaning, respectively, of being the probability that the system is in states~\(\{1,2\}\) and~\(\{3,4\}\).
The last parameter,~\(6\), specifies to construct the slow manifold to errors~\Ord{\eps^6}, that is, errors~\Ord{\epsilon^3}.


\paragraph{The slow manifold} 
The constructed slow manifold is, in terms of the lumped-state probability parameters~\(s_1,s_2\) (to error~\Ord{\eps^2}, and reverse ordering!), 
\begin{align*}&
u_{1}=\eps^{4} \big(-s_{2}+2 s_{1}\big)-\eps^{2} s_{1}+s_{1}\,, &&
u_{3}=\eps^{4} \big(-2 s_{2}+s_{1}\big)+\eps^{2} s_{2}\,, \\&
u_{2}=\eps^{4} \big(s_{2}-2 s_{1}\big)+\eps^{2} s_{1}\,, &&
u_{4}=\eps^{4} \big(2 s_{2}-s_{1}\big)-\eps^{2} s_{2}+s_{2}\,.
\end{align*}

\paragraph{Slow manifold ODEs} 
On this slow manifold the evolution of the lumped-state probabilities is
\begin{equation*}
\dot s_{1}=\eps^{4} \big(s_{2}-s_{1}\big),\quad
\dot s_{2}=\eps^{4} \big(-s_{2}+s_{1}\big).
\end{equation*}
Hence here the long-term evolution is that on a time-scale of~\Ord{1/\epsilon^2}, \(\Ord{1/\eps^4}\), the system equilibrates between the two lumped states, that is, between~\(\{1,2\}\) and~\(\{3,4\}\).


\paragraph{Normals to isochrons at the slow manifold}
To project initial conditions
onto the slow manifold, or non-autonomous
forcing, or modifications of the original system, or to quantify uncertainty \cite[]{Roberts89b, Roberts97b}, use the projection defined by the derived vector
\begin{equation*}
\zv_1=\begin{bmatrix}z_{11}\\z_{12}\\z_{13}\\z_{14}\end{bmatrix}
=\begin{bmatrix}
\eps^{4}+1\\
4 \eps^{4}-\eps^{2}+1\\
-4 \eps^{4}+\eps^{2}\\
-\eps^{4}
\end{bmatrix},\quad
\zv_2=\begin{bmatrix}z_{21}\\z_{22}\\z_{23}\\z_{24}\end{bmatrix}
=\begin{bmatrix}
-\eps^{4}\\
-4 \eps^{4}+\eps^{2}\\
4 \eps^{4}-\eps^{2}+1\\
\eps^{4}+1
\end{bmatrix}
.
\end{equation*}
Evaluate all these at \(\eps^2=\epsilon\) to apply to the original specified system.


