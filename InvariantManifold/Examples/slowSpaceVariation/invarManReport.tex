\documentclass[11pt,a5paper]{article}
\usepackage[a5paper,margin=13mm]{geometry}
\usepackage{parskip,time,verbatim} \def\_{_}
\raggedright \def\eps{\varepsilon}
\def\diff{\partial\makebox[0pt]{\ $\cdot$}}
\title{Invariant manifold of your dynamical system}
\author{A. J. Roberts, University of Adelaide\\
\texttt{http://orcid.org/0000-0001-8930-1552}}
\date{\now, \today}
\begin{document}
\maketitle
Throughout and generally: the lowest order, most
important, terms are near the end of each expression.
\par\leftskip=2em  \parindent=-2em

\(\)
\paragraph{The specified dynamical system}
\(
\)\par

\(\dot u_{1}=-\alpha  \eps^{2} u\sb3-\eps^{2} u\sb1^{3}-2 \eps u\sb1^{2}-4
 u\sb3
\)\par

\(\dot u_{2}=2 u\sb1-2 u\sb2
\)\par

\(\dot u_{3}=2 u\sb2-2 u\sb3
\)\par

\(\)
\paragraph{Invariant subspace basis vectors}
\(
\)\par

\(\vec e_{1}=\left\{
\left\{
1 , -1/2 i+1/2 , -1/2 i
\right\} , \cis\big(2 t\big)
\right\}
\)\par

\(\vec e_{2}=\left\{
\left\{
1 , 1/2 i+1/2 , 1/2 i
\right\} , \cis\big(-2 t\big)
\right\}
\)\par

\(\vec z_{1}=\left\{
\left\{
1/5 i+2/5 , -2/5 i+1/5 , -3/5 i-1/5
\right\} , \cis\big(2 t\big)
\right\}
\)\par

\(\vec z_{2}=\left\{
\left\{
-1/5 i+2/5 , 2/5 i+1/5 , 3/5 i-1/5
\right\} , \cis\big(-2 t\big)
\right\}
\)\par


off echo;

\(
\)
\paragraph{The invariant manifold}
These give the location of the invariant manifold in
terms of parameters~\(s\_ j\).
\(
\)\par
\(u_{1}=O\big(\varepsilon ^{4}\big)+1/2 s_{2}+1/2 s_{1}
\)\par
\(u_{2}=O\big(\varepsilon ^{4}\big)-1/2 s_{2}+1/2 s_{1}
\)\par
\(
\)
\paragraph{Invariant manifold ODEs}
The system evolves on the invariant manifold such
that the parameters evolve according to these ODEs.
\(
\)\par
\(\dot s_{1}=\eps \big(-\frac{{\rm d}\,s_{2}}{{\rm d}\,z}-2 s_{2}\big)+O
\big(\varepsilon ^{5}\big)
\)\par
\(\dot s_{2}=-\eps \frac{{\rm d}\,s_{1}}{{\rm d}\,z}+O\big(\varepsilon ^{5
}\big)
\)\par
\(
\)
\paragraph{Normals to isochrons at the slow manifold}
Use these vectors: to project initial conditions
onto the slow manifold; to project non-autonomous
forcing onto the slow evolution; to predict the
consequences of modifying the original system; in
uncertainty quantification to quantify effects on
the model of uncertainties in the original system.
The normal vector \(\vec z\_j:=(z\_{j1},\ldots,z\_{jn})\)
\(
\)\par
\(z_{11}=O\big(\varepsilon ^{5}\big)+1
\)\par
\(z_{12}=O\big(\varepsilon ^{5}\big)+1
\)\par
\(z_{21}=O\big(\varepsilon ^{5}\big)+1
\)\par
\(z_{22}=O\big(\varepsilon ^{5}\big)-1
\)\par
\(
\)\par
\end{document}
