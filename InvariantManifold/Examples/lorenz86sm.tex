%!TEX root = ../diverseExamples.tex
\subsection{\texttt{lorenz86sm}: Slow manifold of the Lorenz 1986 atmosphere model} 
\label{lorenz86sm}

In this case we construct the slow sub-centre manifold, analogous to quasi-geostrophy, in order to disentangle the slow dynamics from fast oscillations, analogous to gravity waves, in the \cite{Lorenz86} model.
The normals to the isochrons determine `balancing' onto the slow manifold.
\begin{align*}&
\dot u_{1}=b u_2 u_5- u_2 u_3
\,, \\& 
\dot u_{2}=-b u_1 u_5+ u_1 u_3
\,, \\&
\dot u_{3}=- u_1 u_2
\,, \\&
\dot u_{4}=-u_5
\,, \\&
\dot u_{5}=b u_1 u_2+u_4
\,.
\end{align*}
The parameter~\(b\) controls the interaction between slow and fast waves.
\cref{lorenz86nf} constructs its full state space normal form in order to determine the forcing of the slow modes by the mean fast waves.

Start by loading the procedure.
\begin{reduce}
in_tex "../invariantManifold.tex"$
\end{reduce}
Group output expressions on~\(b\).
\begin{reduce}
factor b;
\end{reduce}
Execute the construction of the slow manifold for this system.
\begin{reduce}
invariantmanifold(
    mat(( -u2*u3+b*u2*u5,
        u1*u3-b*u1*u5,
        -u1*u2,
        -u5,
        +u4+b*u1*u2 )),
    mat(( 0,0,0 )),
    mat( (1,0,0,0,0),(0,1,0,0,0),(0,0,1,0,0) ),
    mat( (1,0,0,0,0),(0,1,0,0,0),(0,0,1,0,0) ),
    4 )$
end;
\end{reduce}
The matrix of the linearisation about the origin has eigenvalues zero (multiplicity three) and~\(\pm i\). 
We seek the slow manifold so specify the eigenvalue zero (thrice) in the second parameter to the procedure.
Since the system is already in linearly separated form, the slow eigenvectors are simply the three given unit vectors.
The last parameter,~\(4\), specifies to construct the slow manifold to errors~\Ord{\eps^4}, that is, to errors~\Ord{|\sv|^5}.

The procedure actually analyses the embedding system
\begin{align*}&
\dot u_{1}=b \eps u_2 u_5-\eps u_2 u_3
\,, \\& 
\dot u_{2}=-b \eps u_1 u_5+\eps u_1 u_3
\,, \\&
\dot u_{3}=-\eps u_1 u_2
\,, \\&
\dot u_{4}=-u_5
\,, \\&
\dot u_{5}=b \eps u_1 u_2+u_4
\,.
\end{align*}
Consequently, here the artificial parameter~\(\eps\) has a physical interpretation in that it counts the nonlinearity: a term in~\(\eps^p\) will be a \((p+1)\)th~order term in~\(\sv\).
%Hence the specified error~\Ord{\eps^3} is here the same as error~\Ord{|\uv|^4} and~\Ord{|\sv|^4}.

\paragraph{The slow manifold} 
The constructed slow manifold is, in terms of the parameters~\sv\ (to errors~\Ord{\eps^3}, and reverse ordering!), 
\begin{align*}&
u_{1}=s_{1}
\,, \\&
u_{2}=s_{2}
\,, \\&
u_{3}=s_{3}
\,, \\&
u_{4}=-b \eps s_{2} s_{1}
\,, \\&
u_{5}=b \eps^{2} \big(-s_{3} s_{2}^{2}+s_{3} s_{1}^{2}\big) .
\end{align*}

\paragraph{Slow manifold ODEs} 
On this slow manifold the evolution is
\begin{align*}&
\dot s_{1}=b^{2} \eps^{3} \big(-s_{3} s_{2}^{3}+s_{3} s_{2} s_{1}^{2}
\big)-\eps s_{3} s_{2}
\,, \\&
\dot s_{2}=b^{2} \eps^{3} \big(s_{3} s_{2}^{2} s_{1}-s_{3} s_{1}^{3}
\big)+\eps s_{3} s_{1}
\,, \\&
\dot s_{3}=-\eps s_{2} s_{1}
\,.
\end{align*}
Here the quadratic terms in~\(s_1,s_2,s_3\) is that of nonlinear slow wave oscillations.  The \(b\)-terms modify these slow waves, reflecting the influence of the fast dynamics (as distinct from the effects of fast waves---these effects are quantified by \cref{lorenz86nf}).


\paragraph{Normals to isochrons at the slow manifold}
To project initial conditions
onto the slow manifold, or non-autonomous
forcing, or modifications of the original system, or to quantify uncertainty \cite[]{Roberts89b, Roberts97b}, use the projection defined by the derived vectors
\begin{align*}&
\zv_1 =\begin{bmatrix}
b^{2} \eps^{2} s_{2}^{2}+1\\
b^{2} \eps^{2} s_{2} s_{1}\\
0\\
b^{3} \eps^{3} \big(s_{2}^{3}-s_{2} s_{1}^{2}\big)+b \eps^{3} 
\big(-s_{2}^{3}+s_{2} s_{1}^{2}\big)+b \eps s_{2}\\
0
\end{bmatrix},\\&
\zv_2 =\begin{bmatrix}
-b^{2} \eps^{2} s_{2} s_{1}\\
-b^{2} \eps^{2} s_{1}^{2}+1\\
0\\
b^{3} \eps^{3} \big(-s_{2}^{2} s_{1}+s_{1}^{3}\big)+b \eps^{3} 
\big(s_{2}^{2} s_{1}-s_{1}^{3}\big)-b \eps s_{1}\\
0
\end{bmatrix},\\&
\zv_3 =\begin{bmatrix}
0\\
0\\
1\\
-4 b \eps^{3} s_{3} s_{2} s_{1}\\
b \eps^{2} \big(-s_{2}^{2}+s_{1}^{2}\big)
\end{bmatrix}.
\end{align*}
Evaluate these at \(\eps=1\) to apply to the original specified system, or here just interpret~\(\eps\) as a way to count the order of each term.


