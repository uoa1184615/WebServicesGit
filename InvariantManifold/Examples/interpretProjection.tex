%!TEX root = ../diverseExamples.tex
\subsection{\texttt{interpretProjection}: Interpret the fast-fibre projection vectors of slow manifolds} 
\label{interpretProjection}
\localtableofcontents

For slow manifolds, this package computes \emph{projection vectors}~\zv.
This example explores \emph{very simple} cases in order to introduce how one uses these vectors.




\subsubsection{Simplest scenario}

Construct the slow manifold of the 2-D system
\begin{equation*}
\de tu=av\,,\qquad \de tv=-v\,,
\end{equation*}
and seek long-time predictions from the initial condition \(u(0)=u_0\C v(0)=v_0\).

Analytically we may solve this system exactly:
\begin{equation*}
v=v_0\e^{-t},\qquad
u=u_0+av_0\big[1-\e^{-t}\big].
\end{equation*}
Since \(u=u_0+av_0+\Ord{\e^{-t}}\) we say that \(u=u_0+av_0\) is the long-time behaviour we aim to predict.

Let's make such prediction via the package.
Start by loading the procedure.
\begin{reduce}
in_tex "../invariantManifold.tex"$
\end{reduce}
Construct the slow manifold for this system using \(u_1=u\) and \(u_2=v\):
\begin{reduce}
invariantmanifold({},
    mat(( a*u2, -u2 )),
    mat((0)), mat((1,0)), mat((1,0)), 9)$
\end{reduce}
The output gives the very simple slow manifold is \(u=u_1=s_1\) and \(v=u_2=0\) in terms of its parameter~\(s_1\), with (non-)evolution \(\de t{s_1}=0\), and the projection vector \(\zv=(1,a)\).

How do we use this projection vector?   Answer: choose initial~\(s_1(0)=s_0\) such that \(\zv\cdot[(u_0,v_0)-\uv(s_0)]=0\)\,.    
Here this equation reduces to \((1,a)\cdot(u_0-s_0,v_0)=0\).
That is, \(s_0:=u_0+av_0\).
Solving the slow manifold evolution \(\de t{s_1}=0\) then gives the prediction that \(u=s_1=u_0+av_0\) for all time.
This is the correct prediction to the exp-decaying error~\Ord{\e^{-t}}. 




\subsubsection{Complex valued parameters}
For simplicity I code do Reduce treats all parameters and variables as real-valued, \emph{not} complex-valued---with the exception of \(\i:=\sqrt{-1}\).   

Thus, for example, let's construct the slow manifold of the 2-D complex-valued system
\begin{equation*}
\de tu=(a+\i b)v\,,\qquad \de tv=-v\,,
\end{equation*}
and seek long-time predictions from the initial condition \(u(0)=u_0\C v(0)=v_0\).
As in the previous subsection, the long-time evolution is simply \(v=0\) and \(u=s=u_0+(a+\i b)v_0\) to errors~\Ord{\e^{-t}}.

Construct the slow manifold for this system:
\begin{reduce}
invariantmanifold({},
    mat(( (a+i*b)*u2, -u2 )),
    mat((0)), mat((1,0)), mat((1,0)), 9)$
\end{reduce}
The output gives the very simple slow manifold is \(u=u_1=s_1\) and \(v=u_2=0\) in terms of its parameter~\(s_1\), with (non-)evolution \(\de t{s_1}=0\), and the projection vector \(\zv=(1,a-\i b)\).

How do we use this projection vector to make the correct initial condition that \(s(0)=s_0=u_0+(a+\i b)v_0\)? 
Answer:  instead of using the dot product, we invoke the complex inner product that \(\left<\zv,\uv\right>:=\zv^*\uv\) where the asterisk denotes the complex-conjugate-transpose. 
That is, as \(a\C b\) are both real, here \(\zv^*\) is the row vector~\(\begin{bmatrix}1& a+\i b\end{bmatrix}\).

Then the specified equation that we project according to the following, \(\left<\zv,(u_0,v_0)-\uv(s_0)\right>=0\)\,, becomes \(u_0+(a+\i b)v_0-s_0=0\)\,, that is, \(s_0=u_0+(a+\i b)v_0\) as required to match the long-time evolution.




\subsubsection{Simplest slow space variation scenario}

Construct the slow manifold of the \pde\ system for two fields \(u(x,t)\C v(x,t)\)
\begin{equation*}
\D tu=a\D xv\,,\qquad \D tv=-v\,,
\end{equation*}
and seek long-time predictions from the initial condition \(u(x,0)=u_0(x)\C v(x,0)=v_0(x)\).

Analytically, the exactly solution (where dash denotes \(\D x{}\)) is
\begin{equation*}
v=v_0(x)\e^{-t},\qquad
u=u_0(x)+av'_0(x)\big[1-\e^{-t}\big].
\end{equation*}
Since \(u=u_0+av'_0+\Ord{\e^{-t}}\) we say that \(u=u_0+av'_0\) is the long-time behaviour we aim to predict.

Let's make such prediction via the package.
Construct the slow manifold for this system using \(u_1=u\) and \(u_2=v\):
\begin{reduce}
invariantmanifold({x},
    mat(( a*pdf(u2,x), -u2 )),
    mat((0)), mat((1,0)), mat((1,0)), 9)$
\end{reduce}
The output gives the very simple slow manifold is \(u=u_1=s_1\) and \(v=u_2=0\) in terms of its parameter~\(s_1(x,t)\), with (non-)evolution \(\D t{s_1}=0\), and the projection vector \(\zv=(1,a\eps\diff_x)\).

How do we use this projection vector?   Answer: first,~\eps\ is a bookkeeping parameter that counts the order of a term, so ignore~\eps\ (or equivalently set \(\eps=1\)).
Second, choose initial~\(s_1(x,0)=s_0(x)\) such that the projection \(\left<\zv,(u_0,v_0)-\uv(s_0)\right>=0\).

This inner product is \emph{not} any integral over space~\(x\); it is done keeping `\(x\)~fixed'.
So here the projection equation reduces to \((1,a\diff_x)\cdot(u_0-s_0,v_0)=0\).
That is, \(s_0:=u_0+av'_0\).
Solving the slow manifold evolution \(\D t{s_1}=0\) then gives the prediction that \(u=s_1=u_0+av'_0\) for all time.
As before, this is a correct prediction to an exp-decaying error~\Ord{\e^{-t}}. 





\subsubsection{Nontrivial slow space variation}

Construct the slow manifold of the \pde\ system for two fields \(u(x,t)\C v(x,t)\) where here spatial gradients of~\(u,v\) feed into the \(v,u\)~\pde{}s:
\begin{equation*}
\D tu=\D xv\,,\qquad \D tv=-v+\D xu\,.
\end{equation*}
We seek long-time predictions from the initial condition \(u(x,0)=u_0(x)\C v(x,0)=v_0(x)\).

Let's predict the long-time evolution via the package.
Construct the slow manifold for this system using \(u_1=u\) and \(u_2=v\):
\begin{reduce}
invariantmanifold({x},
    mat(( pdf(u2,x), pdf(u1,x)-u2 )),
    mat((0)), mat((1,0)), mat((1,0)), 4)$
\end{reduce}
The output gives the slow manifold is \(u=u_1=s_1+\Ord{\diff_x^3}\) and \(v=u_2=\D x{s_1}+\Ord{\diff_x^3}\) in terms of its parameter~\(s_1(x,t)\).
Here the corresponding slow evolution is the non-trivial diffusion \pde\ \(\D t{s_1}=\DD x{s_1}+\Ord{\diff_x^4}\). 

Lastly, the projection vector (neglecting~\eps) is \(\zv=(1 -\diff_x^2 \C \diff_x -2\diff_x^3 ) +\Ord{\diff_x^4}\).
Hence choose initial~\(s_1(x,0)=s_0(x)\) such that the projection \(\left<\zv,(u_0,v_0)-\uv(s_0)\right>=0\).
Here this projection equation reduces to \((1 -\diff_x^2 \C \diff_x -2\diff_x^3)\cdot(u_0-s_0,v_0)=0\).
That is, the initial field \(s_0:=u_0+v'_0-u''_0-2v'''_0+\Ord{\diff_x^4}\).





Finish.
\begin{reduce}
end;
\end{reduce}


\endinput
