%!TEX root = ../diverseExamples.tex
% AJR, 26 Nov 2022;
\subsection{\texttt{marcusYamabe}: Discover Marcus--Yamabe instability}
\label{marcusYamabe}

In {nonautonomous} systems, such as \(\dot{\uv}=L(t)\uv\),
just because eigenvalues of~\(L(t)\) have real-part
negative, for all~\(t\), does not mean that all
solutions~\(\uv(t)\) decay. Here consider the
{Marcus--Yamabe system} \cite[p.197]{Chicone2006}
\begin{equation}\label{eq:oscmyi}
\de t{\uv}=L(t)\uv \quad\text{for }L:=\begin{bmatrix} 
-1+\rat32\varepsilon^2\cos^2 t & 1-\rat32\varepsilon^2\sin t\,\cos t\\
-1-\rat32\varepsilon^2\sin t\,\cos t&-1+\rat32\varepsilon^2\sin^2t
\end{bmatrix}.
\end{equation}
For example, for \(\varepsilon=1\), the eigenvalues
of~\(L(t)\) are \(\rat14(-1\pm\sqrt7i)\) (independent of
time). Despite the eigenvalues having negative real-part,
there are growing solutions \(\uv=(-\cos t,\sin t)e^{t/2}\).

Here analyse the system with the late-2022 version of
\verb|invariantManifold.tex| that caters for sinusoidal
non-autonomous coefficients and forcing.   
\begin{reduce}
in_tex "../invariantManifold.tex"$
factor small;
\end{reduce}

Encode the system with \(\verb|small|=\varepsilon\).  We
find instability predicted when \(\tfrac32\varepsilon^2
>1\); that is, \(|\varepsilon|>0.8165\); for example,
\(\varepsilon=1\) as commented above.  Then the induced
growth of complex amplitudes~\(s_1\) and~\(s_2\) overcomes
the \(e^{-t}\) decay that is in \(u_1 = e^{(-1+i)t}s_1 +
e^{(-1-i)t}s_2\).
\begin{reduce}
invariantmanifold(
    mat((-u1+u2 +small*( 3/2*cos(t)^2*u1 -3/2*cos(t)*sin(t)*u2),
         -u1-u2 +small*(-3/2*cos(t)*sin(t)*u1 +3/2*sin(t)^2*u2)
         )),
    mat(( -1+i, -1-i )),
    mat( (1,i), (1,-i) ),
    mat( (1,i), (1,-i) ),
    9)$
end;
\end{reduce}
The function finds the following exact time-dependent
transformation of this linear system. These parameterise
state space in terms of~\(s\sb j\):
\begin{align*}&
u_{1}=\exp \big(-i t-t\big) s_{2}+\exp \big(i t-t\big) s_{1}+O\big(
\varepsilon ^{8}\big)
\\&
u_{2}=-\exp \big(-i t-t\big) s_{2} i+\exp \big(i t-t\big) s_{1} i+O\big(
\varepsilon ^{8}\big)
\end{align*}
Then the system evolves in state space such that the
parameters evolve according to these \ode{}s.
\begin{align*}&
\dot s_{1}=\eps^{2} \big(3/4 s_{2}+3/4 s_{1}\big)+O\big(\varepsilon ^{9}
\big)
\\&
\dot s_{2}=\eps^{2} \big(3/4 s_{2}+3/4 s_{1}\big)+O\big(\varepsilon ^{9}\big)
\end{align*}
The eigenvalues of the above system are \(\lambda =0,
\tfrac32\varepsilon^2\).  Hence the net growth of~\uv\ is at
rate~\(-1+\tfrac32\varepsilon^2\); for example, at the
unstable rate~\(+1/2\) when \(\varepsilon=1\).

