%!TEX root = ../diverseExamples.tex
\subsection{\texttt{randWalkHetero}: including spatial heterogeneity} 
\label{randWalkHetero}
\localtableofcontents

What if the governing \pde{}s have heterogeneous coefficients?    We can still analyse such systems provided the heterogeneity is also slowly-varying in space, that is, if the underlying system has functional graduations.

Here introduce the technique via a modification of the random walker in 2D of \cref{randWalkIn2D}.   Let the West walking be done a speeds that vary in \(x_1x_2\)-space, say at speed \(w(x_1,x_2)\).

Let \(u_j(x_1,x_2,t)\) be the probability of the walker
being at position~\((x_1,x_2)\) and walking in the
\(j\)th~of the mentioned directions. Then the
non-dimensional \pde{}s for these probabilities may be
\begin{align*}
&\D t{u_1}=-\D {x_1}{u_1}-\D {x_2}{u_1}-u_1+u_2\,,
\\&\D t{u_2}=+w(x_1,x_2)\D {x_1}{u_2}+u_1-2u_2+u_3\,,
\\&\D t{u_3}=-\D {x_1}{u_3}+\D{x_2}{u3}+u_2-u_3\,.
\end{align*}
We code the equivalent system obtained by defining a new variable \(u_4:=w\) such that \(\D t{u_4}=0\) so that \(u_4\)~may vary in space but not in time:
\begin{align*}
&\D t{u_1}=-\D {x_1}{u_1}-\D {x_2}{u_1}-u_1+u_2\,,
\\&\D t{u_2}=+u_4\D {x_1}{u_2}+u_1-2u_2+u_3\,,
\\&\D t{u_3}=-\D {x_1}{u_3}+\D{x_2}{u3}+u_2-u_3\,,
\\&\D t{u_4}=0\,.
\end{align*}
Then in the output, \(w=u_4=s_2\).

Start by loading the procedure.
\begin{reduce}
in_tex "../invariantManifold.tex"$
\end{reduce}
In the printed output, group terms depending upon order of
spatial derivatives (which are assumed `small').
\begin{reduce}
factor small;
\end{reduce}
The following procedure call constructs the slow manifold
for this system to errors~\Ord{\grad^5}.
\begin{reduce}
invariantmanifold( {x_1,x_2},
    mat((-pdf(u1,x_1)-pdf(u1,x_2)+(u2-u1)
        ,+u4*pdf(u2,x_1)         +(u1-2*u2+u3)
        ,-pdf(u3,x_1)+pdf(u3,x_2)+(u2-u3)     
        ,0)),
    mat(( 0,0 )),
    mat( (1/3,1/3,1/3,0),(0,0,0,1) ),
    mat( (1,1,1,0),(0,0,0,1) ),
    3 )$
end;
\end{reduce}


The procedure then actually analyses the parametrised system
\begin{align*}&
\D t{u_{1}}=\eps \big(-\frac{\partial\,u_1}{\partial\,x_1}-\frac{\partial
\,u_1}{\partial\,x_2}\big)-u_1+u_2\,,
\\&
\D t{u_{2}}=\eps u_4\frac{\partial\,u_2}{\partial\,x_1}+u_1-2 u_2+u_3\,.
\\&
\D t{u_{3}}=\eps \big(-\frac{\partial\,u_3}{\partial\,x_1}+\frac{\partial
\,u_3}{\partial\,x_2}\big)+u_2-u_3\,.
\end{align*}
Consequently the procedure's artificial parameter~\(\eps\)\
counts the number of spatial derivatives in each term.


\paragraph{The invariant manifold} 
Five iterations constructs the slow manifold model. The slow
manifold is expressed in terms of a series in space
derivatives.
\begin{align*}&
u_{1}=\ParMath{ \eps \big(-1/27 \frac{\partial \,s_{1}}{\partial \,x_1} s_{2}-1/
27 \frac{\partial \,s_{1}}{\partial \,x_1}-1/3 \frac{\partial \,s_{1}}{
\partial \,x_2}\big)+O\big(\varepsilon ^{2}\big)+1/3 s_{1}
}\\&
u_{2}=\ParMath{ \eps \big(2/27 \frac{\partial \,s_{1}}{\partial \,x_1} s_{2}+2/27
 \frac{\partial \,s_{1}}{\partial \,x_1}\big)+O\big(\varepsilon ^{2}
\big)+1/3 s_{1}
}\\&
u_{3}=\ParMath{ \eps \big(-1/27 \frac{\partial \,s_{1}}{\partial \,x_1} s_{2}-1/
27 \frac{\partial \,s_{1}}{\partial \,x_1}+1/3 \frac{\partial \,s_{1}}{
\partial \,x_2}\big)+O\big(\varepsilon ^{2}\big)+1/3 s_{1}
}\\&
u_4=\ParMath{ O\big(\varepsilon ^{2}\big)+s_{2}
}\end{align*}

 
\paragraph{Invariant manifold PDEs} 
The system evolves according to this \pde\ that describes
the effective movement of the random walker: an
advection-diffusion \pde, with anisotropic diffusion, and
gradients of \(w=s_2\) included:
\begin{align*}&
\D t{s_{1}}=\ParMath{ \eps^{2} \big(2/27 \frac{\partial \,s_{2}}{\partial \,x_1} 
\frac{\partial \,s_{1}}{\partial \,x_1} s_{2}+2/27 \frac{\partial \,s_{
2}}{\partial \,x_1} \frac{\partial \,s_{1}}{\partial \,x_1}+2/27 
\frac{\partial ^{2}s_{1}}{\partial x_1^{2}} s_{2}^{2}+4/27 \frac{
\partial ^{2}s_{1}}{\partial x_1^{2}} s_{2}+2/27 \frac{\partial ^{2}s_{
1}}{\partial x_1^{2}}+2/3 \frac{\partial ^{2}s_{1}}{\partial x_2^{2}}
\big)+\eps \big(1/3 \frac{\partial \,s_{1}}{\partial \,x_1} s_{2}-2/3 
\frac{\partial \,s_{1}}{\partial \,x_1}\big)+O\big(\varepsilon ^{3}
\big)
}\\&
\D t{s_2}=0
\end{align*}
So, if \(w=s_2<2\) then, on average, the walker will drift in the
\(+x_1\)-direction, but with significant and growing spread
in the \(x_1x_2\)-meadow.


\paragraph{Project initial conditions et al.}
To project initial conditions onto the slow manifold, or
non-autonomous forcing, or modifications of the original
system, or to quantify uncertainty \cite[Ch.12]{Roberts89b,
Roberts97b, Roberts2014a}, use the projection defined by the
following derived vector.   \emph{Warning: this needs checking.}
%:ToDo: check projection vector in such scenario
\begin{align*}&
\zv_1=\begin{bmatrix}z_{11}&z_{12}&z_{13}&z_{14}\end{bmatrix}^T
\\&
=\begin{bmatrix}
\ParMath{ 
\eps^{2} \big(-1/27 \diff_{x_1}^{2} s_{2}^{2}-2/27 \diff_{x_1}
^{2} s_{2}-1/27 \diff_{x_1}^{2}+4/9 \diff_{x_1} \diff_{x_2} s_{2}+4/9
 \diff_{x_1} \diff_{x_2}-5/9 \diff_{x_2}^{2}\big)+\eps \big(-1/9 
\diff_{x_1} s_{2}-1/9 \diff_{x_1}-\diff_{x_2}\big)+O\big(\varepsilon 
^{3}\big)+1
}\\[4ex]\ParMath{
-8/9 \eps^{2} \diff_{x\_2}^{2}+\eps \big(2/9 \diff_{x\_1} s_{2}+2
/9 \diff_{x\_1}\big)+O\big(\varepsilon ^{3}\big)+1 
}\\[2ex]\ParMath{ 
\eps^{2} \big(-1/27 \diff_{x_1}^{2} s_{2}^{2}-2/27 \diff_{x_1}
^{2} s_{2}-1/27 \diff_{x_1}^{2}-4/9 \diff_{x_1} \diff_{x_2} s_{2}-4/9
 \diff_{x_1} \diff_{x_2}-5/9 \diff_{x_2}^{2}\big)+\eps \big(-1/9 
\diff_{x_1} s_{2}-1/9 \diff_{x_1}+\diff_{x_2}\big)+O\big(\varepsilon 
^{3}\big)+1
}\\[4ex]\ParMath{ 
\eps^{2} \big(-2/81 \frac{\partial \,s_{1}}{\partial \,x_1} 
\diff_{x_1} s_{2}-2/81 \frac{\partial \,s_{1}}{\partial \,x_1} \diff_{
x_1}\big)+O\big(\varepsilon ^{3}\big)
}\end{bmatrix}.
\\&
\zv_2=\begin{bmatrix}0&0&0&1\end{bmatrix}^T.
\end{align*}

