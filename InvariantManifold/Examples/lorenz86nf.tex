%!TEX root = ../diverseExamples.tex
\subsection{\texttt{lorenz86nf}: Paradoxically justify a slow manifold despite being proven to not exist} 
\label{lorenz86nf}

\cite{Lorenz86} proposed the following simple system in order to understand aspects of the quasi-geostrophic approximation in atmospheric dynamics.
\begin{align*}&
\dot u_{1}=b u_2 u_5- u_2 u_3
\,, \\& 
\dot u_{2}=-b u_1 u_5+ u_1 u_3
\,, \\&
\dot u_{3}=- u_1 u_2
\,, \\&
\dot u_{4}=-u_5
\,, \\&
\dot u_{5}=b u_1 u_2+u_4
\,.
\end{align*}
The parameter~\(b\) controls the interaction between slow and fast dynamics.
As in \cref{lorenz86sm}, it appears that a slow manifold of quasi-geostrophy exists and is constructible.  
Nonetheless, \cite{Lorenz87} proved that a slow manifold cannot exist for this system!

A resolution of this apparent paradox comes via backwards theory \cite[\S2.5]{Roberts2018a}.
There are systems exponentially close to the above Lorenz86 system (that is, asymptotically the same to all orders in~\(|\uv|\)) which do possess a slow manifold.
Hence the properties that cause the non-existence are exponentially small, they are beyond all orders, and so are likely to be physically irrelevant---they are likely to be smaller than the mathematical modelling errors of the original system.
 
Let's see this resolution by constructing, to any specified order, a system that has a slow manifold and is close to the Lorenz86 system.
We do this by constructing a coordinate transform of the 5D state space.
Start by loading the procedure.
\begin{reduce}
in_tex "../invariantManifold.tex"$
\end{reduce}
Group output expressions on~\(b\).
\begin{reduce}
factor b;
\end{reduce}
Execute the construction of the coordinate transform for this system.
\begin{reduce}
invariantmanifold(
    mat(( -u2*u3+b*u2*u5,
        u1*u3-b*u1*u5,
        -u1*u2,
        -u5,
        +u4+b*u1*u2 )),
    mat(( 0,0,0,i,-i )),
    mat( (1,0,0,0,0),(0,1,0,0,0),(0,0,1,0,0),
        (0,0,0,1,-i), (0,0,0,1,+i) ),
    mat( (1,0,0,0,0),(0,1,0,0,0),(0,0,1,0,0),
        (0,0,0,1,-i), (0,0,0,1,+i) ),
    4 )$
end;
\end{reduce}
The matrix of the linearisation about the origin has eigenvalues zero (multiplicity three) and~\(\pm i\), as specified for the eigenvalues in the second parameter to the procedure.
Corresponding eigenvectors are simply the three unit vectors and the two complex eigenvectors of the fast waves.
The last parameter,~\(4\), specifies to construct the slow manifold to errors~\Ord{\eps^4}, that is, to errors~\Ord{|\sv|^5}.

The procedure actually analyses the embedding system
\begin{align*}&
\dot u_{1}=b \eps u_2 u_5-\eps u_2 u_3
\,, \\& 
\dot u_{2}=-b \eps u_1 u_5+\eps u_1 u_3
\,, \\&
\dot u_{3}=-\eps u_1 u_2
\,, \\&
\dot u_{4}=-u_5
\,, \\&
\dot u_{5}=b \eps u_1 u_2+u_4
\,.
\end{align*}

\paragraph{The coordinate transform} 
The constructed coordinate transform is, in terms of the slow variables~\(\sv\) and a time-dependent basis (to errors~\Ord{\eps^3}, and reverse ordering!), 
\begin{align*}&
u_{1}=\ParMath{
b^{2} \eps^{2} \big(-1/2 \exp \big(-2 i t\big) s_{5}^{2} s_{1}-1/2
 \exp \big(2 i t\big) s_{4}^{2} s_{1}\big)+b \eps \big(-\exp \big(-i t
\big) s_{5} s_{2}-\exp \big(i t\big) s_{4} s_{2}\big)+s_{1}
\,,} \\&
u_{2}=\ParMath{
b^{2} \eps^{2} \big(-1/2 \exp \big(-2 i t\big) s_{5}^{2} s_{2}-1/2
 \exp \big(2 i t\big) s_{4}^{2} s_{2}\big)+b \eps \big(\exp \big(-i t
\big) s_{5} s_{1}+\exp \big(i t\big) s_{4} s_{1}\big)+s_{2}
\,,} \\&
u_{3}=b \eps^{2} \big(\exp \big(-i t\big) s_{5} s_{2}^{2} i-\exp \big(-i
 t\big) s_{5} s_{1}^{2} i-\exp \big(i t\big) s_{4} s_{2}^{2} i+\exp 
\big(i t\big) s_{4} s_{1}^{2} i\big)+s_{3}
\,, \\&
u_{4}=\ParMath{
b^{2} \eps^{2} \big(1/4 \exp \big(-i t\big) s_{5} s_{2}^{2}-1/4 
\exp \big(-i t\big) s_{5} s_{1}^{2}+1/4 \exp \big(i t\big) s_{4} s_{2}^{
2}-1/4 \exp \big(i t\big) s_{4} s_{1}^{2}\big)-b \eps s_{2} s_{1}+\exp 
\big(-i t\big) s_{5}+\exp \big(i t\big) s_{4}
\,,} \\&
u_{5}=\ParMath{
b^{2} \eps^{2} \big(-1/4 \exp \big(-i t\big) s_{5} s_{2}^{2} i+1/4
 \exp \big(-i t\big) s_{5} s_{1}^{2} i+1/4 \exp \big(i t\big) s_{4} s_{2
}^{2} i-1/4 \exp \big(i t\big) s_{4} s_{1}^{2} i\big)+b \eps^{2} \big(-s
_{3} s_{2}^{2}+s_{3} s_{1}^{2}\big)+\exp \big(-i t\big) s_{5} i-\exp 
\big(i t\big) s_{4} i
\,.}
\end{align*}

\paragraph{Transformed ODEs} 
In the variables~\sv\ the evolution is
\begin{align*}&
\dot s_{1}=b^{2} \eps^{3} \big(-s_{3} s_{2}^{3}+s_{3} s_{2} s_{1}^{2}
\big)-\eps s_{3} s_{2}
\,, \\&
\dot s_{2}=b^{2} \eps^{3} \big(s_{3} s_{2}^{2} s_{1}-s_{3} s_{1}^{3}
\big)+\eps s_{3} s_{1}
\,, \\&
\dot s_{3}={\color{red!70!black}2 b^{2} \eps^{3} s_{5} s_{4} s_{2} s_{1}}-\eps s_{2} s_{1}
\,, \\&
\dot s_{4}=b^{2} \eps^{2} \big(-1/2 s_{4} s_{2}^{2} i+1/2 s_{4} s_{1}^{2
} i\big)
\,, \\&
\dot s_{5}=b^{2} \eps^{2} \big(1/2 s_{5} s_{2}^{2} i-1/2 s_{5} s_{1}^{2}
 i\big)
\,.
\end{align*}
When \(s_4=s_5=0\) we recover precisely the same slow manifold as constructed by \cref{lorenz86sm}.
Hence the above system of \(\uv=\cdots\) and \(\dot{\sv}=\cdots\) together both has a slow manifold, and is \Ord{|\sv|^5} close to the original Lorenz86 system.
Such construction can proceed to any order, and so the above closeness of a system with a slow manifold holds to all orders in~\(|\sv|\).

Also of interest is the red term in the \(\dot s_3\) \ode: it shows that the evolution of the slow variables,~\(s_1,s_2,s_3\), is affected by the presence of fast waves, \(s_4,s_5\)~non-zero.
That is, the evolution on and off the slow manifold differ by this term (and similar higher-order terms).
Users of slow models among fast waves need to be aware of this physical feature.




