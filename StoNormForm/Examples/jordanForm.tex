%!TEX root = ../manyExamples.tex
\subsection{\texttt{jordanForm}: the Jordan form of position-momentum variables} 
\label{jordanForm}

Suppose \(x(t)\) is the spatial position of some particle, and you want to analyse the `mechanical' system of \sde{}s 
\begin{equation*}
\ddot x=-xy \qtq{and} \dot y=-2y+x^2+{\dot x}^2+\sigma w(t)\,,
\end{equation*}
where $w(t)$ denotes the formal derivative~$dW/dt$ of a Stratonovich Wiener process~$W(t,\omega)$, or some other time-dependent forcing, called noise.
Introduce position and velocity variables $x_1=x$ and $x_2=\dot x$\,, and also $y_1=y$ to convert to the system of three coupled first-order \sde{}s
\begin{align*}&
\dot x_1=x_2 \,, \\&
\dot x_2=-x_1y_1 \,, \\&
\dot y_1=-2y_1+x_1^2+x_2^2+\sigma w(t)\,.
\end{align*}



Start by loading the procedure.
\begin{reduce}
in_tex "../stoNormForm.tex"$
\end{reduce}
Execute the construction of a normal form for this system.
\begin{reduce}
stonormalform(
    { x(2)/small,
      -x(1)*y(1) },
    { -2*y(1)+x(1)^2+x(2)^2+w(y) },
    {},
    3 ,sman)$
end;
\end{reduce}
Why divide~\verb|x(2)| by~\verb|small|?  A possible coding is to specify the system as given, but recall that the slow \sde{}s are always multiplied by~\verb|small| thus changing the first \sde\ to \(\dot x_1=\eps x_2\) and hence changing the relation between position and velocity---this would be OK if~\(x_2\) was viewed as momentum \emph{and} the particle had large mass.  But what if really do we want \(x_2\) to be velocity.  Fortunately, the coded iteration scheme works for systems with linear part in Jordan form, but one has to code the system as follows. Namely, divide the off-diagonal term of the Jordan form by~\verb|small| to cancel out the procedure's brutal multiplication by~\verb|small|. 

Then the coded procedure reports that it analyses the following system which not only reduces to the given one for \(\eps=1\)\,, but also preserves the physical relation between position~\(x_1\) and velocity~\(x_2\):
\begin{align*}&
\dot x_{1}=x_{2}
\\&
\dot x_{2}=-\eps x_{1} y_{1}
\\&
\dot y_{1}=\sigma  w_{y}+\eps \big(x_{2}^{2}+x_{1}^{2}\big)-2 y_{1}
\end{align*}
Further, here \(\eps\) counts the order of nonlinearity so truncating to errors \Ord{\eps^3} is the same as truncating to errors \Ord{|(\xv,y)|^4}.

The cost of preserving the physical relation between position~\(x_1\) and velocity~\(x_2\) is that more iterations are needed in the construction.




\paragraph{Time dependent coordinate transform}
This transform is quite complicated, due to the noise, and involve fast-time future and history integrals. 
\begin{align*}&
y_{1}=\sigma  \ou\big(w_{y},tt,-2\big)+\eps \big(3/4 X_{2}^{2}-1/2 X_{2}
 X_{1}+1/2 X_{1}^{2}\big)+Y_{1}
\\&
x_{1}=\ParMath{\sigma  \eps \big(-1/4 \ou\big(w_{y},tt,-2\big) X_{2}-1/4 \ou\big(
w_{y},tt,-2\big) X_{1}\big)+\eps \big(-1/4 X_{2} Y_{1}-1/4 X_{1} Y_{1}
\big)+X_{1}
}
\\&
x_{2}=\ParMath{\sigma  \eps \big(1/4 \ou\big(w_{y},tt,-2\big) X_{2}+1/2 \ou\big(w
_{y},tt,-2\big) X_{1}\big)+\eps \big(1/4 X_{2} Y_{1}+1/2 X_{1} Y_{1}
\big)+X_{2}
}
\end{align*}

\paragraph{Result normal form DEs}
\begin{align*}&
\dot Y_{1}=\eps^{2} \big(1/2 X_{2}^{2} Y_{1}+1/2 X_{2} X_{1} Y_{1}-1/2 X
_{1}^{2} Y_{1}\big)-2 Y_{1}
\\&
\dot X_{1}=\ParMath{\sigma ^{2} \eps^{2} \big(-3/64 \ou\big(w_{y},tt,-2\big) w_{y
} X_{2}-3/32 \ou\big(w_{y},tt,-2\big) w_{y} X_{1}\big)+\sigma  \eps 
\big(1/4 w_{y} X_{2}+1/4 w_{y} X_{1}\big)+X_{2}
}
\\&
\dot X_{2}=\ParMath{\sigma ^{2} \eps^{2} \big(3/32 \ou\big(w_{y},tt,-2\big) w_{y}
 X_{2}+1/8 \ou\big(w_{y},tt,-2\big) w_{y} X_{1}\big)+\sigma  \eps \big(-
1/4 w_{y} X_{2}-1/2 w_{y} X_{1}\big)+\eps^{2} \big(-3/4 X_{2}^{2} X_{1}+
1/2 X_{2} X_{1}^{2}-1/2 X_{1}^{3}\big)
}
\end{align*}
\begin{itemize}
\item As expected,  \(\dot Y_1=0\) when \(Y_1=0\)\,, and so \(Y_1=0\) is the stochastic slow manifold, and is exponentially attractive (almost always) in some domain about the origin.
\item As expected, the slow~\Xv\ evolution is independent of~\(Y_1\): \(X_2\)~is approximately a `velocity' variable for `position'~\(X_1\), and shows some nonlinear noise affected dynamics.
\item The time-dependent coordinate transform maps these predictions back into the \(\xv,y\)-space.  Observe that~\Xv\ are not precisely the physical position-veloxity~\xv, but instead are affected by nonlinearity, and the noise, and their interaction.
\item The stochastic slow invariant manifold could be more efficiently constructed by specifying the optional fifth parameter~\verb|cman| to the \verb|stonormalform()| invocation.
\item For this system we can see from the above that also \(\Xv=0\) is invariant.  Hence a stochastic stable manifold exists.  We may construct it directly by specifying the optional fifth parameter~\verb|sman| to \verb|stonormalform()|.
\end{itemize}


