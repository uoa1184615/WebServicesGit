%!TEX root = ../manyExamples.tex
\subsection{\texttt{offdiagonal}: Levy area contraction: off-diagonal example} 
\label{offdiagonal}

\cite{Pavliotis07} assert the following system of five coupled \sde{}s are interesting for various parameters~\(a\) and for small~\(\epsilon\).
\begin{align*}&
dx_1=\epsilon y_1\,dt \,,\\&
dx_2=\epsilon y_2\,dt \,,\\&
dx_3=\epsilon (x_1y_2-x_2y_1)dt \,,\\&
dy_1=(-y_1-a y_2)dt +dW_1 \,,\\&
dy_2=(+a y_1-y_2)dt +dW_2 \,.
\end{align*}
This stochastic system has two noise sources.
We treat~$W_i(t,\omega)$ as Stratonovich Wiener processes.
Use \verb|x(i)| to denote variable~$x_i$, \verb|y(i)| to denote variable~$y_i$, and \verb|w(i)| to denote noise~$dW_i/dt$.



Start by loading the procedure.
\begin{reduce}
in_tex "../stoNormForm.tex"$
\end{reduce}
It is convenient to factor written results on the two given parameters~\(\epsilon,a\):
\begin{reduce}
factor epsilon,a;
\end{reduce}
Execute the construction of a normal form for this system.
A coding is to specify the system as given: specify the slow \sde{}s via a three component list; and the fast stable \sde{}s via a two component list.
\begin{reduce}
stonormalform(
    {epsilon*y(1),
     epsilon*y(2),
     epsilon*(x(1)*y(2)-x(2)*y(1))},
    {-y(1)-a*y(2)+w(1),
     -y(2)+a*y(1)+w(2)},
    {},
    4 )$
end;
\end{reduce}

Now the approach can only analyse systems which are linearly diagonalised, but this system has two off-diagonal terms in the \yv-\sde{}s (terms that cause oscillations in~\yv\ with frequency~\(a\) as \yv~decays in magnitude like~\(e^{-t}\)).  In order to make some sort of progress, the procedure is brutal with such off-diagonal terms.  \emph{Anything linear and off-diagonal is multiplied by the parameter~\texttt{small} and so is treated as asymptotically small.}  When it does so, it gives the warning message
\begin{verbatim}
***** Warning ****
Off diagonal linear terms in y- or z- equations 
assumed small.  Answers are rubbish if not 
asymptotically appropriate. 
\end{verbatim}  
As the message says, the results may consequently be rubbish. 

Here then, the procedure analyses the following system which reduce to the given one for \(\eps=1\)\,:
\begin{align*}&
\dot x_{1}=\epsilon  \eps y_{1}
\\&
\dot x_{2}=\epsilon  \eps y_{2}
\\&
\dot x_{3}=\epsilon  \eps \big(-x_{2} y_{1}+x_{1} y_{2}\big)
\\&
\dot y_{1}=-a \eps y_{2}+\sigma  w_{1}-y_{1}
\\&
\dot y_{2}=a \eps y_{1}+\sigma  w_{2}-y_{2}
\end{align*}
That is, the code treats the frequency parameter~\(a\) as small, and so the results are appropriate only for small~\(a\), as well as only for small~\(\epsilon\).

If one really needs to analyse non-small~\(a\), then more sophisticated code has to be developed.



\paragraph{Time dependent coordinate transform}
This transform is quite complicated, due to the noise, and involve fast-time future and history integrals. 
\begin{align*}&
y_{1}=\ParMath{-a^{2} \sigma  \eps^{2} \ou\big(\ou\big(\ou\big(w_{1},tt,-1\big),
tt,-1\big),tt,-1\big)-a \sigma  \eps \ou\big(\ou\big(w_{2},tt,-1\big),tt
,-1\big)+\sigma  \ou\big(w_{1},tt,-1\big)+Y_{1}
}\\&
y_{2}=\ParMath{-a^{2} \sigma  \eps^{2} \ou\big(\ou\big(\ou\big(w_{2},tt,-1\big),
tt,-1\big),tt,-1\big)+a \sigma  \eps \ou\big(\ou\big(w_{1},tt,-1\big),tt
,-1\big)+\sigma  \ou\big(w_{2},tt,-1\big)+Y_{2}
}\\&
x_{1}=\ParMath{a \epsilon  \sigma  \eps^{2} \big(\ou\big(\ou\big(w_{2},tt,-1\big)
,tt,-1\big)+\ou\big(w_{2},tt,-1\big)\big)+a \epsilon  \eps^{2} Y_{2}-
\epsilon  \sigma  \eps \ou\big(w_{1},tt,-1\big)-\epsilon  \eps Y_{1}+X_{
1}
}\\&
x_{2}=\ParMath{a \epsilon  \sigma  \eps^{2} \big(-\ou\big(\ou\big(w_{1},tt,-1
\big),tt,-1\big)-\ou\big(w_{1},tt,-1\big)\big)-a \epsilon  \eps^{2} Y_{1
}-\epsilon  \sigma  \eps \ou\big(w_{2},tt,-1\big)-\epsilon  \eps Y_{2}+X
_{2}
}\\&
x_{3}=\ParMath{a \epsilon  \sigma  \eps^{2} \big(-\ou\big(\ou\big(w_{2},tt,-1
\big),tt,-1\big) X_{2}-\ou\big(\ou\big(w_{1},tt,-1\big),tt,-1\big) X_{1}
-\ou\big(w_{2},tt,-1\big) X_{2}-\ou\big(w_{1},tt,-1\big) X_{1}\big)+a 
\epsilon  \eps^{2} \big(-X_{2} Y_{2}-X_{1} Y_{1}\big)+\epsilon ^{2} 
\sigma  \eps^{2} \big(\ou\big(w_{2},tt,1\big) Y_{1}-\ou\big(w_{1},tt,1
\big) Y_{2}\big)+\epsilon  \sigma  \eps \big(-\ou\big(w_{2},tt,-1\big) X
_{1}+\ou\big(w_{1},tt,-1\big) X_{2}\big)+\epsilon  \eps \big(X_{2} Y_{1}
-X_{1} Y_{2}\big)+X_{3}
}
\end{align*}

\paragraph{Result normal form DEs}
\begin{align*}&
\dot Y_{1}=-a \eps Y_{2}-Y_{1}
\\&
\dot Y_{2}=a \eps Y_{1}-Y_{2}
\\&
\dot X_{1}=\ParMath{-a^{2} \epsilon  \sigma  \eps^{3} w_{1}-a \epsilon  \sigma  
\eps^{2} w_{2}+\epsilon  \sigma  \eps w_{1}
}
\\&
\dot X_{2}=\ParMath{-a^{2} \epsilon  \sigma  \eps^{3} w_{2}+a \epsilon  \sigma  
\eps^{2} w_{1}+\epsilon  \sigma  \eps w_{2}
}
\\&
\dot X_{3}=\ParMath{a^{2} \epsilon  \sigma  \eps^{3} \big(-w_{2} X_{1}+w_{1} X_{2
}\big)+a \epsilon ^{2} \sigma ^{2} \eps^{3} \big(\ou\big(\ou\big(w_{2},
tt,-1\big),tt,-1\big) w_{2}+\ou\big(\ou\big(w_{1},tt,-1\big),tt,-1\big) 
w_{1}+\ou\big(w_{2},tt,-1\big) w_{2}+\ou\big(w_{1},tt,-1\big) w_{1}\big)
+a \epsilon  \sigma  \eps^{2} \big(w_{2} X_{2}+w_{1} X_{1}\big)+
\epsilon ^{2} \sigma ^{2} \eps^{2} \big(\ou\big(w_{2},tt,-1\big) w_{1}-
\ou\big(w_{1},tt,-1\big) w_{2}\big)+\epsilon  \sigma  \eps \big(w_{2} X_
{1}-w_{1} X_{2}\big)
}
\end{align*}
\begin{itemize}
\item As expected, \(\Yv=0\) is the stochastic slow manifold, and is exponentially attractive (almost always) in some domain about the origin.
\item The slow~\Xv\ evolution is independent of~\Yv: \(X_1,X_2\)~undergo a correlated `slow' random walk; whereas~\(X_3\) is dominantly some multiplicative random walk.
\item The time-dependent coordinate transform maps these predictions back into the \(\xv,\yv\)-space.
\end{itemize}



\paragraph{But fails to construct a stable manifold}
This procedure can directly construct the (slow) centre manifold of the system, just specify the optional fifth argument as~\verb|cman|.  The result is the same as substituting \(\Yv=0\) into the above, but is significantly more efficient because the combinatorially exploding algebra is much less.  

However, the procedure fails to construct a stable manifold for this system.  Try the following and see several error messages due to a bad ``\emph{x-residual component}''.
\begin{reduce}
stonormalform(
    {epsilon*y(1), epsilon*y(2),
     epsilon*(x(1)*y(2)-x(2)*y(1))},
    {-y(1)-a*y(2)+w(1),
     -y(2)+a*y(1)+w(2)},
    {}, 4 , sman)$
\end{reduce}
The reason is that in the above system the manifold \(\Xv=0\) is \emph{not} invariant.  Instead, here~\(\dot\Xv\) has `irremovable' forcing due to the noise terms, such as \( \epsilon \sigma \eps \ou\big(w_1,tt,-1\big)\).

The procedure does not abandon iteration straight away.  Instead, it attempts more iterations just in case the failure disappears with further iterations (the procedure halves the maximum number of iterations allowed each time such a failure is detected, and stops when the maximum is less than the total number of iterations done so far).  But here, \emph{for general~\(w_j(t)\)}, there is no invariant stable manifold, so the attempt is abandoned.

In some scenarios you might only be interested in a specific form of~\(w_j(t)\), such as zero mean periodic~\(w_j(t)\), when a stable manifold exists but this procedure does not construct it (although the algorithm could be modified to cope).
The coded procedure here is for quite general~\(w_j(t)\), such as white noise. 

