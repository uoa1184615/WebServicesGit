%!TEX root = ../manyExamples.tex
\subsection{\texttt{slowOsc}: Radek's slow oscillation with fast noise} 
\label{slowOsc}

Consider Radek's system
\begin{displaymath}
\dot x=-\epsilon xz\,,\quad
 \dot y=+\epsilon yz \qtq{and} 
 \dot z=-(z-1)+\sigma w(t)\,.
\end{displaymath}
In this linear system, \(x,y\)~oscillate with `frequency'~\(\epsilon z\).  But~\(z(t)\) is an Ornstein--Uhlenbeck process with mean one.
What are \text{the dynamics?}

Transform to our standard form via
\begin{displaymath}
x=x_1 \,,\quad
y=x_2 \qtq{and}
z=1+y_1\,.
\end{displaymath}
Then start by loading the procedure.
\begin{reduce}
in_tex "../stoNormForm.tex"$
\end{reduce}
Execute the construction of a normal form for this system.
\begin{reduce}
factor x;
stonormalform(
    { -x(2)*(1+y(1)),
      x(1)*(1+y(1)) },
    { -y(1)+w(1) },
    {},
    4 )$
end;
\end{reduce}

With the above input the procedure analyses the following system:
\begin{align*}&
\dot x_{1}=x_{2} \eps \big(-y_{1}-1\big)
\\&
\dot x_{2}=x_{1} \eps \big(y_{1}+1\big)
\\&
\dot y_{1}=\sigma  w_{1}-y_{1}
\end{align*}
This is precisely the original system, but with variables changed as above, and with parameter \(\eps=\epsilon\) (here we use the procedure's multiplication by~\(\eps\) to incorporate Radek's~\(\epsilon\)).

\paragraph{Time dependent coordinate transform}
\begin{align*}&
y_{1}=\sigma  \ou\big(w_{1},tt,-1\big)+Y_{1}
\\&
x_{1}=\ParMath{-\sigma  \eps^{2} \ou\big(w_{1},tt,-1\big) X_{1} Y_{1}+\sigma  
\eps \ou\big(w_{1},tt,-1\big) X_{2}-1/2 \eps^{2} X_{1} Y_{1}^{2}+\eps X_
{2} Y_{1}+X_{1}
}
\\&
x_{2}=\ParMath{-\sigma  \eps^{2} \ou\big(w_{1},tt,-1\big) X_{2} Y_{1}-\sigma  
\eps \ou\big(w_{1},tt,-1\big) X_{1}-1/2 \eps^{2} X_{2} Y_{1}^{2}-\eps X_
{1} Y_{1}+X_{2}
}
\end{align*}

\paragraph{Result normal form DEs}
In such linear systems, the following normal form is straightforward.
\begin{align*}&
\dot Y_{1}=-Y_{1}
\\&
\dot X_{1}=-\sigma  \eps w_{1} X_{2}-\eps X_{2}
\\&
\dot X_{2}=\sigma  \eps w_{1} X_{1}+\eps X_{1}
\end{align*}
\begin{itemize}
\item As expected, \(Y_1=0\) is the emergent stochastic slow manifold.
\item The slow~\Xv\ evolution clearly oscillates in~$(X_1,X_2)$, \(X_j\propto e^{i\theta}\), with phase angle $\theta=\eps (t+\sigma W(t,\omega))$, recalling \(W=\int w\,dt\)\,.  This phase grows linearly with a superposed random walk.
\item The time-dependent coordinate transform maps these predictions back into the \(\xv,y_1\)-plane, and thence to the original \(xyz\)-space.
\item In this system, higher-order terms in~\(\eps\) only affect the coordinate transform, they do not change the evolution of~\Xv.
\end{itemize}


