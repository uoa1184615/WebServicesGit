%!TEX root = ../manyExamples.tex
\subsection{\texttt{noisyMMH}: noisy Michaelis--Menten--Henri chemical kinetics} 
\label{nonautoTwo}
\def\b#1{\color{blue}{}#1\color{black}{}}
\def\dash#1{#1'}% change to dash time derivatives


The Michaelis--Menten--Henri system, in non-dimensional form, is
\begin{align*}
&\dot x=\epsilon[-x+(x+\kappa-\lambda)y],
&&\dot y=x-(x+\kappa)y\,.
\end{align*}
The usual approach is singular perturbation theory, but let's not obfuscate with singular limits, and instead use the clarity of being regular.
A manifold of equilibria occur at \(y=x/(x+\kappa)\) and \(\epsilon=0\)  (also if \(\epsilon\neq 0\) and \(\lambda=0\) but we do not consider this case).
Let's explore dynamics based at arbitrary point on this equilibrium manifold:
substitute \(x(t)=x_0+x_1(t)\) and \(y(t)=x_0/(x_0+\kappa)+y_1(t)\).
Also, to get the decay rate of~\(y_1\) to be a simple number, stretch time by the factor~\(x_0+\kappa\): that is, \((x_0+\kappa)dt=d\tau\) where \(\tau\)~is the time of the analysis so that
\begin{equation*}
\frac1{x_0+\kappa}\dot x_1(t)=x'_1(\tau),\quad
\frac1{x_0+\kappa}\dot y_1(t)=y'_1(\tau).
\end{equation*}
This does mean that we have to be careful interpreting the results.  Hence derive
\begin{align*}&
x_1'=\frac\epsilon{x_0+\kappa}\left[
-x_0-x_1+(x_0+x_1+\kappa-\lambda)(x_0/(x_0+\kappa)+y_1)
\right],
\\&
y_1'=-y_1+\frac1{x_0+\kappa}\left[
-x_1y_1+x_0+x_1-\frac{(x_0+x_1+\kappa)x_0}{x_0+\kappa}\right].
\end{align*}
As a prototypical example, let's investigate the simplest
stochastic effects on this MM system of additive noises~\(w_1(\tau)\) and~\(w_2(\tau)\). The
additive noise transforms to a multiplicative noise on
the slow manifold, so it is important to remember that
\emph{all} analysis and results are in the
\emph{Stratonovich} interpretation.

The analysis here is \emph{strong, pathwise}.
The transformations here only rely on the `noise' being
measurable, so the results also apply to deterministic
non-autonomous forcing. The analysis may also apply to
non-Brownian noise provided the appropriate interpretation
is used (e.g., the Marcus interpretation).  That is, as long
as standard rules of integral calculus are valid.

Start by loading the procedure.
\begin{reduce}
in_tex "../stoNormForm.tex"$
\end{reduce}
Set some variables for simplicity.
\begin{reduce}
let rho*x0=>1-rho*kappa; % define rho=1/(x0+kappa)
kappa:=1; lam:=1/2;      % set for simplicity
\end{reduce}
Execute the construction of a normal form for the system.
\begin{reduce}
factor epsilon;
stonormalform(
    { epsilon*rho*( -x0-x(1)
      +(x0+x(1)+kappa-lam)*(x0*rho+y(1)) )
      +w(1) },
    { -y(1)+rho*( -x(1)*y(1)+x0+x(1)
      -(x0+x(1)+kappa)*x0*rho )
      +w(2) },
    {},
    3 )$
end; 
\end{reduce}

The procedure reports that it analyses the following family 
\begin{align*}&
\dot x_{1}=\epsilon  \eps \big(x_{1} y_{1} \rho -x_{1} \rho ^{2}-1/2 y_{
1} \rho +y_{1}+1/2 \rho ^{2}-1/2 \rho \big)+\sigma  w_{1}
\\&
\dot y_{1}=\sigma  w_{2}+\eps \big(-x_{1} y_{1} \rho +x_{1} \rho ^{2}
\big)-y_{1}
\end{align*}



\paragraph{Time dependent coordinate transform}  The algorithm constructs a coordinate transform to
variables~\((X_1,Y_1)\), including terms quadratic
in~\(\sigma\), that to errors~\Ord{\sigma^2,\varepsilon^2},
is the following.  
The coordinate transform depends upon
both the past and the future via convolutions
\(\ou\big({},tt,-1\big)\) and \(\ou\big({},tt,1\big)\),
respectively.
The following expressions are complicated because stochastic
effects interact through nonlinearity in a combinatorial
explosion of ways.  We almost certainly do not need all
these terms.  I subsequently explain why the \b{blue terms}
are the ones describing the emergent stochastic slow
manifold and the evolution thereon. Further, remember that
the dominant terms are towards the end of each expression.
\begin{align*}&
y_{1}=\ParMath{\sigma  \eps \big(-\ou\big(\ou\big(w_{2},tt,-1\big),tt,-1\big) X_{
1} \rho 
\b{-\ou\big(w_{1},tt,-1\big) \rho ^{2}\big)+\sigma  \ou\big(w_{2},
tt,-1\big)}
+\eps X_{1} \rho ^{2}+Y_{1}
}\\&
x_{1}=\ParMath{\epsilon  \sigma  \eps \big(-\ou\big(w_{2},tt,-1\big) X_{1} \rho 
\b{+1/2 \ou\big(w_{2},tt,-1\big) \rho -\ou\big(w_{2},tt,-1\big)
}
-\ou\big(w_{1},tt,1\big) Y_{1} \rho \big)+\epsilon  \eps \big(-X_{1} Y_{1} \rho +1/2 
Y_{1} \rho -Y_{1}\big)+X_{1}
 }
\end{align*}
These new coordinates~\((X_1,Y_1)\) are non-Markovian in
relation to~\((x_1,y_1)\), in some sense, but the
non-Markovian nature is exponentially decaying away from the
current time. The construction of a non-autonomous
stochastic slow manifold has to look to the future and the
past in order to find out what variations are going to stay
bounded for all time.


\paragraph{Result normal form DEs}
In the \((X_1,Y_1)\) coordinates, the stochastic system
satisfies the following Stratonovich system, to
errors~\Ord{\sigma^3,\varepsilon^3}.
\begin{align*}&
\dash Y_{1}=\ParMath{ \epsilon  \sigma ^{2} \eps^{2} \big(\ou\big(\ou\big(w_{2},tt,
-1\big),tt,-1\big) w_{1} Y_{1} \rho ^{2}+2 \ou\big(w_{2},tt,-1\big) w_{1
} Y_{1} \rho ^{2}\big)+\epsilon  \sigma  \eps^{2} \big(2 w_{2} X_{1} Y_{
1} \rho ^{2}-w_{2} Y_{1} \rho ^{2}+2 w_{2} Y_{1} \rho -w_{1} Y_{1} \rho 
^{3}\big)+\epsilon  \eps^{2} \big(-X_{1} Y_{1} \rho ^{3}+1/2 Y_{1} \rho 
^{3}-Y_{1} \rho ^{2}\big)-\eps X_{1} Y_{1} \rho -Y_{1}
}\\&
\dash X_{1}=\ParMath{ \b{1/2 \epsilon ^{2} \sigma  \eps^{2} w_{2} \rho ^{2}}
+\epsilon \sigma ^{2} \eps^{2} \big(-\ou\big(\ou\big(w_{2},tt,-1\big),tt,-1\big) w
_{1} X_{1} \rho ^{2}-2 \ou\big(w_{2},tt,-1\big) w_{1} X_{1} \rho ^{2}
\b{+1/2 \ou\big(w_{2},tt,-1\big) w_{1} \rho ^{2}-\ou\big(w_{2},tt,-1\big) w_{1} \rho -\ou\big(w_{1},tt,-1\big) w_{1} \rho ^{3}\big)+\epsilon  \sigma ^{2} \eps \ou\big(w_{2},tt,-1\big) w_{1} \rho}
+\epsilon  \sigma  \eps^{2} \big(-w_{2} X_{1}^{2} \rho ^{2}+1/2 w_{2} X_{1} \rho ^{2}-w_{2} X_{1} 
\rho -w_{1} X_{1} \rho ^{3}
\b{+1/2 w_{1} \rho ^{3}-w_{1} \rho ^{2}}\big)+
\epsilon  \sigma  \eps \big(w_{2} X_{1} \rho 
\b{-1/2 w_{2} \rho +w_{2}}\big)
+\epsilon  \eps^{2} \big(X_{1}^{2} \rho ^{3}-1/2 X_{1} \rho ^{3}+X_{1} 
\rho ^{2}\big)+\epsilon  \eps \big(-X_{1} \rho ^{2}
\b{+1/2 \rho ^{2}-1/2 
\rho \big)+\sigma  w_{1}}
}
\end{align*}

\paragraph{Discussion}
\begin{itemize}
\item In the \(Y_1'\) \sde, by construction, every term
is\({}\propto Y_1\), and, further, the leading term gives
\(Y_1' \approx -Y_1\). Hence,
\(Y_1 \approx \Ord{e^{-\tau}} =\Ord{e^{-\int(x_0+\kappa)dt}}\)
as time increases. Consequently, by continuity, \emph{in
some finite domain} about~\((x_0,y_0)\), \(Y_1\to 0\), a.s., to
form the emergent stochastic slow manifold \(Y_1=0\).

\item The local shape of the slow manifold is thus given by
substituting \(Y_1=0\) into the expressions
for~\((x_1,y_1)\). Thus the slow manifold is locally
parametrised by~\(X_1,\epsilon,\sigma\). Now the variation
in~\(X_1\) is the Taylor series for the variation in~\(x_0\)
(as they are both describing the same slow manifold). So all
we need is to set \(X_1=0\) and look at the shape of the
slow manifold in terms of~\(x_0,\epsilon,\sigma\), that is,
the \b{blue terms}.
\begin{itemize}

\item Setting \(X_1=Y_1=0\) gives \(y_1\approx
%\epsilon\sfrac12(\rho^3-\rho^4)
-\sigma \ou\big(w_{1},tt,-1\big) \rho ^{2} +\sigma 
\ou\big(w_{2},tt,-1\big)\), that is, since
$\rho=1/(x+\kappa)=1/(x+1)$,
\begin{equation*}
y=y_0+y_1\approx \frac{x_0}{x_0+1}
%+\epsilon\frac{x_0}{2(x_0+1)^4}
-\frac{\sigma\ou\big(w_{1},tt,-1\big)}{ (x+1) ^{2}} 
+\sigma  \ou\big(w_{2},tt,-1\big)\,.
\end{equation*}
Dominantly, the slow manifold jitters up/down in~\(y\) due
to the recent history of noise~\(w_2\), but also is affected
from the recent history of the noise~\(w_1\) in~\(x\). 

\item Setting \(X_1=Y_1=0\) gives \(x_1\approx\epsilon 
\sigma  \varepsilon  \big( 1/2 \ou\big(w_{2},tt,-1\big) \rho
-\ou\big(w_{2},tt,-1\big) \big)\), that is, 
\begin{equation*}
x=x_0+x_1 \approx x_0-\epsilon  \sigma   \frac{2x_0+1}{2(x_0+1)}\ou\big(w_{2},tt,-1\big)\,.
\end{equation*}
The noise~\(w_2\) in~\(y\) generates a history dependent
slip between~\(x\) and the relevant~\(x_0\)!  

This slip may be seen to be due to the slope of isochrons
transversal to the slow manifold---a slope not detected in
Singular Perturbation Analysis.

\end{itemize}
This stochastic-MM example also shows the general property
that although the \emph{existence} of a slow manifold has
future dependence, here via~\(\ou\big({},tt,1\big)\)
convolutions, the slow manifold itself and the evolution
thereon depends only upon the history, here
via~\(\ou\big({},tt,-1\big)\) convolutions.

\item Now for the \(x\)-evolution on the stochastic slow
manifold. Consider \(X(t)=x_0+x_1(t)\), so that \(X'=x_1'\).
Recall that on the slow manifold, \(Y_1=0\) and~\(x\approx
x_0-\epsilon  \sigma \frac {2x+1} {2(x+1)}
\ou\big(w_{2},tt,-1\big)\), \(X_1=0\), so also putting
\(X_1=0\) and \(x_0=X\) give the evolution for the slow
variable~\(X\), namely the \emph{global} slow evolution is
\begin{equation*}
X'\approx -\epsilon \frac{X}{2(X+1)^2} +\epsilon^2
\frac{X(2X+1)}{4(X+1)^ 5} +\sigma w_1 +\epsilon \sigma
\frac{2X+1}{2(X+1)}w_2\,.
\end{equation*}
The stochastic slow variable~\(X\) is not quite the same as
the physical~\(x\). This coordinate transform lacks any
convolutions in time. That lack is part of the art of the
construction.

If, instead, one wants the slow variable to be
precisely~\(x\), as many implicitly assume they can, then
convolutions must occur in~\(x')\). We may see this by
constructing a nonlinear coordinate transform that
maintains, when \(Y_1=0\), that \(x=x_0\), precisely.  It is
straightforward to modify the algorithm to do so.  The
generic consequence is that terms \emph{linear} in the noise
appear in the evolution~\(x'\) that have fast-time history
convolutions. That is, the consequence is that undesirable
fast-time history integrals occur in the evolution of the
supposedly slow variable~\(x\).



\paragraph{Noise-noise interactions}
However, effects which are quadratic in the noise, due to
noise-noise interactions, generally involve convolutions
that \emph{cannot} be removed from the evolution of the slow
variable, as seen in expressions for~\(X_1'\). Here, the
lowest order example is the term
\begin{equation*}
+\epsilon  \sigma ^{2} \ou\big(w_{2},tt,-1\big) w_{1} /(x+1)
\end{equation*}
which could be included in the retained terms of~\(X'_1\).
We argued \cite[\S4]{Chao95} that such terms `bring up'
\emph{new information} from the fluctuations on the
fast-time microscale, and hence cause noise effects in the
slow model that are independent of slow-scale sampling
of~\(w_1\) and~\(w_2\). We argued that such terms, when one
only samples them on the long-times of the slow manifold,
should thus be replaced by a new noise, namely
\(\ou\big(w_{2},tt,-\beta\big)
w_{1}\sim\frac1{2\sqrt\beta}w_3\) when all~\(w_j\) are
formally `the derivatives' of independent Wiener processes.

\end{itemize}

The above results are for one example of a stochastic MM
system.  Almost all other stochastic MM systems would have
the same issues.



