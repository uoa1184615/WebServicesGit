\section{Initialise LaTeX output}
% AJR, 21 aug 2012 -- Jul 2015

This section writes to various files so the output to \verb|cmsyso.txt| must be redone afterwards.

First define how various tokens get printed.
%Using braces for every parenthesis has the virtue that we can typeset the exp functions quite nicely; however, it has the big disadvantage that the formulae are no longer line broken within a brace pair.

\begin{reduce}
load_package rlfi;
%deflist('((!( !{!\!b!i!g!() (!) !\!b!i!g!)!}) (!P!I !\!p!i! )
%         (!p!i !\!p!i! ) (!E !e) (!I !i) (e !e) (i !i)),'name)$
deflist('((!( !\!l!P!a!r! ) (!) !\!r!P!a!r) 
    (!P!I !\!p!i! ) (!p!i !\!p!i! ) (!E !e) (!I !i) 
    (e !e) (i !i)),'name)$
\end{reduce}

Force all fractions (coded in Reduce as \verb|quotient|) to use \verb|\frac| command so we can change how it appears.
\begin{reduce}
put('quotient,'laprifn,'prinfrac);
\end{reduce}


Override the procedure that prints annoying messages about multicharacter symbols.  
It ends the output of one expression.  
This is just a copy from \verb|rlfi.red| with the appropriate if-statement deleted.
While interfering, hardcode that the mathematics is in inline mode.

\begin{reduce}
%write "Ignore immediately following messages";
symbolic procedure prinlaend;
<<terpri();
  prin2t "\)\par";
  if !*verbatim then
      <<prin2t "\begin{verbatim}";
        prin2t "REDUCE Input:">>;
  ncharspr!*:=0;
  if ofl!* then linelength(car linel!*)
    else laline!*:=cdr linel!*;
  nochar!*:=append(nochar!*,nochar1!*);
  nochar1!*:=nil >>$
  %
\end{reduce}
Similarly, hardcode at the beginning of expression output that the mathematics is in inline mode.
\begin{reduce}
symbolic procedure prinlabegin;
% Initializes the output
<<if !*verbatim then
      <<terpri();
        prin2t "\end{verbatim}">>;
  linel!*:=linelength nil . laline!*;
  if ofl!* then linelength(laline!* + 2)
    else laline!*:=car linel!* - 2;
  prin2 "\(" >>$
\end{reduce}

Override the procedure that outputs the \LaTeX\ preamble 
upon the command \verb|on latex|.
Presumably modified from that in \verb|rlfi.red|.
Use it to write a decent header that we use for one master file.

In the following, not clear that we should simply omit parentheses with the exp function.
Could do something cleverer with \verb|\lPar| and \verb|\rPar| such as have a counter and cycle through the alternatives depending upon the counter.

%  prin2t "\def\exp\big(#1\big){\,e^{#1}}";

\begin{reduce}
symbolic procedure latexon;
<<!*!*a2sfn:='texaeval;
  !*raise:=nil;
  prin2t "\documentclass[11pt,a5paper]{article}";
  prin2t "\usepackage[a5paper,margin=13mm]{geometry}";
  prin2t "\usepackage{parskip,time} \raggedright";
  prin2t "\def\lPar{\mathchoice{\big(}{\big(}{(}{(}}";
  prin2t "\def\rPar{\mathchoice{\big)}{\big)}{)}{)}}";
  prin2t "\let\FRaC\frac";
  prin2t "\renewcommand{\frac}[2]{\mathchoice%";
  prin2t "    {\FRaC{#1}{#2}}{\FRaC{#1}{#2}}{#1/#2}{#1/#2}}";
  prin2t "\def\exp{\,e}";
  prin2t "\def\eps{\varepsilon}";
  prin2t "\title{Invariant manifold of your dynamical system}";
  prin2t "\author{A. J. Roberts, University of Adelaide\\";
  prin2t "\texttt{http://www.maths.adelaide.edu.au/anthony.roberts}}";
  prin2t "\date{\now, \today}";
  prin2t "\begin{document}";
  prin2t "\maketitle";
  prin2t "Throughout and generally: the lowest order, most";
  prin2t "important, terms are near the end of each expression.";
  prin2t "
\(\)
\paragraph{The specified dynamical system}
\(
\)\par

\(\dot u_{1}=b \eps u\sb2 u\sb5-\eps u\sb2 u\sb3
\)\par

\(\dot u_{2}=-b \eps u\sb1 u\sb5+\eps u\sb1 u\sb3
\)\par

\(\dot u_{3}=-\eps u\sb1 u\sb2
\)\par

\(\dot u_{4}=-u\sb5
\)\par

\(\dot u_{5}=b \eps u\sb1 u\sb2+u\sb4
\)\par

\(\)
\paragraph{Centre subspace basis vectors}
\(
\)\par

\(\vec e_{1}=\left\{
\left\{
1 , 0 , 0 , 0 , 0
\right\} , \cis\big(0\big)
\right\}
\)\par

\(\vec e_{2}=\left\{
\left\{
0 , 1 , 0 , 0 , 0
\right\} , \cis\big(0\big)
\right\}
\)\par

\(\vec e_{3}=\left\{
\left\{
0 , 0 , 1 , 0 , 0
\right\} , \cis\big(0\big)
\right\}
\)\par

\(\vec z_{1}=\left\{
\left\{
1 , 0 , 0 , 0 , 0
\right\} , \cis\big(0\big)
\right\}
\)\par

\(\vec z_{2}=\left\{
\left\{
0 , 1 , 0 , 0 , 0
\right\} , \cis\big(0\big)
\right\}
\)\par

\(\vec z_{3}=\left\{
\left\{
0 , 0 , 1 , 0 , 0
\right\} , \cis\big(0\big)
\right\}
\)\par
";
  if !*verbatim then
      <<prin2t "\begin{verbatim}";
        prin2t "REDUCE Input:">>;
  put('tex,'rtypefn,'(lambda(x) 'tex)) >>$
\end{reduce}

Set the default output to be inline mathematics.

\begin{reduce}
mathstyle math;
\end{reduce}

Define the Greek alphabet with \verb|small| as well.

\begin{reduce}
defid small,name="\eps";%varepsilon;
defid alpha,name=alpha;
defid beta,name=beta;
defid gamma,name=gamma;
defid delta,name=delta;
defid epsilon,name=epsilon;
defid varepsilon,name=varepsilon;
defid zeta,name=zeta;
defid eta,name=eta;
defid theta,name=theta;
defid vartheta,name=vartheta;
defid iota,name=iota;
defid kappa,name=kappa;
defid lambda,name=lambda;
defid mu,name=mu;
defid nu,name=nu;
defid xi,name=xi;
defid pi,name=pi;
defid varpi,name=varpi;
defid rho,name=rho;
defid varrho,name=varrho;
defid sigma,name=sigma;
defid varsigma,name=varsigma;
defid tau,name=tau;
defid upsilon,name=upsilon;
defid phi,name=phi;
defid varphi,name=varphi;
defid chi,name=chi;
defid psi,name=psi;
defid omega,name=omega;
defid Gamma,name=Gamma;
defid Delta,name=Delta;
defid Theta,name=Theta;
defid Lambda,name=Lambda;
defid Xi,name=Xi;
defid Pi,name=Pi;
defid Sigma,name=Sigma;
defid Upsilon,name=Upsilon;
defid Phi,name=Phi;
defid Psi,name=Psi;
defid Omega,name=Omega;
\end{reduce}


\begin{reduce}
defindex e_(down,down);
defid e_,name="e";
defindex d_(arg,down,down);
defid d_,name="D";
defindex u(down);
%defid u1,name="u\sb1"; 
%defid u2,name="u\sb2"; 
%defid u3,name="u\sb3"; 
%defid u4,name="u\sb4"; 
%defid u5,name="u\sb5"; 
%defid u6,name="u\sb6"; 
%defid u7,name="u\sb7"; 
%defid u8,name="u\sb8"; 
%defid u9,name="u\sb9"; 
defindex s(down);
defindex exp(up);
defid exp,name="e"; %does not work??
\end{reduce}


Can we write the system?  
Not in matrices apparently.
So define a dummy array \verb|tmp| that we use to get the correct symbol typeset.

\begin{reduce}
array tmp(n),tmps(m),tmpz(m);
defindex tmp(down);
defindex tmps(down);
defindex tmpz(down);
defid tmp,name="\dot u";
defid tmps,name="\vec e";
defid tmpz,name="\vec z";
rhs_:=rhsfn_$
for k:=1:m do tmps(k):={for j:=1:n collect ee_(j,k),exp(evl_(k)*t)};
for k:=1:m do tmpz(k):={for j:=1:n collect zz_(j,k),exp(evl_(k)*t)};
\end{reduce}

We have to be shifty here because \verb|rlfi| does not work inside a loop: so write the commands to a file, and then input the file.
The output line length of each `write' statement must be short enough as otherwise Reduce puts in a line break.

\begin{reduce}
out "scratchfile.red";
write "write ""\)
\paragraph{The specified dynamical system}
\("";";
for j:=1:n do write "tmp(",j,"):=coeffn(rhs_,e_(",j,",1),1);";
write "write ""\)
\paragraph{",natureMan_,"
subspace basis vectors}","
\("";";
for j:=1:m do write "tmps(",j,"):=tmps(",j,");";
for j:=1:m do write "tmpz(",j,"):=tmpz(",j,");";
write "end;";
shut "scratchfile.red";
\end{reduce}
Now print the dynamical system to the LaTeX sub-file.

\begin{reduce}
on latex$
out "centreManSys.tex"$
in "scratchfile.red"$
shut "centreManSys.tex"$
off latex$
\end{reduce}

Finish the input.
\begin{reduce}
end;
\end{reduce}
