\documentclass[11pt,a5paper]{article}
\usepackage[a5paper,margin=13mm]{geometry}
\usepackage{parskip,time} \raggedright
\def\cis\big(#1\big){\,e^{#1i}}
\def\eps{\varepsilon}
\title{Centre manifold of your dynamical system}
\author{A. J. Roberts, University of Adelaide\\
\texttt{http://www.maths.adelaide.edu.au/anthony.roberts}}
\date{\now, \today}
\begin{document}
\maketitle
Throughout and generally: the lowest order, most
important, terms are near the end of each expression.

\(\)
\paragraph{The specified dynamical system}
\(
\)\par

\(\dot u_{1}=b \eps u\sb2 u\sb5-\eps u\sb2 u\sb3
\)\par

\(\dot u_{2}=-b \eps u\sb1 u\sb5+\eps u\sb1 u\sb3
\)\par

\(\dot u_{3}=-\eps u\sb1 u\sb2
\)\par

\(\dot u_{4}=-u\sb5
\)\par

\(\dot u_{5}=b \eps u\sb1 u\sb2+u\sb4
\)\par

\(\)
\paragraph{Centre subspace basis vectors}
\(
\)\par

\(\vec e_{1}=\left\{
\left\{
1 , 0 , 0 , 0 , 0
\right\} , \cis\big(0\big)
\right\}
\)\par

\(\vec e_{2}=\left\{
\left\{
0 , 1 , 0 , 0 , 0
\right\} , \cis\big(0\big)
\right\}
\)\par

\(\vec e_{3}=\left\{
\left\{
0 , 0 , 1 , 0 , 0
\right\} , \cis\big(0\big)
\right\}
\)\par

\(\vec z_{1}=\left\{
\left\{
1 , 0 , 0 , 0 , 0
\right\} , \cis\big(0\big)
\right\}
\)\par

\(\vec z_{2}=\left\{
\left\{
0 , 1 , 0 , 0 , 0
\right\} , \cis\big(0\big)
\right\}
\)\par

\(\vec z_{3}=\left\{
\left\{
0 , 0 , 1 , 0 , 0
\right\} , \cis\big(0\big)
\right\}
\)\par


\(
\)
\paragraph{The centre manifold}
These give the location of the centre manifold in
terms of parameters~\(s\sb j\).
\(
\)\par

\(u_{1}=3/8 \eps^{3} s_{1}^{4}-1/2 \eps s_{1}^{2}+s_{1}
\)\par

\(u_{2}=-3/8 \eps^{3} s_{1}^{4}+1/2 \eps s_{1}^{2}+s_{1}
\)\par

\(
\)
\paragraph{Centre manifold ODEs}
The system evolves on the centre manifold such
that the parameters evolve according to these ODEs.
\(
\)\par

\(\dot s_{1}=-3/4 \eps^{4} s_{1}^{5}+\eps^{2} s_{1}^{3}
\)\par

\(
\)
\paragraph{Normals to isochrons at the slow manifold}
Use these vectors: to project initial conditions
onto the slow manifold; to project non-autonomous
forcing onto the slow evolution; to predict the
consequences of modifying the original system; in
uncertainty quantification to quantify effects on
the model of uncertainties in the original system.
The normal vector \(\vec z\sb j:=(z\sb{j1},\ldots,z\sb{jn}\
)\)
\(
\)\par

\(z_{11}=3/2 \eps^{4} s_{1}^{4}+3/4 \eps^{3} s_{1}^{3}-1/2 \eps^{2} s_{1}
^{2}-1/2 \eps s_{1}+1/2
\)\par

\(z_{12}=3/2 \eps^{4} s_{1}^{4}-3/4 \eps^{3} s_{1}^{3}-1/2 \eps^{2} s_{1}
^{2}+1/2 \eps s_{1}+1/2
\)\par
\end{document}
