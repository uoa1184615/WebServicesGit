\documentclass[11pt,a5paper]{article}
\usepackage[a5paper,margin=13mm]{geometry}
\usepackage{amsmath,parskip,time} \raggedright
\def\cis\big(#1\big){\,e^{#1i}}
\def\eps{\varepsilon}
\title{Centre (slow) manifold of the example of \S7.1 by Lee \& Othmer (2010)}
\author{A. J. Roberts, University of Adelaide\\
\texttt{http://www.maths.adelaide.edu.au/anthony.roberts}}
\date{\now, \today}
\begin{document}
\maketitle

Rescale the system~(84) to the fast time so their \(\epsilon=\varepsilon^2\) here.
For simplicity set \(k_i=1\) for all~\(i\).
Then \(\epsilon=0\) and \(\vec c=(\sqrt{C_2},C_2,C_3)\) is a global manifold of equilibria.
So with \(C_1=\sqrt{C_2}\) change to local variables~\(\vec u\) by \(c_i(t)=C_i+u_i(t)\).
My website models the dynamics (creates the slow manifold) with the following input:
\begin{verbatim}
ff_:=tp mat((-2*((sqrt(c2)+u1)^2-(c2+u2))
,((sqrt(c2)+u1)^2-(c2+u2))
+small*((c3+u3)-(c2+u2))
,0-small*((c3+u3)-(c2+u2))));
freqm_:=mat((0,0));
ee_:=tp mat((1/2/sqrt(c2),1,0),(0,0,1));
zz_:=tp mat((0,1,0),(0,0,1));
toosmall:=3;
\end{verbatim}

Throughout and generally: the lowest order, most
important, terms are near the end of each expression.

\(\)
\paragraph{The specified dynamical system}
\(
\)\par
\(\dot u_{1}=-2 \eps u\sb1^{2}-4 \sqrt {C_2} u\sb1+2 u\sb2
\)\par
\(\dot u_{2}=\eps^{2} \big(-\cis\big(0\big) C_2+\cis\big(0\big) C_3-u\sb2+
u\sb3\big)+\eps u\sb1^{2}+2 \sqrt {C_2} u\sb1-u\sb2
\)\par
\(\dot u_{3}=\eps^{2} \big(\cis\big(0\big) C_2-\cis\big(0\big) C_3+u\sb2-
u\sb3\big)
\)\par
\(\)
\paragraph{Centre subspace basis vectors}
\(
\)\par
\(\vec e_{1}=\left\{
\left\{
1/2 \sqrt {C_2} C_2^{-1} , 1 , 0
\right\} , \cis\big(0\big)
\right\}
\)\par
\(\vec e_{2}=\left\{
\left\{
0 , 0 , 1
\right\} , \cis\big(0\big)
\right\}
\)\par
\(\vec z_{1}=\left\{
\left\{
0 , 1 , 0
\right\} , \cis\big(0\big)
\right\}
\)\par
\(\vec z_{2}=\left\{
\left\{
0 , 0 , 1
\right\} , \cis\big(0\big)
\right\}
\)\par

\(
\)
\paragraph{The centre manifold}
These give the location of the centre manifold in
terms of parameters~\(s\sb j\).
\(
\)\par
\(u_{1}=-1/8 \sqrt {C_2} \eps s_{1}^{2} C_2^{-2}+1/2 \sqrt {C_2} s_{1} C_2^{-1
}
\)\par
\(u_{2}=s_{1}
\)\par
\(u_{3}=s_{2}
\)\par
That is, 
\begin{eqnarray*}
c_2&=&C_2+u_2=C_2+s_1,\\
c_3&=&C_3+u_3=C_3+s_2,\\
c_1&=&C_1+u_1\approx\sqrt{C_2}+\tfrac12s_1/\sqrt{C_2}-\varepsilon\tfrac18s_1^2/C_2^{3/2}\approx \sqrt{C_2+s_1}\,.
\end{eqnarray*}




\paragraph{Centre manifold ODEs}
The system evolves on the centre manifold such
that the parameters evolve according to these ODEs.
\(
\)\par
\(\dot s_{1}=\eps^{2} \big(-4 \sqrt {C_2} s_{2}+4 \sqrt {C_2} s_{1}+4 
\sqrt {C_2} C_2-4 \sqrt {C_2} C_3+16 s_{2} C_2-16 s_{1} C_2-16 C_2^{2}+16 C_2 C_3
\big)/\big(16 C_2-1\big)
\)\par
\(\dot s_{2}=\eps^{2} \big(-s_{2}+s_{1}+C_2-C_3\big)
\)\par

To match the result~(92), consider the 2nd evolution equation first:
\begin{eqnarray*}
\dot c_3&=&(C_3+s_2)\dot{}=\dot s_2
=\varepsilon^2(-s_{2}+s_{1}+C_2-C_3)\\
&=&\epsilon[(C_2+s_1)-(C_3+s_2)]
=\epsilon(c_2-c_3).
\end{eqnarray*}
Similarly, but in brief, the first evolution equation goves
\begin{eqnarray*}
\dot c_2&=&(C_2+s_1)\dot{}=\dot s_1
\\&=&\eps^{2} \big(-4 \sqrt {C_2} s_{2}+4 \sqrt {C_2} s_{1}+4 
\sqrt {C_2} C_2-4 \sqrt {C_2} C_3
\\&&{}+16 s_{2} C_2-16 s_{1} C_2-16 C_2^{2}+16 C_2 C_3
\big)/\big(16 C_2-1\big)
\\&\equiv&\epsilon \big(4 
\sqrt {c_2} c_2-4 \sqrt {c_2} c_3+16 s_{2} c_2-16 c_2^{2}+16 c_2 c_3
\big)/\big(16 c_2-1\big)
\\&=&\epsilon4(c_3-c_2)\sqrt{c_2}/(4\sqrt{c_2}+1)
\end{eqnarray*}
which also matches~(92).


The projection of initial conditions arise from the following, but I have not cancelled the common factors to see how the match.


\paragraph{Normals to isochrons at the slow manifold}
Use these vectors: to project initial conditions
onto the slow manifold; to project non-autonomous
forcing onto the slow evolution; to predict the
consequences of modifying the original system; in
uncertainty quantification to quantify effects on
the model of uncertainties in the original system.
The normal vector \(\vec z\sb j:=(z\sb{j1},\ldots,z\sb{jn}\
)\)
\(
\)\par
\(z_{11}=\eps^{2} \big(-3072 \sqrt {C_2} s_{1}^{2} C_2+28 \sqrt {C_2} s_{1}^{
2} C_2^{-1}-1/4 \sqrt {C_2} s_{1}^{2} C_2^{-2}-192 \sqrt {C_2} s_{1}^{2}+
8192 \sqrt {C_2} C_2^{3}+2560 \sqrt {C_2} C_2^{2}-160 \sqrt {C_2} C_2-2 
\sqrt {C_2}+2048 s_{1}^{2} C_2-128 s_{1}^{2}-8192 C_2^{3}+32 C_2\big)/\big(
65536 C_2^{4}-16384 C_2^{3}+1536 C_2^{2}-64 C_2+1\big)+\eps \big(\sqrt {C_2} 
s_{1} C_2^{-1}+16 \sqrt {C_2} s_{1}-8 s_{1}\big)/\big(256 C_2^{2}-32 C_2+1
\big)+\big(-2 \sqrt {C_2}+8 C_2\big)/\big(16 C_2-1\big)
\)\par
\(z_{12}=\eps^{2} \big(-6144 \sqrt {C_2} s_{1}^{2} C_2+56 \sqrt {C_2} s_{1}^{
2} C_2^{-1}-1/2 \sqrt {C_2} s_{1}^{2} C_2^{-2}-384 \sqrt {C_2} s_{1}^{2}+
6144 \sqrt {C_2} C_2^{2}-256 \sqrt {C_2} C_2-8 \sqrt {C_2}+4096 s_{1}^{2} C_2-
256 s_{1}^{2}-8192 C_2^{3}-1024 C_2^{2}+96 C_2\big)/\big(65536 C_2^{4}-16384
 C_2^{3}+1536 C_2^{2}-64 C_2+1\big)+\eps \big(2 \sqrt {C_2} s_{1} C_2^{-1}+32
 \sqrt {C_2} s_{1}-16 s_{1}\big)/\big(256 C_2^{2}-32 C_2+1\big)+\big(-4 
\sqrt {C_2}+16 C_2\big)/\big(16 C_2-1\big)
\)\par
\(z_{13}=\eps^{2} \big(-8 \sqrt {C_2}+16 C_2+1\big)/\big(256 C_2^{2}-32 C_2+1
\big)
\)\par
\(z_{21}=\eps^{2} \big(-32 \sqrt {C_2} C_2-2 \sqrt {C_2}+16 C_2\big)/\big(256 
C_2^{2}-32 C_2+1\big)
\)\par
\(z_{22}=\eps^{2} \big(-8 \sqrt {C_2}+16 C_2+1\big)/\big(256 C_2^{2}-32 C_2+1
\big)
\)\par
\(z_{23}=1
\)\par
\end{document}
