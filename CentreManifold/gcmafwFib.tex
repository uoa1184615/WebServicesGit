\documentclass[11pt,a5paper]{article}

\title{A general centre manifold construction algorithm for the web, including isochrons of slow manifolds}
\author{A.~J. Roberts}
\date{\today}


\usepackage{natbib,amsmath,defns,reducecode}
\let\harvardurl\url
\IfFileExists{ajr.sty}{\usepackage{ajr}}{}
\Cal L \Cal E \Cal Z
\def\dde{\textsc{dde}}
\newcommand{\adj}[1]{#1^\dagger} % denotes cc transpose
% for output of coordinate transform
\def\cis\big(#1\big){\,e^{#1i}}
\def\eps{\varepsilon}

\begin{document}

\maketitle

\begin{abstract}
This code is the heart and muscle of a web service.
The web service derives a centre manifold of any specified system of ordinary differential equations or delay differential equations, when the system has fast and centre modes. 
The centre modes may be slow, as in a pitchfork bifurcation, or oscillatory, as in a Hopf bifurcation, or some more complicated superposition. 
In the case when the fast modes all decay, the centre manifold supplies a faithful large time model of the dynamics. 
Further, this code now derives vectors defining the projection onto the centre manifold along the isochrons: this projection is needed for initial conditions, forcing, system modifications, and uncertainty quantification.
\end{abstract}

\tableofcontents

\section{Overall initialisation}

In the following, assign \verb|thecase:=myweb;| for the web service (or to read a system from file \verb|cmsysb.red|), otherwise assign \verb|thecase| to be any of the example dynamical systems in set~\verb|thecases|.
\begin{reduce}
% see gcmafwFib.pdf for detailed explanation
% AJ Roberts, Nov 2013 -- Aug 2014
thecase:=lorenz86slow; 
thecases:={onedde, anotherdde, twodde, dde2d, dde2d2ha,
dde2d2hb, simple2d, simple2ds, fourstatemarkov, another2d,
another2ds, simple3d, simple3ds, geneigenvec, bifurcate2d,
simpleosc, perturbfreq, nonseparatedosc, quasidelayosc,
quasidelayoscmod, rosslerlike, doubleosc, oscmeanflow,
modulateduffing, modulateoscillator, StoleriuOne,
StoleriuTwo, delayprolif, delayedprolif, normalmodes,
forcedvdp, lorenz86slow }$
\end{reduce}

Define default parameters for the iteration:
\verb|maxiter_|~is the maximum number of allowed iterations;
\verb|toosmall|~is the order of errors in the analysis in terms of the parameter~\verb|small|.
Specific problems may override these defaults.

\begin{reduce}
maxiter_:=29$
factor small; 
toosmall:=3$
\end{reduce}

For optional trace printing of test cases: 
comment out second line when not needed.
\begin{reduce}
trace_:=0$
%trace_:=1; maxiter_:=5; 
\end{reduce}


The \verb|rationalise| switch makes code much faster with complex numbers.
The switch \verb|gcd| seems to wreck convergence, so leave it off.

\begin{reduce}
on div; off allfac; on revpri; 
on rationalize;
linelength 60$
\end{reduce}

Propose to use \verb|e_| as basis vector for matrices and vectors.
Declare it non-commutative so that multiplication does not commute.
\begin{reduce}
operator e_;
noncom e_;
factor e_;
let { e_(~j,~k)*e_(~l,~m)=>0 when k neq l
    , e_(~j,~k)*e_(~l,~m)=>e_(j,m) when k=l 
    , e_(~j,~k)^2=>0 when j neq k
    , e_(~j,j)^2=>e_(j,j) };
\end{reduce}
Also need a transpose operator: do complex conjugation explicitly when needed.
\begin{reduce}
operator tpe_; linear tpe_;
let tpe_(e_(~i,~j),e_)=>e_(j,i);
\end{reduce}



Need to enter delayed factors in the \ode{}s, so use operators for the dependent variables in the dynamical system (for the moment up to nine).
\begin{reduce}
operator u,u1,u2,u3,u4,u5,u6,u7,u8,u9;
\end{reduce}

Empty the output LaTeX file in case of error.

\begin{reduce}
out "centreMan.tex";
write "This empty document indicates error.";
shut "centreMan.tex";
\end{reduce}

Automatically testing a set of examples does not yet work. 
\begin{reduce}
%foreach thecase in thecases do begin
\end{reduce}


\section{Some example systems}

Define the basic linear operator, centre manifold bases, and `nonlinear' function.  
Note that Reduce's matrix transpose does not take complex conjugate.
Then the web service inputs the system from a file, otherwise get the system from one of the examples that follow.

\begin{reduce}
if thecase=myweb then begin
in "cmsysb.red"$
end;
\end{reduce}



\subsection{Simple one variable delay differential equation}

Model a delayed `logistic' system in one variable with
\begin{equation*}
\de tu=-(1+a)[1+u(t)]u(t-\pi/2),
\end{equation*}
for small parameter~\(a\).
We code the parameter~\(a\) as `small', and observe it is consequently considered as `small squared' because all nonlinear terms and already `small' terms are multiplied by~\verb|small|.
The marginal modes are~\(e^{\pm it}\) so nominate the frequencies~\(\pm 1\).
The eigenvectors are just~\(1\cdot e^{\pm it}\). 
Because for delay differential equations the time dependence~\(e^{\pm i\omega t}\) is an integral part of the definition of the eigenvector; hence the coded eigenvectors can be the same, as here, because they are differentiated through the time dependence~\(e^{\pm i\omega t}\).

\begin{reduce}
if thecase=onedde then begin
ff_:=tp mat((-(1+small*a)*(1+u1)*u1(pi/2)));
freqm_:=mat((1,-1));
ee_:=tp mat((1),(1));
zz_:=tp mat((1),(1));
toosmall:=3; 
factor s,a,cis;
end;
\end{reduce}

The code works for orders higher than cubic, but is slow: takes about a minute per iteration.

\paragraph{The centre manifold}
\begin{math}
\end{math}\par

\begin{math}
u_{1}=\cis\big(-2 t\big) s_{2}^{2} \varepsilon  \big(1/5 i+2/5\big)+\cis
\big(-t\big) s_{2}+\cis\big(2 t\big) s_{1}^{2} \varepsilon  \big(-1/5 i+
2/5\big)+\cis\big(t\big) s_{1}
\end{math}\par

\paragraph{Centre manifold ODEs}
\begin{math}
\end{math}\par

\begin{math}
\dot s_{1}=s_{2} s_{1}^{2} \varepsilon ^{2} \big(-2/5 i \pi -12/5 i-6/5 
\pi +4/5\big)/\big(\pi ^{2}+4\big)+s_{1} a \varepsilon ^{2} \big(4 i+2 
\pi \big)/\big(\pi ^{2}+4\big)
\end{math}\par

\begin{math}
\dot s_{2}=s_{2}^{2} s_{1} \varepsilon ^{2} \big(2/5 i \pi +12/5 i-6/5 
\pi +4/5\big)/\big(\pi ^{2}+4\big)+s_{2} a \varepsilon ^{2} \big(-4 i+2 
\pi \big)/\big(\pi ^{2}+4\big)
\end{math}\par

Observe that the real parts of these \ode{}s indicate linear growth for positive parameter~\(a\), limited by nonlinear saturation.
A classic Hopf bifurcation (although I have not recorded here evidence for the attractiveness).


\subsection{Another one variable delay differential equation}

Model a delayed `logistic' system in one variable with
\begin{equation*}
\de tu=-u(t)-(\sqrt2+a)u(t-3\pi/4)+\mu u(t-3\pi/4)^2+\nu u(t-3\pi/4)^3,
\end{equation*}
for small parameter~\(a\) and nonlinearity parameters \(\mu\) and~\(\nu\).  Numerical computation of the spectrum indicates that the system has a Hopf bifurcation as parameter~\(a\) crosses zero.%
\footnote{Replacing \(-(\sqrt2+a)\) with \(+(1+a)\) leads to a pitchfork bifurcation with broken symmetry when \(\mu\neq 0\).}
\begin{verbatim}
ac=-sqrt(2), tau=3*pi/4
ce=@(z) z+1-ac*exp(-tau*z)
lams=fsolve(ce,randn(100,2)*[2;2*i])
plot(real(lams),imag(lams),'o')
\end{verbatim}
We code the parameter~\(a\) as `small', and observe it is consequently considered as `small squared' because all nonlinear terms and already `small' terms are multiplied by~\verb|small|.
The marginal modes are~\(e^{\pm it}\) so nominate the frequencies~\(\pm 1\).
The eigenvectors are just~\(1\cdot e^{\pm it}\). 
Because for delay differential equations the time dependence~\(e^{\pm i\omega t}\) is an integral part of the definition of the eigenvector; hence the coded eigenvectors can be the same, as here, because they are differentiated through the time dependence~\(e^{\pm i\omega t}\).

\begin{reduce}
if thecase=anotherdde then begin
ff_:=tp mat((-u1-(sqrt(2)+small*a)*u1(3*pi/4)
    +mu*u1(3*pi/4)^2 +small*nu*u1(3*pi/4)^3));
freqm_:=mat((1,-1));
ee_:=tp mat((1),(1));
zz_:=tp mat((1),(1));
toosmall:=3; 
factor s,a,mu,nu,cis;
end;
\end{reduce}

The modelling predicts a supercritical Hopf bifurcation as parameter~\(a\) increases through zero, although if nonlinearity parameter~\(\nu\) is large enough negative, then the bifurcation will be subcritical.





\subsection{Separated delay differential equations}

Now consider the system
\begin{equation*}
\dot x=-[1+a-y(t)]x(t-\pi/2)
\quad\text{and}\quad
\dot y=-y+x^2.
\end{equation*}
Without the `fast' variable~\(y\) the \(x\)-\ode\ would be at marginal criticality when parameter \(a=0\).  
With the coupling, any oscillations in~\(x\) should drive a positive~\(y\) which then helps stabilise the oscillations.
Let's see this in analysis.

Code the system as follows with small parameter~\(a\).
Because the system is linearly separated, the eigenvectors are simple: the eigenvectors of the marginal modes are \((1,0)e^{\pm it}\), as are the adjoint's eigenvectors.

\begin{reduce}
if thecase=twodde then begin
ff_:=tp mat((
    -(1+small*a-u2)*u1(pi/2),
    -u2+u1^2
    ));
freqm_:=mat((1,-1));
ee_:=tp mat((1,0),(1,0));
zz_:=tp mat((1,0),(1,0));
toosmall:=3; 
factor s,a,cis;
end;
\end{reduce}


\paragraph{The centre manifold}
\begin{math}
\end{math}\par

\begin{math}
u_{1}=\cis\big(-t\big) s_{2}+\cis\big(t\big) s_{1}
\end{math}\par

\begin{math}
u_{2}=\cis\big(-2 t\big) s_{2}^{2} \varepsilon  \big(2/5 i+1/5\big)+\cis
\big(2 t\big) s_{1}^{2} \varepsilon  \big(-2/5 i+1/5\big)+2 s_{2} s_{1} 
\varepsilon 
\end{math}\par

\paragraph{Centre manifold ODEs}
\begin{math}
\end{math}\par

\begin{math}
\dot s_{1}=s_{2} s_{1}^{2} \varepsilon ^{2} \big(-4/5 i \pi -36/5 i-18/5
 \pi +8/5\big)/\big(\pi ^{2}+4\big)+s_{1} a \varepsilon ^{2} \big(4 i+2 
\pi \big)/\big(\pi ^{2}+4\big)
\end{math}\par

\begin{math}
\dot s_{2}=s_{2}^{2} s_{1} \varepsilon ^{2} \big(4/5 i \pi +36/5 i-18/5 
\pi +8/5\big)/\big(\pi ^{2}+4\big)+s_{2} a \varepsilon ^{2} \big(-4 i+2 
\pi \big)/\big(\pi ^{2}+4\big)
\end{math}




\subsection{Linearly coupled 2D DDE}

Here we explore a system where the centre modes involve both variables.  Consider the system
\begin{equation*}
\dot u_1=u_2(t-\pi/2)-u_1^2
\quad\text{and}\quad
\dot u_2=u_1(t-\pi/2)+u_2^2.
\end{equation*}
We find the quadratic reaction does not stabilise oscillating growth.

Numerical solution of the characteristic equation indicate that there is one unstable mode, \(\lambda=0.4745\), two centre modes, \(\lambda=\pm i\), and all the rest are stable modes with the gravest having eigenvalue \(\lambda=-0.6846\pm i2.8499\).
The analysis gives the centre modes are nonlinearly unstable: \(\dot a\approx (0.6758\pm i1.8616)|a|^2a\).
The following Matlab/Octave code finds eigenvalues.
\begin{verbatim}
ce=@(z) z.^2-exp(-pi*z)
lams=fsolve(ce,randn(100,2)*[2;10*i])
plot(real(lams),imag(lams),'o')
\end{verbatim}

Interestingly, the centre eigenvectors are \((1,-1)e^{\pm it}\) so that \(u_2\) is in opposite phase to~\(u_1\).
The adjoint's eigenvectors are the same.

\begin{reduce}
if thecase=dde2d then begin
ff_:=tp mat((+u2(pi/2)-u1^2,+u1(pi/2)+u2^2));
freqm_:=mat((1,-1));
ee_:=tp mat((1,-1),(1,-1));
zz_:=tp mat((1,-1),(1,-1));
toosmall:=3; factor s,small;
end;
\end{reduce}

\paragraph{The centre manifold}
\begin{math}
\end{math}\par

\begin{math}
u_{1}=s_{2}^{2} \varepsilon  \big(-2/5 \cis\big(-2 t\big) i+1/5 \cis
\big(-2 t\big)\big)-2 s_{2} s_{1} \varepsilon +s_{2} \cis\big(-t\big)+s_
{1}^{2} \varepsilon  \big(2/5 \cis\big(2 t\big) i+1/5 \cis\big(2 t\big)
\big)+s_{1} \cis\big(t\big)
\end{math}\par

\begin{math}
u_{2}=s_{2}^{2} \varepsilon  \big(2/5 \cis\big(-2 t\big) i-1/5 \cis\big(
-2 t\big)\big)+2 s_{2} s_{1} \varepsilon -s_{2} \cis\big(-t\big)+s_{1}^{
2} \varepsilon  \big(-2/5 \cis\big(2 t\big) i-1/5 \cis\big(2 t\big)\big)
-s_{1} \cis\big(t\big)
\end{math}\par


\paragraph{Centre manifold ODEs}
\begin{math}
\end{math}\par

\begin{math}
\dot s_{1}=s_{2} s_{1}^{2} \varepsilon ^{2} \big(-36/5 i \pi -16/5 i-8/5
 \pi +72/5\big)/\big(\pi ^{2}+4\big)
\end{math}\par

\begin{math}
\dot s_{2}=s_{2}^{2} s_{1} \varepsilon ^{2} \big(36/5 i \pi +16/5 i-8/5 
\pi +72/5\big)/\big(\pi ^{2}+4\big)
\end{math}

This model predicts nonlinear growth of the centre modes, in addition to the growth of the unstable mode.


\subsection{Double Hopf 2D DDE}

\cite{Erneux2009} [\S7.2] explored an example of a laser subject to optoelectronic feedback.
For certain parameter values it has a two frequency Hopf bifurcation.

\cite{Erneux2009} [eq.~(7.42)] transformed the laser system to the non-dimensional
\begin{equation*}
(1+\eta)\dd t{\log[1+y]}=-\theta^2
\left[y(t)+\eta y(t-\pi)\right],
\end{equation*}
for parameters \(\eta\) and~\(\theta\).  
\cite{Erneux2009} identified double Hopf bifurcations from the origin at parameters~\((\eta,\theta)\) of \((3/5,2)\), \((7/25,4)\), \((-5/13,2)\) and~\((-9/41,4)\), among others.
Here we work with a system of first order, \dde{}s, so transform the  \dde\ to
\begin{eqnarray*}
&&\dot x=-\theta^2
\left[y(t)+\eta y(t-\pi)\right]/(1+\eta),
\\&& \dot y=[1+y(t)]x(t).
\end{eqnarray*}

The following Octave/Matlab code plots the spectrum for the equilibrium at the origin.  
The results indicate that in all four cases mentioned the centre manifold is attractive.
The gravest eigenvalue being, respectively, \(-0.69\pm i3.87\), \(-0.38\pm i1.02\), \(-0.31\) and \(-0.41\pm i2.03\).
\begin{verbatim}
eta=3/5, theta=2
ce=@(z) (1+eta)*z.^2+theta^2*(1+eta*exp(-pi*z))
lams=fsolve(ce,randn(100,2)*[2;10*i])
plot(real(lams),imag(lams),'o')
\end{verbatim}



Ensure you interpret `left-eigenvectors' as the eigenvectors of the adjoint operator (the complex conjugate transpose of the operator).

\subsubsection{Parameters $(\eta,\theta)=(3/5,2)$}
I invoke a slightly different perturbation of the parameter~\(\eta\) to that of \cite{Erneux2009}.
The eigenvectors are \((1,\mp i/\omega)e^{\pm i\omega t}\) for frequencies \(\omega=1,2\), while the eigenvectors of the adjoint are \((1,\mp i\omega)e^{\pm i\omega t}\). 

\begin{reduce}
if thecase=dde2d2ha then begin
eta:=3/5;
theta:=2*(1+small*delta);
ff_:=tp mat((
    -theta^2*((1/(1+eta)-small*nu)*u2
           +(eta/(1+eta)+small*nu)*u2(pi)),
    +u1*(1+u2)
    ));
freqm_:=mat((1,2,-1,-2));
ee_:=tp mat((1,-i),(1,-i/2),(1,+i),(1,+i/2));
zz_:=tp mat((1,-i),(1,-2*i),(1,+i),(1,+2*i));
toosmall:=3; 
factor s,delta,nu,cis;
end;
\end{reduce}

\paragraph{The centre manifold}
is rather complicated.


\begin{math}
u_{1}=1/6 \cis\big(-4 t\big) s_{4}^{2} \varepsilon  i+3/16 \cis\big(-3 t
\big) s_{4} s_{2} \varepsilon  i+\cis\big(-2 t\big) s_{4}+\cis\big(-2 t
\big) s_{2}^{2} \varepsilon  \big(-9/2 i \pi ^{2}-16 i-6 \pi \big)/\big(
9 \pi ^{2}+64\big)+\cis\big(-t\big) s_{4} s_{1} \varepsilon  \big(9/4 i 
\pi ^{2}+2 i-3/2 \pi \big)/\big(9 \pi ^{2}+16\big)+\cis\big(-t\big) s_{2
}-1/6 \cis\big(4 t\big) s_{3}^{2} \varepsilon  i-3/16 \cis\big(3 t\big) 
s_{3} s_{1} \varepsilon  i+\cis\big(2 t\big) s_{3}+\cis\big(2 t\big) s_{
1}^{2} \varepsilon  \big(9/2 i \pi ^{2}+16 i-6 \pi \big)/\big(9 \pi ^{2}
+64\big)+\cis\big(t\big) s_{3} s_{2} \varepsilon  \big(-9/4 i \pi ^{2}-2
 i-3/2 \pi \big)/\big(9 \pi ^{2}+16\big)+\cis\big(t\big) s_{1}
\end{math}\par

\begin{math}
u_{2}=-1/6 \cis\big(-4 t\big) s_{4}^{2} \varepsilon -9/16 \cis\big(-3 t
\big) s_{4} s_{2} \varepsilon +1/2 \cis\big(-2 t\big) s_{4} i+\cis\big(-
2 t\big) s_{2}^{2} \varepsilon  \big(3 i \pi -9/4 \pi ^{2}-8\big)/\big(9
 \pi ^{2}+64\big)+\cis\big(-t\big) s_{4} s_{1} \varepsilon  \big(3/2 i 
\pi +9/4 \pi ^{2}+2\big)/\big(9 \pi ^{2}+16\big)+\cis\big(-t\big) s_{2} 
i-1/6 \cis\big(4 t\big) s_{3}^{2} \varepsilon -9/16 \cis\big(3 t\big) s_
{3} s_{1} \varepsilon -1/2 \cis\big(2 t\big) s_{3} i+\cis\big(2 t\big) s
_{1}^{2} \varepsilon  \big(-3 i \pi -9/4 \pi ^{2}-8\big)/\big(9 \pi ^{2}
+64\big)+\cis\big(t\big) s_{3} s_{2} \varepsilon  \big(-3/2 i \pi +9/4 
\pi ^{2}+2\big)/\big(9 \pi ^{2}+16\big)-\cis\big(t\big) s_{1} i
\end{math}\par


\paragraph{Centre manifold ODEs}
describe complicated interactions, but mainly it is the coefficients that are complicated functions of~\(\pi\).

\begin{math}
\dot s_{1}=s_{4} s_{3} s_{1} \varepsilon ^{2} \big(-9963/4 i \pi ^{6}-
38340 i \pi ^{4}-167424 i \pi ^{2}-147456 i+21141/16 \pi ^{7}+20007 \pi 
^{5}+84096 \pi ^{3}+61440 \pi \big)/\big(6561 \pi ^{8}+116640 \pi ^{6}+
684288 \pi ^{4}+1474560 \pi ^{2}+1048576\big)+s_{3} s_{2} \varepsilon  
\big(-3 i \pi -4\big)/\big(9 \pi ^{2}+16\big)+s_{2} s_{1}^{2} 
\varepsilon ^{2} \big(-2916 i \pi ^{6}-17280 i \pi ^{4}-3072 i \pi ^{2}-
196608 i-8019/2 \pi ^{7}-44064 \pi ^{5}-93312 \pi ^{3}+122880 \pi \big)/
\big(6561 \pi ^{8}+116640 \pi ^{6}+684288 \pi ^{4}+1474560 \pi ^{2}+
1048576\big)+s_{1} \delta  \varepsilon ^{2} \big(16 i-12 \pi \big)/\big(
9 \pi ^{2}+16\big)+s_{1} \nu  \varepsilon ^{2} \big(-64 i+48 \pi \big)/
\big(9 \pi ^{2}+16\big)
\end{math}\par

\begin{math}
\dot s_{2}=s_{4} s_{3} s_{2} \varepsilon ^{2} \big(9963/4 i \pi ^{6}+
38340 i \pi ^{4}+167424 i \pi ^{2}+147456 i+21141/16 \pi ^{7}+20007 \pi 
^{5}+84096 \pi ^{3}+61440 \pi \big)/\big(6561 \pi ^{8}+116640 \pi ^{6}+
684288 \pi ^{4}+1474560 \pi ^{2}+1048576\big)+s_{4} s_{1} \varepsilon  
\big(3 i \pi -4\big)/\big(9 \pi ^{2}+16\big)+s_{2}^{2} s_{1} 
\varepsilon ^{2} \big(2916 i \pi ^{6}+17280 i \pi ^{4}+3072 i \pi ^{2}+
196608 i-8019/2 \pi ^{7}-44064 \pi ^{5}-93312 \pi ^{3}+122880 \pi \big)/
\big(6561 \pi ^{8}+116640 \pi ^{6}+684288 \pi ^{4}+1474560 \pi ^{2}+
1048576\big)+s_{2} \delta  \varepsilon ^{2} \big(-16 i-12 \pi \big)/
\big(9 \pi ^{2}+16\big)+s_{2} \nu  \varepsilon ^{2} \big(64 i+48 \pi 
\big)/\big(9 \pi ^{2}+16\big)
\end{math}\par

\begin{math}
\dot s_{3}=s_{4} s_{3}^{2} \varepsilon ^{2} \big(-16/3 i-2 \pi \big)/
\big(9 \pi ^{2}+64\big)+s_{3} s_{2} s_{1} \varepsilon ^{2} \big(-34992 i
 \pi ^{6}-252288 i \pi ^{4}-559104 i \pi ^{2}-393216 i-10206 \pi ^{7}-
64800 \pi ^{5}-138240 \pi ^{3}-98304 \pi \big)/\big(6561 \pi ^{8}+116640
 \pi ^{6}+684288 \pi ^{4}+1474560 \pi ^{2}+1048576\big)+s_{3} \delta  
\varepsilon ^{2} \big(128 i+48 \pi \big)/\big(9 \pi ^{2}+64\big)+s_{1}^{
2} \varepsilon  \big(-24 i \pi +64\big)/\big(9 \pi ^{2}+64\big)
\end{math}\par

\begin{math}
\dot s_{4}=s_{4}^{2} s_{3} \varepsilon ^{2} \big(16/3 i-2 \pi \big)/
\big(9 \pi ^{2}+64\big)+s_{4} s_{2} s_{1} \varepsilon ^{2} \big(34992 i 
\pi ^{6}+252288 i \pi ^{4}+559104 i \pi ^{2}+393216 i-10206 \pi ^{7}-
64800 \pi ^{5}-138240 \pi ^{3}-98304 \pi \big)/\big(6561 \pi ^{8}+116640
 \pi ^{6}+684288 \pi ^{4}+1474560 \pi ^{2}+1048576\big)+s_{4} \delta  
\varepsilon ^{2} \big(-128 i+48 \pi \big)/\big(9 \pi ^{2}+64\big)+s_{2}
^{2} \varepsilon  \big(24 i \pi +64\big)/\big(9 \pi ^{2}+64\big)
\end{math}




\subsubsection{Parameters $(\eta,\theta)=(7/25,4)$}
The eigenvectors are \((1,\mp i/\omega)e^{\pm i\omega t}\) for frequencies \(\omega=3,4\), while the eigenvectors of the adjoint are \((1,\mp i\omega)e^{\pm i\omega t}\). 

\begin{reduce}
if thecase=dde2d2hb then begin
eta:=7/25;
theta:=4*(1+small*delta);
ff_:=tp mat((
    -theta^2*((1/(1+eta)-small*nu)*u2
           +(eta/(1+eta)+small*nu)*u2(pi)),
    +u1*(1+u2)
    ));
freqm_:=mat((3,-3,4,-4));
ee_:=tp mat((1,-i/3),(1,+i/3),(1,-i/4),(1,+i/4));
zz_:=tp mat((1,-3*i),(1,+3*i),(1,-4*i),(1,+4*i));
toosmall:=3; 
factor s,delta,nu,cis;
end;
\end{reduce}


\paragraph{The centre manifold}
\begin{math}
\end{math}\par

\begin{math}
u_{1}=1/12 \cis\big(-8 t\big) s_{4}^{2} \varepsilon  i+21/160 \cis\big(-
7 t\big) s_{4} s_{2} \varepsilon  i+4/15 \cis\big(-6 t\big) s_{2}^{2} 
\varepsilon  i+\cis\big(-4 t\big) s_{4}+\cis\big(-3 t\big) s_{2}+3/32 
\cis\big(-t\big) s_{4} s_{1} \varepsilon  i-1/12 \cis\big(8 t\big) s_{3}
^{2} \varepsilon  i-21/160 \cis\big(7 t\big) s_{3} s_{1} \varepsilon  i-
4/15 \cis\big(6 t\big) s_{1}^{2} \varepsilon  i+\cis\big(4 t\big) s_{3}+
\cis\big(3 t\big) s_{1}-3/32 \cis\big(t\big) s_{3} s_{2} \varepsilon  i
\end{math}\par

\begin{math}
u_{2}=-1/24 \cis\big(-8 t\big) s_{4}^{2} \varepsilon -49/480 \cis\big(-7
 t\big) s_{4} s_{2} \varepsilon -1/10 \cis\big(-6 t\big) s_{2}^{2} 
\varepsilon +1/4 \cis\big(-4 t\big) s_{4} i+1/3 \cis\big(-3 t\big) s_{2}
 i-1/96 \cis\big(-t\big) s_{4} s_{1} \varepsilon -1/24 \cis\big(8 t\big)
 s_{3}^{2} \varepsilon -49/480 \cis\big(7 t\big) s_{3} s_{1} 
\varepsilon -1/10 \cis\big(6 t\big) s_{1}^{2} \varepsilon -1/4 \cis\big(
4 t\big) s_{3} i-1/3 \cis\big(3 t\big) s_{1} i-1/96 \cis\big(t\big) s_{3
} s_{2} \varepsilon 
\end{math}\par

\paragraph{Centre manifold ODEs}
\ 

\begin{math}
\dot s_{1}=s_{4} s_{3} s_{1} \varepsilon ^{2} \big(-243/20 i+567/80 \pi 
\big)/\big(49 \pi ^{2}+144\big)+s_{2} s_{1}^{2} \varepsilon ^{2} \big(-
12/5 i+7/5 \pi \big)/\big(49 \pi ^{2}+144\big)+s_{1} \delta  
\varepsilon ^{2} \big(432 i-252 \pi \big)/\big(49 \pi ^{2}+144\big)+s_{1
} \nu  \varepsilon ^{2} \big(-768 i+448 \pi \big)/\big(49 \pi ^{2}+144
\big)
\end{math}\par

\begin{math}
\dot s_{2}=s_{4} s_{3} s_{2} \varepsilon ^{2} \big(243/20 i+567/80 \pi 
\big)/\big(49 \pi ^{2}+144\big)+s_{2}^{2} s_{1} \varepsilon ^{2} \big(12
/5 i+7/5 \pi \big)/\big(49 \pi ^{2}+144\big)+s_{2} \delta  \varepsilon 
^{2} \big(-432 i-252 \pi \big)/\big(49 \pi ^{2}+144\big)+s_{2} \nu  
\varepsilon ^{2} \big(768 i+448 \pi \big)/\big(49 \pi ^{2}+144\big)
\end{math}\par

\begin{math}
\dot s_{3}=s_{4} s_{3}^{2} \varepsilon ^{2} \big(-32/3 i-14/3 \pi \big)/
\big(49 \pi ^{2}+256\big)+s_{3} s_{2} s_{1} \varepsilon ^{2} \big(-256/5
 i-112/5 \pi \big)/\big(49 \pi ^{2}+256\big)+s_{3} \delta  \varepsilon 
^{2} \big(1024 i+448 \pi \big)/\big(49 \pi ^{2}+256\big)
\end{math}\par

\begin{math}
\dot s_{4}=s_{4}^{2} s_{3} \varepsilon ^{2} \big(32/3 i-14/3 \pi \big)/
\big(49 \pi ^{2}+256\big)+s_{4} s_{2} s_{1} \varepsilon ^{2} \big(256/5 
i-112/5 \pi \big)/\big(49 \pi ^{2}+256\big)+s_{4} \delta  \varepsilon ^{
2} \big(-1024 i+448 \pi \big)/\big(49 \pi ^{2}+256\big)
\end{math}

The interaction appears a lot simpler in this case.  
Presumably simpler because the frequencies are `more irrational'.





\subsection{Simple 2D ODE}

Consider the system
\begin{math}
\dot u_{1}=-\varepsilon  u_{1}^{2}+u_{2}-u_{1}
\end{math}
and 
\begin{math}
\dot u_{2}=\varepsilon  u_{2}^{2}-u_{2}+u_{1}
\end{math}\par

\begin{reduce}
if thecase=simple2d then begin
ff_:=tp mat((-u1+u2-u1^2,u1-u2+u2^2));
freqm_:=mat((0));
ee_:=tp mat((1,1));
zz_:=tp mat((1,1));
toosmall:=5; 
end;
\end{reduce}



\paragraph{The centre manifold}

\begin{math}
u_{1}=3/8 \varepsilon ^{3} s_{1}^{4}-1/2 \varepsilon  s_{1}^{2}+s_{1}
\end{math}\par

\begin{math}
u_{2}=-3/8 \varepsilon ^{3} s_{1}^{4}+1/2 \varepsilon  s_{1}^{2}+s_{1}
\end{math}\par

\paragraph{Centre manifold ODEs}

\begin{math}
\dot s_{1}=-3/4 \varepsilon ^{4} s_{1}^{5}+\varepsilon ^{2} s_{1}^{3}
\end{math}\par

\paragraph{Normals to isochrons at the slow manifold}
Use these vectors: to project initial conditions
onto the slow manifold; to project non-autonomous
forcing onto the slow evolution; to predict the
consequences of modifying the original system; in
uncertainty quantification to quantify effects on
the model of uncertainties in the original system.

\(z_{11}=3/2 \eps^{4} s_{1}^{4}+3/4 \eps^{3} s_{1}^{3}-1/2 \eps^{2} s_{1}^{2}-1/2 \eps s_{1}+1/2
\)\par

\(z_{12}=3/2 \eps^{4} s_{1}^{4}-3/4 \eps^{3} s_{1}^{3}-1/2 \eps^{2} s_{1}^{2}+1/2 \eps s_{1}+1/2
\)\par


\subsubsection{The stable manifold}
Appears to get sensible answers even for the stable manifold!
Just invoke this case to characterise the linear stable subspace.
\begin{reduce}
if thecase=simple2ds then begin
ff_:=tp mat((-u1+u2-u1^2,u1-u2+u2^2));
freqm_:=mat((i*2));
ee_:=tp mat((1,-1));
zz_:=tp mat((1,-1));
toosmall:=5;
end;
\end{reduce}

\paragraph{The stable manifold}
where the double factor of~\(i\) in the exponentials give decaying modes of~\(e^{-2t},e^{-6t},e^{-8t}\).

\begin{math}
u_{1}=1/8 \varepsilon ^{3} \cis\big(8 i t\big) s_{1}^{4}+1/4 
\varepsilon ^{2} \cis\big(6 i t\big) s_{1}^{3}+1/2 \varepsilon  \cis
\big(4 i t\big) s_{1}^{2}+\cis\big(2 i t\big) s_{1}
\end{math}\par

\begin{math}
u_{2}=-1/8 \varepsilon ^{3} \cis\big(8 i t\big) s_{1}^{4}-1/4 
\varepsilon ^{2} \cis\big(6 i t\big) s_{1}^{3}-1/2 \varepsilon  \cis
\big(4 i t\big) s_{1}^{2}-\cis\big(2 i t\big) s_{1}
\end{math}\par


\paragraph{Stable manifold ODEs} is the trivial
\begin{math}
\dot s_{1}=0
\end{math}






\subsection{Four state Markov chain}
Variable $\varepsilon$ characterise the perturbation.

\begin{math}
\dot u_{1}=-\varepsilon  u_{1}+u_{2}
\end{math}\par

\begin{math}
\dot u_{2}=\varepsilon  \big(u_{3}-u_{2}+u_{1}\big)-u_{2}
\end{math}\par

\begin{math}
\dot u_{3}=\varepsilon  \big(u_{4}-u_{3}+u_{2}\big)-u_{3}
\end{math}\par

\begin{math}
\dot u_{4}=-\varepsilon  u_{4}+u_{3}
\end{math}\par

The linear perturbation terms gets multiplied by \verb|small| again, but I do not see how to avoid that without wrecking other desirable things: such as, it is useful to multiply some nonlinear terms by small to show they are of higher order than other nonlinear terms. 

\begin{reduce}
if thecase=fourstatemarkov then begin
factor epsilon;
ff_:=tp mat((u2,-u2,-u3,u3))
+small*tp mat((-u1,+u1-u2+u3,+u2-u3+u4,-u4));
freqm_:=mat((0,0));
ee_:=tp mat((0,0,0,1),(1,0,0,0));
zz_:=tp mat((0,0,1,1),(1,1,0,0));
toosmall:=7; 
end;
\end{reduce}


\paragraph{The centre manifold}

\begin{math}
u_{1}=\varepsilon ^{2} \big(2 s_{2}-s_{1}\big)-
\varepsilon  s_{2}+s_{2}
\end{math}\par

\begin{math}
u_{2}=\varepsilon ^{2} \big(-2 s_{2}+s_{1}\big)+
\varepsilon  s_{2}
\end{math}\par

\begin{math}
u_{3}=\varepsilon ^{2} \big(s_{2}-2 s_{1}\big)+
\varepsilon  s_{1}
\end{math}\par

\begin{math}
u_{4}=\varepsilon ^{2} \big(-s_{2}+2 s_{1}\big)-
\varepsilon  s_{1}+s_{1}
\end{math}\par


\paragraph{Centre manifold ODEs}

\begin{math}
\dot s_{1}=\varepsilon ^{3} \big(-3 s_{2}+3 s_{1}\big)+
\varepsilon ^{2} \big(s_{2}-s_{1}\big)
\end{math}\par

\begin{math}
\dot s_{2}=\varepsilon ^{3} \big(3 s_{2}-3 s_{1}\big)+
\varepsilon ^{2} \big(-s_{2}+s_{1}\big)
\end{math}

\paragraph{Normals to isochrons at the slow manifold}
\(
\)\par

\(z_{11}=6 \eps^{6}-\eps^{4}
\)\par

\(z_{12}=19 \eps^{6}-4 \eps^{4}+\eps^{2}
\)\par

\(z_{13}=-19 \eps^{6}+4 \eps^{4}-\eps^{2}+1
\)\par

\(z_{14}=-6 \eps^{6}+\eps^{4}+1
\)\par

\(z_{21}=-6 \eps^{6}+\eps^{4}+1
\)\par

\(z_{22}=-19 \eps^{6}+4 \eps^{4}-\eps^{2}+1
\)\par

\(z_{23}=19 \eps^{6}-4 \eps^{4}+\eps^{2}
\)\par

\(z_{24}=6 \eps^{6}-\eps^{4}
\)\par


\subsection{Bifurcating 2D system}

This example tests labelling a small parameter and having a cubic term labelled as smaller than a quadratic term.

\begin{math}
\dot u_{1}=-\varepsilon ^{2} u_{2} u_{1}^{2}-u_{2}-1/2 u_{1}
\end{math}\par

\begin{math}
\dot u_{2}=\varepsilon  \big(-u_{2}^{2}+u_{2} \epsilon \big)-2 u_{2}-u_{
1}
\end{math}

\begin{reduce}
if thecase=another2d then begin
ff_:=tp mat((
    -u1/2-u2-small*u1^2*u2,
    -u1-2*u2+small*epsilon*u2-u2^2
  ));
freqm_:=mat((0));
ee_:=tp mat((1,-1/2));
zz_:=tp mat((1,-1/2));
end;
\end{reduce}


\paragraph{The centre manifold}

\begin{math}
u_{1}=\varepsilon  \big(-1/25 s_{1}^{2}-2/25 s_{1} \epsilon \big)+s_{1}
\end{math}\par

\begin{math}
u_{2}=\varepsilon  \big(-2/25 s_{1}^{2}-4/25 s_{1} \epsilon \big)-1/2 s_
{1}
\end{math}\par


\paragraph{Centre manifold ODEs}


\begin{math}
\dot s_{1}=\varepsilon ^{2} \big(54/125 s_{1}^{3}+12/125 s_{1}^{2} 
\epsilon +8/125 s_{1} \epsilon ^{2}\big)+\varepsilon  \big(1/10 s_{1}^{2
}+1/5 s_{1} \epsilon \big)
\end{math}

\paragraph{Normals to isochrons at the slow manifold}
\(
\)\par

\(z_{11}=\eps^{2} \big(-352/3125 s_{1}^{2}-8/125 \epsilon \big)-8/125 \eps
 s_{1}+4/5
\)\par

\(z_{12}=\eps^{2} \big(-544/3125 s_{1}^{2}-16/125 \epsilon \big)-16/125 
\eps s_{1}-2/5
\)\par



\subsubsection{The stable manifold}

Appears to also get the stable manifold.
\begin{reduce}
if thecase=another2ds then begin
ff_:=tp mat((
    -u1/2-u2-small*u1^2*u2,
    -u1-2*u2+small*epsilon*u2-u2^2
  ));
freqm_:=mat((i*5/2));
ee_:=tp mat((1,2));
zz_:=tp mat((1,2));
toosmall:=5;
end;
\end{reduce}

\paragraph{The stable manifold}
ignoring the as yet awful formatting of the exponential,

\begin{math}
u_{1}=\varepsilon ^{2} \big(838/1875 \cis\big(\big(15 i t\big)/2\big) s_
{1}^{3}+8/25 \cis\big(\big(5 i t\big)/2\big) s_{1} \epsilon \big)+8/25 
\varepsilon  \cis\big(5 i t\big) s_{1}^{2}+\cis\big(\big(5 i t\big)/2
\big) s_{1}
\end{math}\par

\begin{math}
u_{2}=\varepsilon ^{2} \big(2116/1875 \cis\big(\big(15 i t\big)/2\big) s
_{1}^{3}-4/25 \cis\big(\big(5 i t\big)/2\big) s_{1} \epsilon \big)+36/25
 \varepsilon  \cis\big(5 i t\big) s_{1}^{2}+2 \cis\big(\big(5 i t\big)/2
\big) s_{1}
\end{math}


\paragraph{Stable manifold ODEs}
shows the change in rate due to parameter variation:
\begin{math}
\dot s_{1}=4/5 \varepsilon ^{2} s_{1} \epsilon 
\end{math}







\subsection{Simple 3D system}

This example is straightforward.

\begin{math}
\dot u_{1}=\varepsilon  u_{3} u_{2}+2 u_{3}+u_{2}+2 u_{1}
\end{math}\par

\begin{math}
\dot u_{2}=-\varepsilon  u_{3} u_{1}+u_{3}-u_{2}+u_{1}
\end{math}\par

\begin{math}
\dot u_{3}=-\varepsilon  u_{2} u_{1}-3 u_{3}-u_{2}-3 u_{1}
\end{math}

\begin{reduce}
if thecase=simple3d then begin
ff_:=tp mat((2*u1+u2+2*u3+u2*u3
  ,u1-u2+u3-u1*u3
  ,-3*u1-u2-3*u3-u1*u2));
freqm_:=mat((0));
ee_:=tp mat((1,0,-1));
zz_:=tp mat((4,1,3));
end;
\end{reduce}


\paragraph{The centre manifold} 

\begin{math}
u_{1}=-\varepsilon  s_{1}^{2}+s_{1}
\end{math}\par

\begin{math}
u_{2}=\varepsilon  s_{1}^{2}
\end{math}\par

\begin{math}
u_{3}=\varepsilon  s_{1}^{2}-s_{1}
\end{math}\par


\paragraph{Centre manifold ODEs} 

\begin{math}
\dot s_{1}=-9 \varepsilon ^{2} s_{1}^{3}+\varepsilon  s_{1}^{2}
\end{math}

\paragraph{Normals to isochrons at the slow manifold}
\(
\)\par

\(z_{11}=258 \eps^{2} s_{1}^{2}-16 \eps s_{1}+4
\)\par

\(z_{12}=93 \eps^{2} s_{1}^{2}-9 \eps s_{1}+1
\)\par

\(z_{13}=240 \eps^{2} s_{1}^{2}-16 \eps s_{1}+3
\)\par





\subsubsection{Its 2D stable manifold with generalised eigenvectors}

Despite the generalised eigenvectors, the following alternative appears to generate the stable manifold if you wish:
\begin{reduce}
if thecase=simple3ds then begin
ff_:=tp mat((2*u1+u2+2*u3+u2*u3
  ,u1-u2+u3-u1*u3
  ,-3*u1-u2-3*u3-u1*u2));
freqm_:=mat((i,i));
ee_:=tp mat((1,-1,-1),(1,7/2,-5/2));
zz_:=tp mat((0,1,0),(1,0,1));
end;
\end{reduce}

\paragraph{The adjusted dynamical system}
Modified in order cater for the generalised eigenvector.

\begin{math}
\dot u_{1}=\varepsilon  \big(u_{3} u_{2}-u_{3}-u_{1}\big)+3 u_{3}+u_{2}+
3 u_{1}
\end{math}\par

\begin{math}
\dot u_{2}=\varepsilon  \big(-u_{3} u_{1}+u_{3}+u_{1}\big)-u_{2}
\end{math}\par

\begin{math}
\dot u_{3}=\varepsilon  \big(u_{3}-u_{2} u_{1}+u_{1}\big)-4 u_{3}-u_{2}-
4 u_{1}
\end{math}

\paragraph{The stable manifold}
noting the double~\(i\) factors give decaying modes.

\begin{math}
u_{1}=\varepsilon  \big(-51/4 \cis\big(2 i t\big) s_{2}^{2}-3 \cis\big(2
 i t\big) s_{2} s_{1}+3 \cis\big(2 i t\big) s_{1}^{2}\big)+\cis\big(i t
\big) s_{2}+\cis\big(i t\big) s_{1}
\end{math}\par

\begin{math}
u_{2}=\varepsilon  \big(-5/2 \cis\big(2 i t\big) s_{2}^{2}-7/2 \cis\big(
2 i t\big) s_{2} s_{1}-\cis\big(2 i t\big) s_{1}^{2}\big)+7/2 \cis\big(i
 t\big) s_{2}-\cis\big(i t\big) s_{1}
\end{math}\par

\begin{math}
u_{3}=\varepsilon  \big(25 \cis\big(2 i t\big) s_{2}^{2}+13/2 \cis\big(2
 i t\big) s_{2} s_{1}-5 \cis\big(2 i t\big) s_{1}^{2}\big)-5/2 \cis\big(
i t\big) s_{2}-\cis\big(i t\big) s_{1}
\end{math}\par


\paragraph{Stable manifold ODEs}
\begin{math}
\dot s_{1}=3/2 \varepsilon  s_{2}
\end{math}
and
\begin{math}
\dot s_{2}=0
\end{math}




\subsection{3D system with a generalised eigenvector}
Took longer to converge, but converge it does.
However, now I force the off-diagonal term to be small.

\begin{math}
\dot u_{1}=\varepsilon  \big(u_{3} u_{2}+u_{3}+u_{2}+u_{1}\big)+u_{3}+u_
{1}
\end{math}\par

\begin{math}
\dot u_{2}=-\varepsilon  u_{3} u_{1}+u_{3}+u_{1}
\end{math}\par

\begin{math}
\dot u_{3}=\varepsilon  \big(-u_{3}-u_{2} u_{1}-u_{2}-u_{1}\big)-2 u_{3}
-2 u_{1}
\end{math}

\begin{reduce}
if thecase=geneigenvec then begin
ff_:=tp mat((
    2*u1+u2+2*u3+u2*u3,
    u1+u3-u1*u3,
    -3*u1-u2-3*u3-u1*u2
    ));
freqm_:=mat((0,0));
ee_:=tp mat((1,0,-1),(0,1,0));
zz_:=tp mat((1,-1,0),(1,1,1));
toosmall:=3;
end;
\end{reduce}

\paragraph{The centre manifold} 

\begin{math}
u_{1}=2 \varepsilon  s_{2} s_{1}+s_{1}
\end{math}\par

\begin{math}
u_{2}=2 \varepsilon  s_{2} s_{1}+s_{2}
\end{math}\par

\begin{math}
u_{3}=-4 \varepsilon  s_{2} s_{1}-s_{1}
\end{math}\par
 
\paragraph{Centre manifold ODEs}

\begin{math}
\dot s_{1}=\varepsilon ^{2} \big(-10 s_{2}^{2} s_{1}-6 s_{2} s_{1}^{2}
\big)+\varepsilon  \big(-3 s_{2} s_{1}+s_{2}\big)
\end{math}\par

\begin{math}
\dot s_{2}=\varepsilon ^{2} \big(-6 s_{2}^{2} s_{1}+2 s_{2} s_{1}^{2}
\big)+\varepsilon  \big(-2 s_{2} s_{1}+s_{1}^{2}\big)
\end{math}

\paragraph{Normals to isochrons at the slow manifold}
\(
\)\par

\(z_{11}=\eps^{2} \big(50 s_{2}^{2}+60 s_{2} s_{1}+14 s_{1}^{2}+s_{1}\big)
+\eps \big(5 s_{2}+3 s_{1}\big)+2
\)\par

\(z_{12}=\eps^{2} \big(10 s_{2} s_{1}+6 s_{1}^{2}\big)
\)\par

\(z_{13}=\eps^{2} \big(40 s_{2}^{2}+54 s_{2} s_{1}+14 s_{1}^{2}+s_{1}\big)
+\eps \big(5 s_{2}+3 s_{1}\big)+1
\)\par

\(z_{21}=\eps^{2} \big(31 s_{2}^{2}+8 s_{2} s_{1}-s_{2}-9 s_{1}^{2}\big)+
\eps \big(3 s_{2}-s_{1}\big)+1
\)\par

\(z_{22}=\eps^{2} \big(6 s_{2} s_{1}-2 s_{1}^{2}\big)+1
\)\par

\(z_{23}=\eps^{2} \big(25 s_{2}^{2}+10 s_{2} s_{1}-s_{2}-9 s_{1}^{2}\big)+
\eps \big(3 s_{2}-s_{1}\big)+1
\)\par





\subsection{Separated system}
To see if small part in the slow variable ruins convergence.
The answer is that it did---hence we include code to make anything non-oscillatory in the slow variables to be small.
Also test a non-zero constant forcing.

\begin{math}
\dot u_{1}=\varepsilon  \big(-u_{2} u_{1}+u_{1} \alpha\big)
\end{math}\par

\begin{math}
\dot u_{2}=\varepsilon  \big(\beta -2 u_{2}^{2}+u_{1}^{2}\big)-u_{2}
\end{math}

\begin{reduce}
if thecase=bifurcate2d then begin
ff_:=tp mat((
    alpha*u1-u1*u2,
    -u2+u1^2-2*u2^2+beta
    ));
freqm_:=mat((0));
ee_:=tp mat((1,0));
zz_:=tp mat((1,0));
toosmall:=4;
end;
\end{reduce}

\paragraph{The centre manifold} 

\begin{math}
u_{1}=s_{1}
\end{math}\par

\begin{math}
u_{2}=\varepsilon ( s_{1}^{2}+\beta)
\end{math}\par
 
\paragraph{Centre manifold ODEs} 

\begin{math}
\dot s_{1}=-\varepsilon ^{2} (s_{1}^{3}-\beta s_1)+\varepsilon  s_{1} \alpha
\end{math}

\paragraph{Normals to isochrons at the slow manifold}
\(
\)\par

\(z_{11}=2 \eps^{2} s_{1}^{2}+1
\)\par

\(z_{12}=-\eps s_{1}
\)\par




 
\subsection{Oscillatory centre manifold---separated form}

Let's try complex eigenvectors.
Adjoint eigenvectors~\verb|zz_| must be the eigenvectors of the complex conjugate transpose matrix.

\begin{math}
\dot u_{1}=u_{2}
\end{math}\par

\begin{math}
\dot u_{2}=-\varepsilon  u_{3} u_{1}-u_{1}
\end{math}\par

\begin{math}
\dot u_{3}=5 \varepsilon  u_{1}^{2}-u_{3}
\end{math}

\begin{reduce}
if thecase=simpleosc then begin
ff_:=tp mat((u2,-u1-u1*u3,-u3+5*u1^2));
freqm_:=mat((1,-1));
ee_:=tp mat((1,+i,0),(1,-i,0));
zz_:=tp mat((1,+i,0),(1,-i,0));
end;
\end{reduce}

\paragraph{The centre manifold} 

\begin{math}
u_{1}=\cis\big(-t\big) s_{2}+\cis\big(t\big) s_{1}
\end{math}\par

\begin{math}
u_{2}=-\cis\big(-t\big) s_{2} i+\cis\big(t\big) s_{1} i
\end{math}\par

\begin{math}
u_{3}=\varepsilon  \big(2 \cis\big(-2 t\big) s_{2}^{2} i+\cis\big(-2 t
\big) s_{2}^{2}-2 \cis\big(2 t\big) s_{1}^{2} i+\cis\big(2 t\big) s_{1}
^{2}+10 s_{2} s_{1}\big)
\end{math}\par
 
\paragraph{Centre manifold ODEs} 
\begin{math}
\dot s_{1}=\varepsilon ^{2} \big(11/2 s_{2} s_{1}^{2} i+s_{2} s_{1}^{2}
\big)
\end{math}\par

\begin{math}
\dot s_{2}=\varepsilon ^{2} \big(-11/2 s_{2}^{2} s_{1} i+s_{2}^{2} s_{1}
\big)
\end{math}




\subsection{Perturbed frequency oscillatory centre manifold---separated form}
  Putting real parameters into the linear operator works here also.

\begin{math}
\dot u_{1}=\varepsilon  \big(u_{2} b+u_{1} a\big)+u_{2}
\end{math}\par

\begin{math}
\dot u_{2}=\varepsilon  \big(u_{2} d-u_{1} c\big)-u_{1}
\end{math}\par

\begin{math}
\dot u_{3}=-u_{3}
\end{math}

\begin{reduce}
if thecase=perturbfreq then begin
ff_:=tp mat((a*u1+(1+b)*u2,d*u2-(1+c)*u1,-u3));
freqm_:=mat((1,-1));
ee_:=tp mat((1,+i,0),(1,-i,0));
zz_:=tp mat((1,+i,0),(1,-i,0));
end;
\end{reduce}

\paragraph{The centre manifold} 

\begin{math}
u_{1}=\varepsilon  \big(1/4 \cis\big(-t\big) s_{2} a i+1/4 \cis\big(-t
\big) s_{2} b-1/4 \cis\big(-t\big) s_{2} c-1/4 \cis\big(-t\big) s_{2} d 
i-1/4 \cis\big(t\big) s_{1} a i+1/4 \cis\big(t\big) s_{1} b-1/4 \cis
\big(t\big) s_{1} c+1/4 \cis\big(t\big) s_{1} d i\big)+\cis\big(-t\big) 
s_{2}+\cis\big(t\big) s_{1}
\end{math}\par

\begin{math}
u_{2}=\varepsilon  \big(-1/4 \cis\big(-t\big) s_{2} a+1/4 \cis\big(-t
\big) s_{2} b i-1/4 \cis\big(-t\big) s_{2} c i+1/4 \cis\big(-t\big) s_{2
} d-1/4 \cis\big(t\big) s_{1} a-1/4 \cis\big(t\big) s_{1} b i+1/4 \cis
\big(t\big) s_{1} c i+1/4 \cis\big(t\big) s_{1} d\big)-\cis\big(-t\big) 
s_{2} i+\cis\big(t\big) s_{1} i
\end{math}\par

\begin{math}
u_{3}=0
\end{math}\par

\paragraph{Centre manifold ODEs}
\begin{math}
\end{math}\par

\begin{math}
\dot s_{1}=\varepsilon ^{2} \big(-1/8 s_{1} a^{2} i+1/4 s_{1} a d i-1/8 
s_{1} b^{2} i+1/4 s_{1} b c i-1/8 s_{1} c^{2} i-1/8 s_{1} d^{2} i\big)+
\varepsilon  \big(1/2 s_{1} a+1/2 s_{1} b i+1/2 s_{1} c i+1/2 s_{1} d
\big)
\end{math}\par

\begin{math}
\dot s_{2}=\varepsilon ^{2} \big(1/8 s_{2} a^{2} i-1/4 s_{2} a d i+1/8 s
_{2} b^{2} i-1/4 s_{2} b c i+1/8 s_{2} c^{2} i+1/8 s_{2} d^{2} i\big)+
\varepsilon  \big(1/2 s_{2} a-1/2 s_{2} b i-1/2 s_{2} c i+1/2 s_{2} d
\big)
\end{math}





\subsection{More general oscillatory centre manifold}
Consider the frequency two dynamics of the following system in  non-separated form.

\begin{math}
\dot u_{1}=\varepsilon  \big(u_{2} u_{1}+u_{1} \epsilon \big)-2 u_{3}-2 
u_{2}
\end{math}\par

\begin{math}
\dot u_{2}=-2 u_{3}-3 u_{2}+u_{1}
\end{math}\par

\begin{math}
\dot u_{3}=2 u_{3}+3 u_{2}+u_{1}
\end{math}

\begin{reduce}
if thecase=nonseparatedosc then begin
ff_:=tp mat((
    -2*u2-2*u3+epsilon*u1+u1*u2,
    u1-3*u2-2*u3,
    u1+3*u2+2*u3
    ));
freqm_:=mat((+2,-2));
ee_:=tp mat((1,1,-1-i),(1,1,-1+i));
zz_:=tp mat((1,-i,-i),(1,+i,+i));
end;
\end{reduce}

\paragraph{The centre manifold} 

\begin{math}
u_{1}=\varepsilon  \big(1/3 \cis\big(-4 t\big) s_{2}^{2} i+1/8 \cis\big(
-2 t\big) s_{2} \epsilon  i-1/3 \cis\big(4 t\big) s_{1}^{2} i-1/8 \cis
\big(2 t\big) s_{1} \epsilon  i\big)+\cis\big(-2 t\big) s_{2}+\cis\big(2
 t\big) s_{1}
\end{math}\par

\begin{math}
u_{2}=\varepsilon  \big(5/51 \cis\big(-4 t\big) s_{2}^{2} i-1/17 \cis
\big(-4 t\big) s_{2}^{2}-11/40 \cis\big(-2 t\big) s_{2} \epsilon  i-1/5 
\cis\big(-2 t\big) s_{2} \epsilon -5/51 \cis\big(4 t\big) s_{1}^{2} i-1/
17 \cis\big(4 t\big) s_{1}^{2}+11/40 \cis\big(2 t\big) s_{1} \epsilon  i
-1/5 \cis\big(2 t\big) s_{1} \epsilon -2 s_{2} s_{1}\big)+\cis\big(-2 t
\big) s_{2}+\cis\big(2 t\big) s_{1}
\end{math}\par

\begin{math}
u_{3}=\varepsilon  \big(-5/51 \cis\big(-4 t\big) s_{2}^{2} i-11/102 \cis
\big(-4 t\big) s_{2}^{2}+11/40 \cis\big(-2 t\big) s_{2} \epsilon  i+13/
40 \cis\big(-2 t\big) s_{2} \epsilon +5/51 \cis\big(4 t\big) s_{1}^{2} i
-11/102 \cis\big(4 t\big) s_{1}^{2}-11/40 \cis\big(2 t\big) s_{1} 
\epsilon  i+13/40 \cis\big(2 t\big) s_{1} \epsilon +3 s_{2} s_{1}\big)+
\cis\big(-2 t\big) s_{2} i-\cis\big(-2 t\big) s_{2}-\cis\big(2 t\big) s_
{1} i-\cis\big(2 t\big) s_{1}
\end{math}\par
 
\paragraph{Centre manifold ODEs} 

\begin{math}
\dot s_{1}=\varepsilon ^{2} \big(-11/51 s_{2} s_{1}^{2} i-35/34 s_{2} s_
{1}^{2}-1/16 s_{1} \epsilon ^{2} i\big)+1/2 \varepsilon  s_{1} \epsilon 
\end{math}\par

\begin{math}
\dot s_{2}=\varepsilon ^{2} \big(11/51 s_{2}^{2} s_{1} i-35/34 s_{2}^{2}
 s_{1}+1/16 s_{2} \epsilon ^{2} i\big)+1/2 \varepsilon  s_{2} \epsilon 
\end{math}






\subsection{Quasi-delay differential equation}
Shows Hopf bifurcation as parameter~$a$ crosses~$-4$ to oscillations with base frequency two.

\begin{math}
\dot u_{1}=\varepsilon ^{2} \big(-u_{3} \alpha -u_{1}^{3}\big)-2 
\varepsilon  u_{1}^{2}-4 u_{3}
\end{math}\par

\begin{math}
\dot u_{2}=-2 u_{2}+2 u_{1}
\end{math}\par

\begin{math}
\dot u_{3}=-2 u_{3}+2 u_{2}
\end{math}

\begin{reduce}
if thecase=quasidelayosc then begin
ff_:=tp mat((
    -4*u3-small*alpha*u3-2*u1^2-small*u1^3,
    2*u1-2*u2,
    2*u2-2*u3
    ));
freqm_:=mat((2,-2));
ee_:=tp mat((1,1/2-i/2,-i/2),(1,1/2+i/2,+i/2));
zz_:=tp mat((1,-i,-1-i),(1,+i,-1+i));
end;
\end{reduce}

\paragraph{The centre manifold}

\begin{math}
u_{1}=\varepsilon  \big(-7/12 \cis\big(-4 t\big) s_{2}^{2} i+1/12 \cis
\big(-4 t\big) s_{2}^{2}+7/12 \cis\big(4 t\big) s_{1}^{2} i+1/12 \cis
\big(4 t\big) s_{1}^{2}-s_{2} s_{1}\big)+\cis\big(-2 t\big) s_{2}+\cis
\big(2 t\big) s_{1}
\end{math}\par

\begin{math}
u_{2}=\varepsilon  \big(-1/12 \cis\big(-4 t\big) s_{2}^{2} i+1/4 \cis
\big(-4 t\big) s_{2}^{2}+1/12 \cis\big(4 t\big) s_{1}^{2} i+1/4 \cis
\big(4 t\big) s_{1}^{2}-s_{2} s_{1}\big)+1/2 \cis\big(-2 t\big) s_{2} i+
1/2 \cis\big(-2 t\big) s_{2}-1/2 \cis\big(2 t\big) s_{1} i+1/2 \cis\big(
2 t\big) s_{1}
\end{math}\par

\begin{math}
u_{3}=\varepsilon  \big(1/12 \cis\big(-4 t\big) s_{2}^{2} i+1/12 \cis
\big(-4 t\big) s_{2}^{2}-1/12 \cis\big(4 t\big) s_{1}^{2} i+1/12 \cis
\big(4 t\big) s_{1}^{2}-s_{2} s_{1}\big)+1/2 \cis\big(-2 t\big) s_{2} i-
1/2 \cis\big(2 t\big) s_{1} i
\end{math}\par
 
\paragraph{Centre manifold ODEs} 
\begin{math}
\dot s_{1}=\varepsilon ^{2} \big(-16/15 s_{2} s_{1}^{2} i-1/5 s_{2} s_{1
}^{2}+1/5 s_{1} \alpha  i+1/10 s_{1} \alpha \big)
\end{math}\par

\begin{math}
\dot s_{2}=\varepsilon ^{2} \big(16/15 s_{2}^{2} s_{1} i-1/5 s_{2}^{2} s
_{1}-1/5 s_{2} \alpha  i+1/10 s_{2} \alpha \big)
\end{math}






\subsection{Detuned version of quasi-delayed}
The following modified version of the previous shows that we can `detune' the linear operator and my `adjustment' of the linear operator seems to work.  
Here the $1/2$ in $\mathcal L_{1,1}$ should be zero for these eigenvectors: my adjustment seems to fix it OK.
But now, knowing the frequencies, my adjustment is different (and probably better).

\begin{math}
\dot u_{1}=\varepsilon ^{2} \big(-u_{3} \alpha -u_{1}^{3}\big)+
\varepsilon  \big(-1/5 u_{3}+1/5 u_{2}-2 u_{1}^{2}+2/5 u_{1}\big)-19/5 u
_{3}-1/5 u_{2}+1/10 u_{1}
\end{math}\par

\begin{math}
\dot u_{2}=-2 u_{2}+2 u_{1}
\end{math}\par

\begin{math}
\dot u_{3}=-2 u_{3}+2 u_{2}
\end{math}

\begin{reduce}
if thecase=quasidelayoscmod then begin
ff_:=tp mat((
    u1/2-4*u3-small*alpha*u3-2*u1^2-small*u1^3,
    2*u1-2*u2,
    2*u2-2*u3
    ));
freqm_:=mat((2,-2));
ee_:=tp mat((1,1/2-i/2,-i/2),(1,1/2+i/2,+i/2));
zz_:=tp mat((1,-i,-1-i),(1,+i,-1+i));
toosmall:=3;
end;
\end{reduce}



\paragraph{The centre manifold}
\begin{math}
\end{math}\par

\begin{math}
u_{1}=\varepsilon  \big(-1840/3121 \cis\big(-4 t\big) s_{2}^{2} i+860/
9363 \cis\big(-4 t\big) s_{2}^{2}+237/3842 \cis\big(-2 t\big) s_{2} i+87
/1921 \cis\big(-2 t\big) s_{2}+1840/3121 \cis\big(4 t\big) s_{1}^{2} i+
860/9363 \cis\big(4 t\big) s_{1}^{2}-237/3842 \cis\big(2 t\big) s_{1} i+
87/1921 \cis\big(2 t\big) s_{1}-40/39 s_{2} s_{1}\big)+\cis\big(-2 t
\big) s_{2}+\cis\big(2 t\big) s_{1}
\end{math}\par

\begin{math}
u_{2}=\varepsilon  \big(-760/9363 \cis\big(-4 t\big) s_{2}^{2} i+2380/
9363 \cis\big(-4 t\big) s_{2}^{2}+21/7684 \cis\big(-2 t\big) s_{2} i+137
/7684 \cis\big(-2 t\big) s_{2}+760/9363 \cis\big(4 t\big) s_{1}^{2} i+
2380/9363 \cis\big(4 t\big) s_{1}^{2}-21/7684 \cis\big(2 t\big) s_{1} i+
137/7684 \cis\big(2 t\big) s_{1}-40/39 s_{2} s_{1}\big)+1/2 \cis\big(-2 
t\big) s_{2} i+1/2 \cis\big(-2 t\big) s_{2}-1/2 \cis\big(2 t\big) s_{1} 
i+1/2 \cis\big(2 t\big) s_{1}
\end{math}\par

\begin{math}
u_{3}=\varepsilon  \big(800/9363 \cis\big(-4 t\big) s_{2}^{2} i+260/3121
 \cis\big(-4 t\big) s_{2}^{2}-4/1921 \cis\big(-2 t\big) s_{2} i+353/7684
 \cis\big(-2 t\big) s_{2}-800/9363 \cis\big(4 t\big) s_{1}^{2} i+260/
3121 \cis\big(4 t\big) s_{1}^{2}+4/1921 \cis\big(2 t\big) s_{1} i+353/
7684 \cis\big(2 t\big) s_{1}-40/39 s_{2} s_{1}\big)+1/2 \cis\big(-2 t
\big) s_{2} i-1/2 \cis\big(2 t\big) s_{1} i
\end{math}\par


\paragraph{Centre manifold ODEs}
\begin{math}
\end{math}\par

\begin{math}
\dot s_{1}=\varepsilon ^{2} \big(-259684400/233822199 s_{2} s_{1}^{2} i-
1154340/5995441 s_{2} s_{1}^{2}+390/1921 s_{1} \alpha  i+200/1921 s_{1} 
\alpha -90446425/7088952961 s_{1} i-1300360/7088952961 s_{1}\big)+
\varepsilon  \big(-200/1921 s_{1} i+390/1921 s_{1}\big)
\end{math}\par

\begin{math}
\dot s_{2}=\varepsilon ^{2} \big(259684400/233822199 s_{2}^{2} s_{1} i-
1154340/5995441 s_{2}^{2} s_{1}-390/1921 s_{2} \alpha  i+200/1921 s_{2} 
\alpha +90446425/7088952961 s_{2} i-1300360/7088952961 s_{2}\big)+
\varepsilon  \big(200/1921 s_{2} i+390/1921 s_{2}\big)
\end{math}

Observe the terms linear in~\(\varepsilon\) due to my fudging of the linear dynamics.




\subsection{Rossler-like system}
Has Hopf bifurcation as parameter crosses zero to oscillations of base frequency one.
 
 \begin{math}
\dot u_{1}=-u_{3}-u_{2}
\end{math}\par

\begin{math}
\dot u_{2}=\varepsilon  u_{2} a+u_{1}
\end{math}\par

\begin{math}
\dot u_{3}=\varepsilon  \big(u_{3} u_{1}-1/5 u_{2} u_{1}\big)-5 u_{3}
\end{math}

\begin{reduce}
if thecase=rosslerlike then begin
ff_:=tp mat((
    -u2-u3,
    u1+small*a*u2,
    -5*u3-u1*u2/5+u1*u3
    ));
freqm_:=mat((1,-1));
ee_:=tp mat((1,-i,0),(1,i,0));
zz_:=tp mat((-5+i,1+5*i,1),(-5-i,1-5*i,1));
end;
\end{reduce}

\paragraph{The centre manifold} 

\begin{math}
u_{1}=\varepsilon  \big(-4/435 \cis\big(-2 t\big) s_{2}^{2} i-2/87 \cis
\big(-2 t\big) s_{2}^{2}-1/4 \cis\big(-t\big) s_{2} a i+4/435 \cis\big(2
 t\big) s_{1}^{2} i-2/87 \cis\big(2 t\big) s_{1}^{2}+1/4 \cis\big(t\big)
 s_{1} a i\big)+\cis\big(-t\big) s_{2}+\cis\big(t\big) s_{1}
\end{math}\par

\begin{math}
u_{2}=\varepsilon  \big(-1/87 \cis\big(-2 t\big) s_{2}^{2} i+2/435 \cis
\big(-2 t\big) s_{2}^{2}-1/4 \cis\big(-t\big) s_{2} a+1/87 \cis\big(2 t
\big) s_{1}^{2} i+2/435 \cis\big(2 t\big) s_{1}^{2}-1/4 \cis\big(t\big) 
s_{1} a\big)+\cis\big(-t\big) s_{2} i-\cis\big(t\big) s_{1} i
\end{math}\par

\begin{math}
u_{3}=\varepsilon  \big(-1/29 \cis\big(-2 t\big) s_{2}^{2} i+2/145 \cis
\big(-2 t\big) s_{2}^{2}+1/29 \cis\big(2 t\big) s_{1}^{2} i+2/145 \cis
\big(2 t\big) s_{1}^{2}\big)
\end{math}\par
 
\paragraph{Centre manifold ODEs} 

\begin{math}
\dot s_{1}=\varepsilon ^{2} \big(-92/28275 s_{2} s_{1}^{2} i-4/1885 s_{2
} s_{1}^{2}-1/8 s_{1} a^{2} i\big)+1/2 \varepsilon  s_{1} a
\end{math}\par

\begin{math}
\dot s_{2}=\varepsilon ^{2} \big(92/28275 s_{2}^{2} s_{1} i-4/1885 s_{2}
^{2} s_{1}+1/8 s_{2} a^{2} i\big)+1/2 \varepsilon  s_{2} a
\end{math}






\subsection{Fudge a couple of these oscillations together}
Use say different base frequencies of one and two.
Put in a couple of coupling terms.
It seems to work fine, although the computation time zooms up even for the basic third order errors.

\begin{math}
\dot u_{1}=\varepsilon  u_{4}^{2}-u_{3}-u_{2}
\end{math}\par

\begin{math}
\dot u_{2}=\varepsilon  u_{2} a+u_{1}
\end{math}\par

\begin{math}
\dot u_{3}=\varepsilon  \big(u_{3} u_{1}-1/5 u_{2} u_{1}\big)-5 u_{3}
\end{math}\par

\begin{math}
\dot u_{4}=\varepsilon  \big(u_{6} u_{5}+u_{4} \epsilon \big)-2 u_{6}-2 
u_{5}
\end{math}\par

\begin{math}
\dot u_{5}=\varepsilon  u_{1}^{2}-2 u_{6}-3 u_{5}+u_{4}
\end{math}\par

\begin{math}
\dot u_{6}=2 u_{6}+3 u_{5}+u_{4}
\end{math}

\begin{reduce}
if thecase=doubleosc then begin
ff_:=tp mat((
  -u2-u3+u4^2,
  u1+a*u2,
  -5*u3-u1*u2/5+u1*u3,
  -2*u5-2*u6+small*epsilon*u4+u5*u6,
  u4-3*u5-2*u6+u1^2,
  u4+3*u5+2*u6
  ));
freqm_:=mat((1,-1,2,-2));
ee_:=tp mat((1,-i,0,0,0,0),(1,i,0,0,0,0)
  ,(0,0,0,1,1,-1-i),(0,0,0,1,1,-1+i));
zz_:=tp mat((-5+i,1+5*i,1,0,0,0),(-5-i,1-5*i,1,0,0,0)
  ,(0,0,0,1,-i,-i),(0,0,0,1,+i,+i));
end;
\end{reduce}

\paragraph{The centre manifold} 

\begin{math}
u_{1}=\varepsilon  \big(4/15 \cis\big(-4 t\big) s_{4}^{2} i-4/435 \cis
\big(-2 t\big) s_{2}^{2} i-2/87 \cis\big(-2 t\big) s_{2}^{2}-1/4 \cis
\big(-t\big) s_{2} a i-4/15 \cis\big(4 t\big) s_{3}^{2} i+4/435 \cis
\big(2 t\big) s_{1}^{2} i-2/87 \cis\big(2 t\big) s_{1}^{2}+1/4 \cis\big(
t\big) s_{1} a i\big)+\cis\big(-t\big) s_{2}+\cis\big(t\big) s_{1}
\end{math}\par

\begin{math}
u_{2}=\varepsilon  \big(-1/15 \cis\big(-4 t\big) s_{4}^{2}-1/87 \cis
\big(-2 t\big) s_{2}^{2} i+2/435 \cis\big(-2 t\big) s_{2}^{2}-1/4 \cis
\big(-t\big) s_{2} a-1/15 \cis\big(4 t\big) s_{3}^{2}+1/87 \cis\big(2 t
\big) s_{1}^{2} i+2/435 \cis\big(2 t\big) s_{1}^{2}-1/4 \cis\big(t\big) 
s_{1} a+2 s_{4} s_{3}\big)+\cis\big(-t\big) s_{2} i-\cis\big(t\big) s_{1
} i
\end{math}\par

\begin{math}
u_{3}=\varepsilon  \big(-1/29 \cis\big(-2 t\big) s_{2}^{2} i+2/145 \cis
\big(-2 t\big) s_{2}^{2}+1/29 \cis\big(2 t\big) s_{1}^{2} i+2/145 \cis
\big(2 t\big) s_{1}^{2}\big)
\end{math}\par

\begin{math}
u_{4}=\varepsilon  \big(-1/3 \cis\big(-4 t\big) s_{4}^{2} i-1/3 \cis
\big(-4 t\big) s_{4}^{2}+1/8 \cis\big(-2 t\big) s_{4} \epsilon  i-1/8 
\cis\big(-2 t\big) s_{2}^{2}+1/3 \cis\big(4 t\big) s_{3}^{2} i-1/3 \cis
\big(4 t\big) s_{3}^{2}-1/8 \cis\big(2 t\big) s_{3} \epsilon  i-1/8 \cis
\big(2 t\big) s_{1}^{2}-s_{2} s_{1}\big)+\cis\big(-2 t\big) s_{4}+\cis
\big(2 t\big) s_{3}
\end{math}\par

\begin{math}
u_{5}=\varepsilon  \big(-8/51 \cis\big(-4 t\big) s_{4}^{2} i-2/51 \cis
\big(-4 t\big) s_{4}^{2}-11/40 \cis\big(-2 t\big) s_{4} \epsilon  i-1/5 
\cis\big(-2 t\big) s_{4} \epsilon +2/5 \cis\big(-2 t\big) s_{2}^{2} i+3/
40 \cis\big(-2 t\big) s_{2}^{2}+8/51 \cis\big(4 t\big) s_{3}^{2} i-2/51 
\cis\big(4 t\big) s_{3}^{2}+11/40 \cis\big(2 t\big) s_{3} \epsilon  i-1/
5 \cis\big(2 t\big) s_{3} \epsilon -2/5 \cis\big(2 t\big) s_{1}^{2} i+3/
40 \cis\big(2 t\big) s_{1}^{2}+2 s_{4} s_{3}+s_{2} s_{1}\big)+\cis\big(-
2 t\big) s_{4}+\cis\big(2 t\big) s_{3}
\end{math}\par

\begin{math}
u_{6}=\varepsilon  \big(-1/102 \cis\big(-4 t\big) s_{4}^{2} i+7/34 \cis
\big(-4 t\big) s_{4}^{2}+11/40 \cis\big(-2 t\big) s_{4} \epsilon  i+13/
40 \cis\big(-2 t\big) s_{4} \epsilon -11/40 \cis\big(-2 t\big) s_{2}^{2}
 i-3/40 \cis\big(-2 t\big) s_{2}^{2}+1/102 \cis\big(4 t\big) s_{3}^{2} i
+7/34 \cis\big(4 t\big) s_{3}^{2}-11/40 \cis\big(2 t\big) s_{3} 
\epsilon  i+13/40 \cis\big(2 t\big) s_{3} \epsilon +11/40 \cis\big(2 t
\big) s_{1}^{2} i-3/40 \cis\big(2 t\big) s_{1}^{2}-3 s_{4} s_{3}-s_{2} s
_{1}\big)+\cis\big(-2 t\big) s_{4} i-\cis\big(-2 t\big) s_{4}-\cis\big(2
 t\big) s_{3} i-\cis\big(2 t\big) s_{3}
\end{math}\par
 
\paragraph{Centre manifold ODEs} 

\begin{math}
\dot s_{1}=\varepsilon ^{2} \big(-1/130 s_{4} s_{3} s_{1} i+1/26 s_{4} s
_{3} s_{1}-92/28275 s_{2} s_{1}^{2} i-4/1885 s_{2} s_{1}^{2}-1/8 s_{1} a
^{2} i\big)+1/2 \varepsilon  s_{1} a
\end{math}\par

\begin{math}
\dot s_{2}=\varepsilon ^{2} \big(1/130 s_{4} s_{3} s_{2} i+1/26 s_{4} s_
{3} s_{2}+92/28275 s_{2}^{2} s_{1} i-4/1885 s_{2}^{2} s_{1}+1/8 s_{2} a
^{2} i\big)+1/2 \varepsilon  s_{2} a
\end{math}\par

\begin{math}
\dot s_{3}=\varepsilon ^{2} \big(-223/204 s_{4} s_{3}^{2} i-167/68 s_{4}
 s_{3}^{2}-1/2 s_{3} s_{2} s_{1} i-s_{3} s_{2} s_{1}-1/16 s_{3} 
\epsilon ^{2} i-1/4 s_{1}^{2} a-1/16 s_{1}^{2} \epsilon \big)+
\varepsilon  \big(1/2 s_{3} \epsilon +1/2 s_{1}^{2} i\big)
\end{math}\par

\begin{math}
\dot s_{4}=\varepsilon ^{2} \big(223/204 s_{4}^{2} s_{3} i-167/68 s_{4}
^{2} s_{3}+1/2 s_{4} s_{2} s_{1} i-s_{4} s_{2} s_{1}+1/16 s_{4} 
\epsilon ^{2} i-1/4 s_{2}^{2} a-1/16 s_{2}^{2} \epsilon \big)+
\varepsilon  \big(1/2 s_{4} \epsilon -1/2 s_{2}^{2} i\big)
\end{math}






\subsection{Fudge an oscillatory mode}
With frequency two, with a system with one slow mode.
Couple them with something ad hoc.

\begin{math}
\dot u_{1}=\varepsilon  \big(u_{4} u_{1}+u_{2} u_{1}\big)-2 u_{3}-2 u_{2
}
\end{math}\par

\begin{math}
\dot u_{2}=-2 u_{3}-3 u_{2}+u_{1}
\end{math}\par

\begin{math}
\dot u_{3}=2 u_{3}+3 u_{2}+u_{1}
\end{math}\par

\begin{math}
\dot u_{4}=\varepsilon  \big(-u_{4}^{2}-u_{2} u_{1}\big)+u_{5}-u_{4}
\end{math}\par

\begin{math}
\dot u_{5}=\varepsilon  u_{5}^{2}-u_{5}+u_{4}
\end{math}

\begin{reduce}
if thecase=oscmeanflow then begin
ff_:=tp mat((
    -2*u2-2*u3+u4*u1+u1*u2,
    u1-3*u2-2*u3,
    u1+3*u2+2*u3,
    -u4+u5-u4^2-u1*u2,
    +u4-u5+u5^2
    ));
freqm_:=mat((2,-2,0));
ee_:=tp mat((1,1,-1-i,0,0),(1,1,-1+i,0,0)
  ,(0,0,0,1,1));
zz_:=tp mat((1,-i,-i,0,0),(1,+i,+i,0,0)
  ,(0,0,0,1,1));
end;
\end{reduce}

\paragraph{The centre manifold} 

\begin{math}
u_{1}=\varepsilon  \big(1/3 \cis\big(-4 t\big) s_{2}^{2} i+1/8 \cis\big(
-2 t\big) s_{3} s_{2} i-1/3 \cis\big(4 t\big) s_{1}^{2} i-1/8 \cis\big(2
 t\big) s_{3} s_{1} i\big)+\cis\big(-2 t\big) s_{2}+\cis\big(2 t\big) s_
{1}
\end{math}\par

\begin{math}
u_{2}=\varepsilon  \big(5/51 \cis\big(-4 t\big) s_{2}^{2} i-1/17 \cis
\big(-4 t\big) s_{2}^{2}-11/40 \cis\big(-2 t\big) s_{3} s_{2} i-1/5 \cis
\big(-2 t\big) s_{3} s_{2}-5/51 \cis\big(4 t\big) s_{1}^{2} i-1/17 \cis
\big(4 t\big) s_{1}^{2}+11/40 \cis\big(2 t\big) s_{3} s_{1} i-1/5 \cis
\big(2 t\big) s_{3} s_{1}-2 s_{2} s_{1}\big)+\cis\big(-2 t\big) s_{2}+
\cis\big(2 t\big) s_{1}
\end{math}\par

\begin{math}
u_{3}=\varepsilon  \big(-5/51 \cis\big(-4 t\big) s_{2}^{2} i-11/102 \cis
\big(-4 t\big) s_{2}^{2}+11/40 \cis\big(-2 t\big) s_{3} s_{2} i+13/40 
\cis\big(-2 t\big) s_{3} s_{2}+5/51 \cis\big(4 t\big) s_{1}^{2} i-11/102
 \cis\big(4 t\big) s_{1}^{2}-11/40 \cis\big(2 t\big) s_{3} s_{1} i+13/40
 \cis\big(2 t\big) s_{3} s_{1}+3 s_{2} s_{1}\big)+\cis\big(-2 t\big) s_{
2} i-\cis\big(-2 t\big) s_{2}-\cis\big(2 t\big) s_{1} i-\cis\big(2 t
\big) s_{1}
\end{math}\par

\begin{math}
u_{4}=\varepsilon  \big(-9/40 \cis\big(-4 t\big) s_{2}^{2} i-1/20 \cis
\big(-4 t\big) s_{2}^{2}+9/40 \cis\big(4 t\big) s_{1}^{2} i-1/20 \cis
\big(4 t\big) s_{1}^{2}-1/2 s_{3}^{2}-1/2 s_{2} s_{1}\big)+s_{3}
\end{math}\par

\begin{math}
u_{5}=\varepsilon  \big(-1/40 \cis\big(-4 t\big) s_{2}^{2} i+1/20 \cis
\big(-4 t\big) s_{2}^{2}+1/40 \cis\big(4 t\big) s_{1}^{2} i+1/20 \cis
\big(4 t\big) s_{1}^{2}+1/2 s_{3}^{2}+1/2 s_{2} s_{1}\big)+s_{3}
\end{math}\par
 
\paragraph{Centre manifold ODEs} 

\begin{math}
\dot s_{1}=\varepsilon ^{2} \big(-1/16 s_{3}^{2} s_{1} i-1/4 s_{3}^{2} s
_{1}-421/4080 s_{2} s_{1}^{2} i-887/680 s_{2} s_{1}^{2}\big)+1/2 
\varepsilon  s_{3} s_{1}
\end{math}\par

\begin{math}
\dot s_{2}=\varepsilon ^{2} \big(1/16 s_{3}^{2} s_{2} i-1/4 s_{3}^{2} s_
{2}+421/4080 s_{2}^{2} s_{1} i-887/680 s_{2}^{2} s_{1}\big)+1/2 
\varepsilon  s_{3} s_{2}
\end{math}\par

\begin{math}
\dot s_{3}=\varepsilon ^{2} \big(s_{3}^{3}+6/5 s_{3} s_{2} s_{1}\big)-
\varepsilon  s_{2} s_{1}
\end{math}



Used this system for a benchmark to compare several ways of handling matrices and vectors.
This analysis using \verb|e_| as basis for matrices and vectors takes about a second or two in the following five iterations.
\begin{verbatim}
lengthres := 10
Time: 20 ms
lengthres := 124
Time: 120 ms
lengthres := 289
Time: 420 ms
lengthres := 169
Time: 580 ms
lengthres := 1
Time: 420 ms
SUCCESS: converged to an expansion
\end{verbatim}






\subsection{Modulate Duffing oscillation}

Tests that this code generates complex amplitude model for purely oscillating dynamics.
Here model the frequency correction in the Duffing oscillator \(\ddot u+u-u^3=0\).  
Code for \(u_1=u\) and \(u_2=\dot u\).

\begin{math}
\dot u_{1}=u_{2}
\end{math}\par

\begin{math}
\dot u_{2}=\varepsilon  u_{1}^{3}-u_{1}
\end{math}

\begin{reduce}
if thecase=modulateduffing then begin
ff_:=tp mat((u2,-u1+u1^3-small*2*u2));
freqm_:=mat((1,-1));
ee_:=tp mat((1,i),(1,-i));
zz_:=tp mat((1,i),(1,-i));
%maxiter_:=2; %%%%%%%%%%%%% for testing
end;
\end{reduce}

Find the coordinate transform is 
\begin{math}
u_{1}=\varepsilon  \big(-1/8 \cis\big(-3 t\big) s_{2}^{3}+3/4 \cis\big(-
t\big) s_{2}^{2} s_{1}-1/8 \cis\big(3 t\big) s_{1}^{3}+3/4 \cis\big(t
\big) s_{2} s_{1}^{2}\big)+\cis\big(-t\big) s_{2}+\cis\big(t\big) s_{1}
\end{math}
where the amplitudes evolve according to
\begin{math}
\dot s_{1}=-51/16 \varepsilon ^{2} s_{2}^{2} s_{1}^{3} i-3/2 
\varepsilon  s_{2} s_{1}^{2} i
\end{math}
and its complex conjugate.  This correctly predicts the frequency shift in the Duffing oscillator.






\subsection{Modulate another oscillation}

Retest that this code generates complex amplitude model for purely oscillating dynamics.
Here model the frequency correction in the oscillator \(\ddot u+u+\dot u^3=0\).  
Code for \(u_1=u\) and \(u_2=\dot u\).

\begin{math}
\dot u_{1}=u_{2}
\end{math}\par

\begin{math}
\dot u_{2}=-\varepsilon  u_{2}^{3}-u_{1}
\end{math}

\begin{reduce}
if thecase=modulateoscillator then begin
ff_:=tp mat((u2,-u1-u2^3));
freqm_:=mat((1,-1));
ee_:=tp mat((1,i),(1,-i));
zz_:=tp mat((1,i),(1,-i));
end;
\end{reduce}

The coordinate transform \begin{math}
u_{1}=\cis\big(-t\big) s_{2}+\cis\big(t
\big) s_{1}
+\varepsilon  \big(1/8 \cis\big(-3 t\big) s_{2}^{3} i+3/4 \cis\big(
-t\big) s_{2}^{2} s_{1} i-1/8 \cis\big(3 t\big) s_{1}^{3} i-3/4 \cis
\big(t\big) s_{2} s_{1}^{2} i\big)
\end{math}
looks fine; although note that here higher orders do differ to other work due to the orthogonality I build in.
The evolution seems appropriate:
\begin{math}
\dot s_{1}=-3/2 \varepsilon  s_{2} s_{1}^{2}
-27/16 \varepsilon ^{2} s_{2}^{2} s_{1}^{3} i
\end{math}





\subsection{An example from Iulian Stoleriu}
Consider the case \cite{Stoleriu2012} calls \((3\pi/4,k^2/2)\).
Use Taylor expansions for trigonometric functions in the \ode{}s.
Eigenvalues are $\pm 1$~and~$\pm i$, so we find the centre manifold among stable and unstable modes.
Sometimes we can have a parameter (here~\(\sigma\)) in the linear operator, but may need to specify its real and imaginary parts.

\begin{reduce}
if thecase=StoleriuOne then begin
let {repart(sigma)=>sigma,impart(sigma)=>0};
ff_:=tp mat((
    u2,
    sigma*u3+u1^2/2-small*u1^4/24,
    u4,
    u1/sigma+u3*u1+(u3+1/sigma)*(-small*u1^3/6)
    ));
freqm_:=mat((1,-1));
ee_:=tp mat((sigma,i*sigma,-1,-i),(sigma,-i*sigma,-1,+i));
zz_:=tp mat((+i,-1,-i*sigma,sigma),(-i,-1,+i*sigma,sigma));
end;
\end{reduce}

A centre manifold is \begin{math}
x=u_{1}=\varepsilon  \big(-1/5 \cis\big(-2 t\big) s_{2}^{2} \sigma ^{2}-1/
5 \cis\big(2 t\big) s_{1}^{2} \sigma ^{2}+2 s_{2} s_{1} \sigma ^{2}\big)
+\cis\big(-t\big) s_{2} \sigma +\cis\big(t\big) s_{1} \sigma 
\end{math} and \begin{math}
y=u_{3}=\varepsilon  \big(3/10 \cis\big(-2 t\big) s_{2}^{2} \sigma +3/10 
\cis\big(2 t\big) s_{1}^{2} \sigma -s_{2} s_{1} \sigma \big)-\cis\big(-t
\big) s_{2}-\cis\big(t\big) s_{1}
\end{math}.
On this centre manifold the oscillations have a frequency shift, but no amplitude evolution (to this order nor the next): \begin{math}
\dot s_{1}=-6/5 \varepsilon ^{2} s_{2} s_{1}^{2} i \sigma ^{2}
\end{math}.
Remember the system is unstable due to the unstable mode.




\subsection{An second example from Iulian Stoleriu}
Consider the case \cite{Stoleriu2012} calls \((\pi/2,0)\).
Use Taylor expansions for trigonometric functions in the \ode{}s.
Eigenvalues are~$\pm i$, multiplicity two, so we find modulation equations for coupled oscillators.

The system is
\begin{itemize}
\item \begin{math}
\dot u_{1}=u_{2}
\end{math}

\item \begin{math}
\dot u_{2}=-1/120 \varepsilon ^{2} u_{1}^{5}+1/6 \varepsilon  u_{1}^{3}+
u_{3} \sigma -u_{1}
\end{math}

\item \begin{math}
\dot u_{3}=u_{4}
\end{math}

\item \begin{math}
\dot u_{4}=-1/24 \varepsilon ^{2} u_{3} u_{1}^{4}+1/2 \varepsilon  u_{3}
 u_{1}^{2}-u_{3}
\end{math}
\end{itemize}


\begin{reduce}
if thecase=StoleriuTwo then begin
ff_:=tp mat((
    u2,
    -u1+u1^3/6-small*u1^5/120+sigma*u3,
    u4,
    -u3+u3*(u1^2/2-small*u1^4/24)
    ));
freqm_:=mat((1,-1,1,-1));
ee_:=tp mat((1,i,0,0),(1,-i,0,0),(0,0,1,i),(0,0,1,-i));
zz_:=tp mat((1,i,0,0),(1,-i,0,0),(0,0,1,i),(0,0,1,-i));
toosmall:=3;
end;
\end{reduce}

This used to take five iterates to construct the coordinate transform and modulation equations, but now less as the off-diagonal term is made small by the linear adjustment.  
The original variables are approximately
\begin{itemize}
\item \begin{math}
x=u_{1}=1/4 \cis\big(-t\big) s_{4} \sigma +\cis\big(-t\big) s_{2}+1/4 \cis
\big(t\big) s_{3} \sigma +\cis\big(t\big) s_{1}
\end{math}
\item  
\begin{math}
y=u_{3}=\cis\big(-t\big) s_{4}+\cis\big(t\big) s_{3}
\end{math}
\end{itemize}
The modulation equations are the following, and their complex conjugates: 
\begin{itemize}
\item \begin{math}
\dot s_{1}=\varepsilon  \big(-1/64 s_{4} s_{3}^{2} i \sigma ^{3}-3/32 s_
{4} s_{3} s_{1} i \sigma ^{2}-1/8 s_{4} s_{1}^{2} i \sigma -5/64 s_{3}^{
2} s_{2} i \sigma ^{2}-1/4 s_{3} s_{2} s_{1} i \sigma -1/4 s_{2} s_{1}^{
2} i\big)-1/2 s_{3} i \sigma 
\end{math};
\item \begin{math}
\dot s_{3}=\varepsilon  \big(-3/64 s_{4} s_{3}^{2} i \sigma ^{2}-1/4 s_{
4} s_{3} s_{1} i \sigma -1/4 s_{4} s_{1}^{2} i-1/8 s_{3}^{2} s_{2} i 
\sigma -1/2 s_{3} s_{2} s_{1} i\big)
\end{math}.
\end{itemize}
Since every term is multiplied by~$i$ one expects there to be just frequency shifts, but there are oscillator interaction terms as well.
These should be equivalent to the averaging method, but more easily extended to higher order (just change parameter \verb|toosmall|).







\subsection{Periodic chronic myelogenous leukemia}

\cite{Ion2012} explored Hopf bifurcations in a delay differential equation modelling leukaemia:%
\footnote{Their parameter~$\beta_0$ is absorbed in a time scaling.}
\begin{equation*}
\dot x=-\frac {x(t)}{1+x(t)^n}-\delta x(t)+\frac{kx(t-r)}{1+x(t-r)^n}
\end{equation*}
For simplicity we fix upon parameters $n=2$\,, $\delta\approx 1/8$\,, $k=3/2$ and time delay $r=64/3$\,; that is,
\begin{equation*}
\dot x=-\frac {x(t)}{1+x(t)^2}-(\rat18+\delta')x(t)+\frac{\rat32x(t-r)}{1+x(t-r)^2}
\end{equation*}
Near these parameters the equilibrium $x=X=\sqrt3$ perhaps undergoes a Hopf bifurcation.
`Perhaps' because instead of a precise time delay, we model~$x(t-r)$ via two intermediaries in the system, after defining $x(t)=X+u_1(t)$,
\begin{eqnarray*}
\dot u_1&=&-\frac {(X+u_1)}{1+(X+u_1)^2}-(\rat18+\delta')(X+u_1)+\frac{\rat32 (X+u_3)}{1+(X+u_3)^2}\,,
\\\dot u_2&=&\frac3{32}(u_1-u_2),
\\\dot u_3&=&\frac3{32}(u_2-u_3).
\end{eqnarray*}
This system does undergo a Hopf bifurcation as~$\delta'$ decreases through zero.
My code only analyses multinomial forms, so Taylor expand the rational function:
\begin{align*}
\frac{X+u}{1+(X+u)^2}
&=\frac{X}{1+X^2}
 +\frac{1-X^2}{(1+X^2)^2}u
 +\frac{X(X^2-3)}{(1+X^2)^3}u^2
 +\frac{-1+6X^2-X^4}{(1+X^2)^4}u^3
 +\cdots
\\&=\frac{\sqrt3}{4}
 -\frac{1}{8}u
 +0u^2
 +\frac{1}{32}u^3
 +\cdots \quad\text{at }X=\sqrt3\,.
\end{align*}


\begin{reduce}
if thecase=delayprolif then begin
ff_:=tp mat((
    -3/16*u3-u1^3/32-small*delta*(sqrt(3)+u1)+3/64*u3^3,
    3/32*u1-3/32*u2,
    3/32*u2-3/32*u3    
    ));
freqm_:=mat((3/32,-3/32));
ee_:=tp mat((1,1/2-i/2,-i/2),(1,1/2+i/2,+i/2));
zz_:=tp mat((1,-i,-1-i),(1,+i,-1+i));
toosmall:=2;
factor delta,s;
end;
\end{reduce}

\paragraph{The specified dynamical system}
\begin{math}
\end{math}\par
\begin{math}
\dot u_{1}=\varepsilon  \big(-\sqrt {3} \delta +3/64 u_{
3}^{3}-1/32 u_{1}^{3}-u_{1} \delta \big)-3/16 u_{3}
\end{math}\par
\begin{math}
\dot u_{2}=-3/32 u_{2}+3/32 u_{1}
\end{math}\par
\begin{math}
\dot u_{3}=-3/32 u_{3}+3/32 u_{2}
\end{math}\par

\paragraph{The centre manifold}
\begin{math}
\end{math}\par
\begin{math}
u_{1}=\cis\big(-3 t/32\big) s_{2}+\cis\big(3 t/32
\big) s_{1}
\end{math}\par
\begin{math}
u_{2}=1/2 \cis\big(-3 t/32\big) s_{2} i+1/2 \cis\big(-3 t
/32\big) s_{2}-1/2 \cis\big(3 t/32\big) s_{1} i+1/2 \cis
\big(3 t/32\big) s_{1}
\end{math}\par
\begin{math}
u_{3}=1/2 \cis\big(-3 t/32\big) s_{2} i-1/2 \cis\big(3 t
/32\big) s_{1} i
\end{math}\par

\paragraph{Centre manifold ODEs}
\begin{math}
\end{math}\par
\begin{math}
\dot s_{1}=\varepsilon  \big(3/256 s_{2} s_{1}^{2} i-21/512 s_{2} s_{1}
^{2}+1/5 s_{1} \delta  i-2/5 s_{1} \delta \big)
\end{math}\par
\begin{math}
\dot s_{2}=\varepsilon  \big(-3/256 s_{2}^{2} s_{1} i-21/512 s_{2}^{2} s
_{1}-1/5 s_{2} \delta  i-2/5 s_{2} \delta \big)
\end{math}\par

These indicate that $\vec s=\vec 0$ is stable for $\delta'\geq0$\,.  For parameter $\delta'<0$ there is a stable limit cycle of amplitude $|s_j|=16\sqrt{\frac{-2\delta'}{105}}$.
 


\subsubsection{Delayed version}

Return to the original system linearised about \(x=\sqrt3\), the following finds the spectrum and identifies a Hopf bifurcation of frequency~\(3/16\).
\begin{verbatim}
% linearised about x=sqrt3, freq is 3/16
delta=1/8, k=1+4*delta, r=8/3*pi
ce=@(z) -z+1/8-delta-k/8*exp(-r*z)
lams=fsolve(ce,randn(100,2)*[1;3*i]/2)
plot(real(lams),imag(lams),'o')
\end{verbatim}

The following works only by careful use of smallness.

\begin{reduce}
if thecase=delayedprolif then begin
r3:=sqrt(3);
delta:=1/8; k:=1+4*delta; r:=8/3*pi;
ff_:=tp mat((
    -r3*(1/4-3/8/r3*u1+1/8*u1^2-3/32/r3*u1^3*small)
    -u1*(1/4-3/8/r3*u1+1/8*u1^2*small)
%    -(r3+u1)*(1/4-3/8/r3*u1+1/8*u1^2-3/32/r3*small^2*u1^3)
    -delta*(r3+u1)
    +k*r3*(1/4-3/8/r3*u1(r)+1/8*u1(r)^2-3/32/r3*u1(r)^3*small)
    +k*u1(r)*(1/4-3/8/r3*u1(r)+1/8*u1(r)^2*small)
%    +k*(r3+u1(r))*(1/4-3/8/r3*u1(r)+1/8*u1(r)^2-small^2*3/32/r3*u1(r)^3)
    ));
freqm_:=mat((3/16,-3/16));
ee_:=tp mat((1),(1));
zz_:=tp mat((1),(1));
toosmall:=4;
factor s;
end;
\end{reduce}

\paragraph{The specified dynamical system}
\begin{math}
\end{math}\par

\begin{math}
\dot u_{1}=\varepsilon ^{2} \big(3/64 D_{t,\big(8 \pi \big)/3}\big(u_{1}
\big)^{3}-1/32 u_{1}^{3}\big)-3/16 D_{t,\big(8 \pi \big)/3}\big(u_{1}
\big)
\end{math}

\paragraph{The centre manifold}
\begin{math}
\end{math}\par

\begin{math}
u_{1}=s_{2}^{3} \varepsilon ^{2} \big(-1/24 \cis\big(\big(-9 t\big)/16
\big) i+1/16 \cis\big(\big(-9 t\big)/16\big)\big)+s_{2} \cis\big(\big(-3
 t\big)/16\big)+s_{1}^{3} \varepsilon ^{2} \big(1/24 \cis\big(\big(9 t
\big)/16\big) i+1/16 \cis\big(\big(9 t\big)/16\big)\big)+s_{1} \cis\big(
\big(3 t\big)/16\big)
\end{math}\par


\paragraph{Centre manifold ODEs}
\begin{math}
\end{math}\par

\begin{math}
\dot s_{1}=s_{2} s_{1}^{2} \varepsilon ^{2} \big(3/16 i \pi -9/16 i-9/32
 \pi -3/8\big)/\big(\pi ^{2}+4\big)
\end{math}\par

\begin{math}
\dot s_{2}=s_{2}^{2} s_{1} \varepsilon ^{2} \big(-3/16 i \pi +9/16 i-9/
32 \pi -3/8\big)/\big(\pi ^{2}+4\big)
\end{math}


\subsection{Nonlinear normal modes}

\cite{Renson2012} explored finite element construction of the nonlinear normal modes of a pair of coupled oscillators. 
Defining two new variables one of their example systems is
\begin{eqnarray*}
&&\dot x_1=x_3\,,
\\&&\dot x_2=x_4\,,
\\&&\dot x_3=-2x_1+x_2-\rat12x_1^3+\rat3{10}(-x_3+x_4)\,,
\\&&\dot x_4=x_1-2x_2+\rat3{10}(x_3-2x_4)\,.
\end{eqnarray*}
In the following code, force the linear damping to be effectively small (which then makes it small squared); consequently scale the smallness of the cubic nonlinearity.

\begin{reduce}
if thecase=normalmodes then begin
r3:=sqrt(3);
ff_:=tp mat((
    u3,
    u4,
    -2*u1+u2-small*u1^3/2+small*3/10*(-u3+u4),
    u1-2*u2+small*3/10*(u3-2*u4)
    ));
freqm_:=mat((1,-1,r3,-r3));
ee_:=tp mat((1,1,+i,+i),(1,1,-i,-i)
          ,(1,-1,i*r3,-i*r3),(1,-1,-i*r3,i*r3));
zz_:=tp mat((1,1,+i,+i),(1,1,-i,-i)
          ,(-i*r3,+i*r3,1,-1),(+i*r3,-i*r3,1,-1));
toosmall:=3;
end;
\end{reduce}

The square root frequencies do not cause any trouble (although may need to reformat the LaTeX of the cis operator).
In the model, observe that $s_1=s_2=0$ is invariant, as is $s_3=s_4=0$.  
These are the nonlinear normal modes.

\paragraph{The centre manifold}
\begin{math}
\end{math}\par

\begin{math}
u_{1}=\cis\big(-\sqrt {3} t\big) s_{4}+\cis\big(-t\big) s_{2}+\cis\big(
\sqrt {3} t\big) s_{3}+\cis\big(t\big) s_{1}
\end{math}\par

\begin{math}
u_{2}=-\cis\big(-\sqrt {3} t\big) s_{4}+\cis\big(-t\big) s_{2}-\cis\big(
\sqrt {3} t\big) s_{3}+\cis\big(t\big) s_{1}
\end{math}\par

\begin{math}
u_{3}=-\sqrt {3} \cis\big(-\sqrt {3} t\big) s_{4} i-\cis\big(-t\big) s_{
2} i+\sqrt {3} \cis\big(\sqrt {3} t\big) s_{3} i+\cis\big(t\big) s_{1} i
\end{math}\par

\begin{math}
u_{4}=\sqrt {3} \cis\big(-\sqrt {3} t\big) s_{4} i-\cis\big(-t\big) s_{2
} i-\sqrt {3} \cis\big(\sqrt {3} t\big) s_{3} i+\cis\big(t\big) s_{1} i
\end{math}\par

\paragraph{Centre manifold ODEs}
\begin{math}
\end{math}\par

\begin{math}
\dot s_{1}=\varepsilon  \big(3/4 s_{4} s_{3} s_{1} i+3/8 s_{2} s_{1}^{2}
 i-3/40 s_{1}\big)
\end{math}\par

\begin{math}
\dot s_{2}=\varepsilon  \big(-3/4 s_{4} s_{3} s_{2} i-3/8 s_{2}^{2} s_{1
} i-3/40 s_{2}\big)
\end{math}\par

\begin{math}
\dot s_{3}=\varepsilon  \big(1/8 \sqrt {3} s_{4} s_{3}^{2} i+1/4 \sqrt {
3} s_{3} s_{2} s_{1} i-3/8 s_{3}\big)
\end{math}\par

\begin{math}
\dot s_{4}=\varepsilon  \big(-1/8 \sqrt {3} s_{4}^{2} s_{3} i-1/4 
\sqrt {3} s_{4} s_{2} s_{1} i-3/8 s_{4}\big)
\end{math}\par







\subsection{Periodically forced van der Pol oscillator}

Hinvi et al. (2013) used renormalisation group to explore periodically forced van der Pol oscillator
\begin{equation*}
\ddot x+x-\epsilon(1-ax^2-b\dot x^2)\dot x=\epsilon c\sin \Omega t\,.
\end{equation*}
Introducing \(u_1=x\), rewrite as the system
\begin{eqnarray*}
&&\dot u_1=u_2\,,
\\&&\dot u_2=-u_1+\epsilon(1-au_1^2-bu_2^2)u_2+\epsilon cu_3\,,
\\&&\dot u_3=\Omega u_4\,,
\\&&\dot u_4=-\Omega u_3\,.
\end{eqnarray*}
This system has eigenvalues~\(\pm i\) and~\(\pm i\Omega\) so we seek the modulation equations of the oscillations.

Only the directly resonant case appears to be interesting, so set \(\Omega=1\), and then perturb it in the equations.

\begin{reduce}
if thecase=forcedvdp then begin
om:=1;
ff_:=tp mat((
    +u2,
    -u1+small*(1-a*u1^2-b*u2^2)*u2+small*c*u3,
    +om*u4*(1+small*omega),
    -om*u3*(1+small*omega)
    ));
freqm_:=mat((1,-1,om,-om));
ee_:=tp mat((1,+i,0,0),(1,-i,0,0)
          ,(0,0,1,+i),(0,0,1,-i));
zz_:=tp mat((1,+i,0,0),(1,-i,0,0)
          ,(0,0,1,+i),(0,0,1,-i));
toosmall:=4;
end;
\end{reduce}




\subsection{Slow manifold of Lorenz 1986 model}
In this case we actually construct the slow sub-centre manifold, analogous to quasi-geostrophy, in order to disentangle the slow dynamics from fast oscillations, analogous to gravity waves.
The algorithm still works.
The normals to the isochrons determine `balancing' onto the slow manifold.

\begin{reduce}
if thecase=lorenz86slow then begin
factor b;
ff_:=tp mat((-u2*u3+b*u2*u5
    ,u1*u3-b*u1*u5
    ,-u1*u2
    ,-u5
    ,+u4+b*u1*u2));
freqm_:=mat((0,0,0));
ee_:=zz_:=tp mat((1,0,0,0,0),(0,1,0,0,0),(0,0,1,0,0));
toosmall:=4;
end;
\end{reduce}

\paragraph{The centre manifold}
These give the location of the centre manifold in
terms of parameters~\(s\sb j\).
\(
\)\par

\(u_{1}=s_{1}
\)\par

\(u_{2}=s_{2}
\)\par

\(u_{3}=s_{3}
\)\par

\(u_{4}=-b \eps s_{2} s_{1}
\)\par

\(u_{5}=b \eps^{2} \big(-s_{3} s_{2}^{2}+s_{3} s_{1}^{2}\big)
\)\par

\(
\)
\paragraph{Centre manifold ODEs}
The system evolves on the centre manifold such
that the parameters evolve according to these ODEs.
\(
\)\par

\(\dot s_{1}=b^{2} \eps^{3} \big(-s_{3} s_{2}^{3}+s_{3} s_{2} s_{1}^{2}
\big)-\eps s_{3} s_{2}
\)\par

\(\dot s_{2}=b^{2} \eps^{3} \big(s_{3} s_{2}^{2} s_{1}-s_{3} s_{1}^{3}
\big)+\eps s_{3} s_{1}
\)\par

\(\dot s_{3}=-\eps s_{2} s_{1}
\)\par

\(
\)
\paragraph{Normals to isochrons at the slow manifold}
The normal vector \(\vec z\sb j:=(z\sb{j1},\ldots,z\sb{jn}\
)\)
\(
\)\par

\(z_{11}=b^{2} \eps^{2} s_{2}^{2}+1
\)\par

\(z_{12}=b^{2} \eps^{2} s_{2} s_{1}
\)\par

\(z_{13}=0
\)\par

\(z_{14}=b^{3} \eps^{3} \big(s_{2}^{3}-s_{2} s_{1}^{2}\big)+b \eps^{3} 
\big(-s_{2}^{3}+s_{2} s_{1}^{2}\big)+b \eps s_{2}
\)\par

\(z_{15}=0
\)\par

\(z_{21}=-b^{2} \eps^{2} s_{2} s_{1}
\)\par

\(z_{22}=-b^{2} \eps^{2} s_{1}^{2}+1
\)\par

\(z_{23}=0
\)\par

\(z_{24}=b^{3} \eps^{3} \big(-s_{2}^{2} s_{1}+s_{1}^{3}\big)+b \eps^{3} 
\big(s_{2}^{2} s_{1}-s_{1}^{3}\big)-b \eps s_{1}
\)\par

\(z_{25}=0
\)\par

\(z_{31}=0
\)\par

\(z_{32}=0
\)\par

\(z_{33}=1
\)\par

\(z_{34}=-4 b \eps^{3} s_{3} s_{2} s_{1}
\)\par

\(z_{35}=b \eps^{2} \big(-s_{2}^{2}+s_{1}^{2}\big)
\)\par

 


\subsection{Check the dimensionality of specified system}
Extract dimension information from the specification of the dynamical system: seek $m$D~centre manifold of an $n$D~system.

\begin{reduce}
if thecase=myweb then begin
  out "cmsyso.txt"$
  ODE_function:=ff_; 
  centre_frequencies:=freqm_;
  centre_eigenvectors:=ee_; 
  adjoint_eigenvectors:=zz_;
end;
\end{reduce}

\begin{reduce}
write "total no. of modes  ",
n:=part(length(ee_),1);
write "no. of centre modes ",
m:=part(length(ee_),2);
if {length(freqm_),length(zz_),length(ee_),length(ff_)}
  ={{1,m},{n,m},{n,m},{n,1}} 
  then write "Input dimensions are OK" 
  else <<write "INCONSISTENT INPUT DIMENSIONS, I QUIT"; 
      quit>>;
\end{reduce}

For the moment limit to a maximum of nine components.
\begin{reduce}
if n>9 then <<write "SORRY, TOO MANY ODEs FOR ME, I QUIT"; 
    quit>>;
\end{reduce}

Need an \(m\times m\)~identity matrix for normalisation of the isochron projection.
\begin{reduce}
eyem_:=for j:=1:m sum e_(j,j)$
\end{reduce}


\section{Dissect the linear part}

Define complex exponential $\verb|cis|(u)=e^{iu}$.
Do not (yet) invoke the simplification of $\verb|cis|(0)$ as I want it to label modes of no oscillation, zero frequency.

\begin{reduce}
operator cis;
let { df(cis(~u),t) => i*df(u,t)*cis(u)
    , cis(~u)*cis(~v) => cis(u+v)
    , cis(~u)^~p => cis(p*u)
    };
\end{reduce}

Need function~\verb|conj_| to do parsimonious complex conjugation.
\begin{reduce}
operator cis__;
procedure conj_(a)$    
    ((a where {i=>i__, cis(~b)=>cis__(b) })
        where {i__=>-i,cis__(~b)=>cis(-b)})$
\end{reduce}

Make another array of frequencies for simplicity.

\begin{reduce}
array freq(m);
for j:=1:m do freq(j):=freqm_(1,j);
\end{reduce}



\subsection{Normalise the adjoint eigenvectors}
When we include delay differential equations, then we need to account for the history of the eigenvector as well.
Hence multiply each eigenvector by its oscillating factor,~\(e^{i\omega t}\), and then take the mean.
This multiplication by its oscillating factor should not make any difference for non-delay equations by the natural orthogonality of left and right eigenvectors of different eigenvalues.
Reduce implements \verb|conj| via \verb|repart| and \verb|impart|, so let \verb|repart| do the conjugation of the cis factors.

Note: the `left eigenvectors' have to be the eigenvectors of the complex conjugate transpose, and for the complex conjugate frequency.
This seems best: for example, when the linear operator is \(\begin{bmatrix} 0&1\\-1&0 \end{bmatrix}\) then the adjoint and the right eigenvectors are the same.

For un/stable manifolds we have to cope with complex frequencies.  
Seems to need \verb|zz_| to have complex conjugated frequency so store in~\verb|ccis_|---which is the same as \verb|dcis_| for real frequencies.

\begin{reduce}
matrix aa_(m,m),dcis_(m,m),ccis_(m,m);
for j:=1:m do dcis_(j,j):=cis(freq(j)*t);
for j:=1:m do ccis_(j,j):=cis(conj_(freq(j))*t);
aa_:=(tp map(conj_(~b),ee_*dcis_)*zz_*ccis_ )$
write "Normalising the left-eigenvectors:";
aa_:=(aa_ where {cis(0)=>1, cis(~a)=>0 when a neq 0})$
if det(aa_)=0 then << write
    "ORTHOGONALITY ERROR IN EIGENVECTORS; I QUIT"; quit>>;
zz_:=zz_*aa_^(-1);
\end{reduce}


\subsection{Operator to represent delays}

Introduce an operator to represent delay factors more conveniently for analysis.

\begin{reduce}
operator d_; linear d_;
let { d_(~a^~p,t,~dt)=>d_(a,t,dt)^p
    , d_(~a*~b,t,~dt)=>d_(a,t,dt)*d_(b,t,dt)
    , d_(cis(~a),t,~dt)=>cis(a)
        *sub(t=-dt,cos(a)+i*sin(a))
    , df(d_(~a,t,~dt),~b)=>d_(df(a,b),t,dt)
    , d_(~a,t,0)=>a
    , d_(d_(~a,t,~dta),t,~dtb)=>d_(a,t,dta+dtb)
    };
\end{reduce}

Now rewrite the (delay) factors in terms of this operator.
Need to say that the symbol~\verb|u| depends upon time; later we write things into~\verb|u| and this dependence would be forgotten.
For the moment limit to a maximum of nine \ode{}s.

\begin{reduce}
somerules:={}$
depend u1,t;somerules:=(u1(~dt)=d_(u1,t,dt)).somerules$
depend u2,t;somerules:=(u2(~dt)=d_(u2,t,dt)).somerules$
depend u3,t;somerules:=(u3(~dt)=d_(u3,t,dt)).somerules$
depend u4,t;somerules:=(u4(~dt)=d_(u4,t,dt)).somerules$
depend u5,t;somerules:=(u5(~dt)=d_(u5,t,dt)).somerules$
depend u6,t;somerules:=(u6(~dt)=d_(u6,t,dt)).somerules$
depend u7,t;somerules:=(u7(~dt)=d_(u7,t,dt)).somerules$
depend u8,t;somerules:=(u8(~dt)=d_(u8,t,dt)).somerules$
depend u9,t;somerules:=(u9(~dt)=d_(u9,t,dt)).somerules$
ff_:=(ff_ where somerules)$
\end{reduce}

\subsection{Linearise at the origin}
Assume the equilibrium is at the origin.
Find the linear operator at the equilibrium.
Include \verb|small=0| as we notionally adjoin it in the list of variables.
Do not need to here make small any non-zero forcing at the equilibrium as it gets multiplied by \verb|small| later??
For some reason using \verb|mkid(u,k)=>0| does not resolve the mkid, but \verb|mkid(u,k)=0| does; however, not clear if it is a problem??

\begin{reduce}
matrix ll_(n,n);
uzero:=(for k:=1:n collect (mkid(u,k)=0))$
equilibrium:=(small=0).uzero$
for j:=1:n do for k:=1:n do begin 
  ll_(j,k):=df(ff_(j,1),mkid(u,k));
  ll_(j,k):=sub(equilibrium,ll_(j,k));
end;
write "Find the linear operator is";
ll_:=ll_;
\end{reduce}

We need a vector of unknowns for a little while.
Should call this plain u??

\begin{reduce}
matrix uvec(n,1);
for j:=1:n do uvec(j,1):=mkid(u,j);
\end{reduce}


\subsection{Eigen-check}

%Force slow variable dynamics to be small.  
%Should not be necessary for Jordan form linear operator, but I do not think we can risk it in general.
Variable \verb|aa_| appears here as the diagonal matrix of frequencies.
Check that the frequencies and eigenvectors are specified correctly.

\begin{reduce}
write "Check centre subspace linearisation ";
for j:=1:m do for k:=1:m do aa_(j,k):=0;
for j:=1:m do aa_(j,j):=i*freq(j);
reslin:=(ll_*(ee_*dcis_)-(ee_*dcis_)*aa_
    where cis(~a)*d_(1,t,~dt)=>sub(t=-dt,cos(a)+i*sin(a))*cis(a) )$ 
ok_:=1$
for j:=1:n do for k:=1:m do 
    ok_:=if reslin(j,k)=0 then ok_ else 0$
if ok_ then write "Linearisation is OK";
\end{reduce}

Try to find a correction of the linear operator that is `close'.
Multiply by the adjoint eigenvectors and then average over time: operator \(\cL_{\text{new}}:=\cL-\cL_{\text{adj}}\) should now have zero residual.
Lastly, correspondingly adjust the \ode{}s, since \verb|lladj| does not involve delays we do not need delay operator transforms in the product.

\begin{reduce}
if not ok_ then begin
write "WARNING: I NEED TO ADJUST LINEAR OPERATOR";
lladj:=reslin*tp map(conj_(~b),zz_*ccis_);
write
lladj:=(lladj where {cis(0)=>1, cis(~a)=>0 when a neq 0});
write
ll_:=ll_-lladj;
reslin:=(ll_*(ee_*dcis_)-(ee_*dcis_)*aa_
    where cis(~a)*d_(1,t,~dt)=>sub(t=-dt,cos(a)+i*sin(a))*cis(a) ); 
for j:=1:n do for k:=1:m do 
    if reslin(j,k) neq 0 then << write
    "OOPS, INCONSISTENT EIGENVALUES, EIGENVECTORS AND OPERATOR.
    EMAIL ME; I QUIT"; quit >>; 
end;
\end{reduce}




\subsection{Ameliorate the nonlinearity}

Anything not in the linear operator gets multiplied by \verb|small| to be treated as small in the analysis.
The feature of the second alternative is that when a user invokes~\verb|small| then the power of smallness is not then changed; however, causes issues in the relative scaling of some terms, so restore to the original version.

This might need reconsidering ??  but the if always chooses the first simple alternative.
\begin{reduce}
somerules:=for j:=1:n collect 
  (d_(1,t,~dt)*mkid(u,j)=d_(mkid(u,j),t,dt))$
ff_:=(if 1 then small*ff_
    else ff_-(1-small)*sub(small=0,ff_)) +(1-small)
    *(ll_*uvec where somerules)$
\end{reduce}

Any constant term in the equations~\verb|ff_| has to be multiplied by~\verb|cis(0)|.

\begin{reduce}
ff_:=ff_+(cis(0)-1)*(ff_ where uzero)$
\end{reduce}

From the matrix versions of the equations, create algebraic form using the matrix basis.

\begin{reduce}
rhsfn_:=for i:=1:n sum e_(i,1)*ff_(i,1)$
\end{reduce}

Also, create the algebraic form of the jacobian transpose using the matrix basis: take the conjugate later when used.

\begin{reduce}
rhsjact_:=for i:=1:n sum for j:=1:n sum 
    e_(j,i)*df(ff_(i,1),mkid(u,j))$
\end{reduce}




\subsection{Store centre manifold frequencies}
Extract all the frequencies in the centre manifold, and the set of all the corresponding modes in the centre manifold variables.
The slow modes are accounted for as having zero frequency.
Remember the frequency set is not in the `correct' order.
Array \verb|modes| stores the set of indices of all the modes of a given frequency.

\begin{reduce}
array freqs(m),modes(m); 
nfreq:=0$ freqset:={}$ 
for j:=1:m do if not(freq(j) member freqset) then begin
  nfreq:=nfreq+1; 
  freqs(nfreq):=freq(j);
  freqset:=freq(j).freqset;
  modes(nfreq):=for k:=j:m join 
    if freq(j)=freq(k) then {k} else {};
end;
\end{reduce}

Set a flag for the case of a slow manifold when all frequencies are zero, as then we compute the isochron projection.
The next challenge is to get this isochron code working for the case of non-slow centre manifolds.
\begin{reduce}
itisSlowMan_:=if freqset={0} then 1 else 0$
if trace_ then write itisSlowMan_:=itisSlowMan_;
\end{reduce}

Put in the non-singular general case as the zero entry of the arrays.

\begin{reduce}
freqs(0):=genfreq$
modes(0):={}$
\end{reduce}
 
\subsection{Precompute matrices for updates}

Precompute matrices to solve for updates for each of the critical frequencies, and the general case $\verb|k|=0$.
The matrix 
\begin{equation*}
\verb|llzz| = \begin{bmatrix} -\cL+\partial_t & \cE_0 
\\ \adj\cZ_0 & 0 \end{bmatrix}
\end{equation*}
and then put its inverse in place.
Subsequently, extract the blocks for the generalised inverses and solvability condition into \verb|linvs| and~\verb|ginvs|.

\begin{reduce}
matrix llzz(n+m,n+m);
array linvs(nfreq),ginvs(nfreq);
array l1invs(nfreq),g1invs(nfreq),l2invs(nfreq),g2invs(nfreq);
operator sp_; linear sp_;
for k:=0:nfreq do begin
\end{reduce}

Code the operator \(\cL\hat v\) where the delay is to only act on the oscillation part.

\begin{reduce}
  for ii:=1:n do for jj:=1:n do llzz(ii,jj):=(
      -sub(small=0,ll_(ii,jj))
      where d_(1,t,~dt)=>cos(freqs(k)*dt)-i*sin(freqs(k)*dt));
\end{reduce}

Code the operator \(\D t{\hat v}\) where it only acts on the oscillation part.

\begin{reduce}
  for j:=1:n do llzz(j,j):=i*freqs(k)+llzz(j,j);
\end{reduce}

Now code the part leading to the solvability condition which arises from allowing the (oscillation) amplitude to evolve.
Use operator \verb|sp_| to extract the delay parts that subtly affect the updates of the evolution.

\begin{reduce}
  for j:=1:length(modes(k)) do 
    for ii:=1:n do llzz(ii,n+j):=ee_(ii,part(modes(k),j))
     +(for jj:=1:n sum 
       sp_(ll_(ii,jj)*ee_(jj,part(modes(k),j)),d_)
       where { sp_(1,d_)=>0
             , sp_(d_(1,t,~dt),d_)=>dt*(
               cos(freqs(k)*dt)-i*sin(freqs(k)*dt))
             });
\end{reduce}

Force the updates to be orthogonal to the left-eigenvectors in the complex conjugate transpose adjoint.,

\begin{reduce}
  for i:=1:length(modes(k)) do 
    for j:=1:n do llzz(n+i,j):=conj_(zz_(j,part(modes(k),i)));
\end{reduce}

Set the bottom-right corner of the matrix to zero.

\begin{reduce}
  for i:=1:length(modes(k)) do 
    for j:=1:m do llzz(n+i,n+j):=0;
\end{reduce}

Add some trivial rows and columns to make the matrix up to the same size for all frequencies.

\begin{reduce}
  for i:=length(modes(k))+1:m do begin 
    for j:=1:n+i-1 do llzz(n+i,j):=llzz(j,n+i):=0;
    llzz(n+i,n+i):=1;
  end;
\end{reduce}

Invert the matrix and unpack into arrays ready for use by the inversion operators.

\begin{reduce}
if trace_ then write llzz:=llzz; 
  llzz:=llzz^(-1);
if trace_ then write llzz:=llzz;
  linvs(k):=for i:=1:n sum for j:=1:n sum e_(i,j)*llzz(i,j);
  ginvs(k):=for i:=1:length(modes(k)) sum 
    for j:=1:n sum e_(part(modes(k),i),j)*llzz(i+n,j);
\end{reduce}

Unpack the conjugate transpose for inverse operators used for the isochrons.  
A difference here is that the orthogonality condition is non-trivial (in the slow manifold we assumed amplitudes were exactly orthogonal to the left-eigenvectors), so we need to remember more parts of the inverse of the matrix.
Will it need to be more subtle for the un/stable manifolds case??

\begin{reduce}
%  realgenfreq:={repart(genfreq)=>genfreq, impart(genfreq)=>0}$
  l1invs(k):=for i:=1:n sum for j:=1:n sum 
      e_(i,j)*conj_(llzz(j,i));
  l2invs(k):=for i:=1:n sum for j:=1:length(modes(k)) sum 
      e_(i,part(modes(k),j))*conj_(llzz(j+n,i));
  g1invs(k):=for i:=1:length(modes(k)) sum for j:=1:n sum 
      e_(part(modes(k),i),j)*conj_(llzz(j,i+n));
  g2invs(k):=
    for i:=1:length(modes(k)) sum for j:=1:length(modes(k)) sum 
      e_(part(modes(k),i),part(modes(k),j))*conj_(llzz(j+n,i+n));
end;
\end{reduce}


\subsection{Define operators that invoke these inverses}
Decompose residuals into parts, and operate on each.
First for the centre manifold.
But making \verb|e_| non-commutative means that it does not get factored out of these linear operators: must post-multiply by~\verb|e_| because the linear inverse is a premultiply.

\begin{reduce}
operator linv; linear linv;
let linv(e_(~j,~k)*cis(~a),cis)=>linvproc(a/t)*e_(j,k);
procedure linvproc(a);
  if a member freqset
  then << k:=0; 
    repeat k:=k+1 until a=freqs(k);
    linvs(k)*cis(a*t) >>
  else sub(genfreq=a,linvs(0))*cis(a*t)$
\end{reduce}

Second for the evolution on the centre manifold.

\begin{reduce}
operator ginv; linear ginv;
let ginv(e_(~j,~k)*cis(~a),cis)=>ginvproc(a/t)*e_(j,k);
procedure ginvproc(a); 
  if a member freqset
  then << k:=0; 
    repeat k:=k+1 until a=freqs(k);
    ginvs(k) >>
  else sub(genfreq=a,ginvs(0))$
\end{reduce}

Copy and adjust the above for the projection.
But first define the generic procedure.
Perhaps use conjugate\slash negative of the frequency when applying to the general case of oscillations---but it might already have been accounted for??
\begin{reduce}
procedure invproc(a,invs);
  if a member freqset
  then << k:=0; 
    repeat k:=k+1 until a=freqs(k);
    invs(k)*cis(a*t) >>
  else sub(genfreq=a,invs(0))*cis(a*t)$
\end{reduce}

Then define operators that we use to update the projection.
\begin{reduce}
operator l1inv; linear l1inv;
operator l2inv; linear l2inv;
operator g1inv; linear g1inv;
operator g2inv; linear g2inv;
let { l1inv(e_(~j,~k)*cis(~a),cis)=>invproc(a/t,l1invs)*e_(j,k)
    , l2inv(e_(~j,~k)*cis(~a),cis)=>invproc(a/t,l2invs)*e_(j,k)
    , g1inv(e_(~j,~k)*cis(~a),cis)=>invproc(a/t,g1invs)*e_(j,k)
    , g2inv(e_(~j,~k)*cis(~a),cis)=>invproc(a/t,g2invs)*e_(j,k)
    };
\end{reduce}








This section writes to various files so the output to \verb|cmsyso.txt| must be redone afterwards.

\section{Initialise LaTeX output}
% AJR, 21 aug 2012 -- Aug 2014

This section writes to various files so the output to \verb|cmsyso.txt| must be redone afterwards.

First define how various tokens get printed.

\begin{reduce}
load_package rlfi;
deflist('((!( !\!b!i!g!() (!) !\!b!i!g!)) (!P!I !\!p!i! )
         (!p!i !\!p!i! ) (!E !e) (!I !i) (e !e) (i !i)),'name)$
\end{reduce}

Override the procedure that prints annoying messages about multicharacter symbols.  
It ends the output of one expression.  
This is just a copy from \verb|rlfi.red| with the appropriate if-statement deleted.
While interfering, hardcode that the mathematics is in inline mode.

\begin{reduce}
%write "Ignore immediately following messages";
symbolic procedure prinlaend;
<<terpri();
  prin2t "\)\par";
  if !*verbatim then
      <<prin2t "\begin{verbatim}";
        prin2t "REDUCE Input:">>;
  ncharspr!*:=0;
  if ofl!* then linelength(car linel!*)
    else laline!*:=cdr linel!*;
  nochar!*:=append(nochar!*,nochar1!*);
  nochar1!*:=nil >>$
  %
\end{reduce}
Similarly, hardcode at the beginning of expression output that the mathematics is in inline mode.
\begin{reduce}
symbolic procedure prinlabegin;
% Initializes the output
<<if !*verbatim then
      <<terpri();
        prin2t "\end{verbatim}">>;
  linel!*:=linelength nil . laline!*;
  if ofl!* then linelength(laline!* + 2)
    else laline!*:=car linel!* - 2;
  prin2 "\(" >>$
\end{reduce}

Override the procedure that outputs the \LaTeX\ preamble 
upon the command \verb|on latex|.
Presumably modified from that in \verb|rlfi.red|.
Use it to write a decent header that we use for one master file.

\begin{reduce}
symbolic procedure latexon;
<<!*!*a2sfn:='texaeval;
  !*raise:=nil;
  prin2t "\documentclass[11pt,a5paper]{article}";
  prin2t "\usepackage[a5paper,margin=13mm]{geometry}";
  prin2t "\usepackage{parskip,time} \raggedright";
  prin2t "\def\cis\big(#1\big){\,e^{#1i}}";
  prin2t "\def\eps{\varepsilon}";
  prin2t "\title{Centre manifold of your dynamical system}";
  prin2t "\author{A. J. Roberts, University of Adelaide\\";
  prin2t "\texttt{http://www.maths.adelaide.edu.au/anthony.roberts}}";
  prin2t "\date{\now, \today}";
  prin2t "\begin{document}";
  prin2t "\maketitle";
  prin2t "Throughout and generally: the lowest order, most";
  prin2t "important, terms are near the end of each expression.";
  prin2t "
\(\)
\paragraph{The specified dynamical system}
\(
\)\par

\(\dot u_{1}=b \eps u\sb2 u\sb5-\eps u\sb2 u\sb3
\)\par

\(\dot u_{2}=-b \eps u\sb1 u\sb5+\eps u\sb1 u\sb3
\)\par

\(\dot u_{3}=-\eps u\sb1 u\sb2
\)\par

\(\dot u_{4}=-u\sb5
\)\par

\(\dot u_{5}=b \eps u\sb1 u\sb2+u\sb4
\)\par

\(\)
\paragraph{Centre subspace basis vectors}
\(
\)\par

\(\vec e_{1}=\left\{
\left\{
1 , 0 , 0 , 0 , 0
\right\} , \cis\big(0\big)
\right\}
\)\par

\(\vec e_{2}=\left\{
\left\{
0 , 1 , 0 , 0 , 0
\right\} , \cis\big(0\big)
\right\}
\)\par

\(\vec e_{3}=\left\{
\left\{
0 , 0 , 1 , 0 , 0
\right\} , \cis\big(0\big)
\right\}
\)\par

\(\vec z_{1}=\left\{
\left\{
1 , 0 , 0 , 0 , 0
\right\} , \cis\big(0\big)
\right\}
\)\par

\(\vec z_{2}=\left\{
\left\{
0 , 1 , 0 , 0 , 0
\right\} , \cis\big(0\big)
\right\}
\)\par

\(\vec z_{3}=\left\{
\left\{
0 , 0 , 1 , 0 , 0
\right\} , \cis\big(0\big)
\right\}
\)\par
";
  if !*verbatim then
      <<prin2t "\begin{verbatim}";
        prin2t "REDUCE Input:">>;
  put('tex,'rtypefn,'(lambda(x) 'tex)) >>$
\end{reduce}

The above definition for \verb|\cis| is not quite right for \verb|rlfi|, but I do not know how to fix it.

Set the default output to be inline mathematics.

\begin{reduce}
mathstyle math;
\end{reduce}

Define the Greek alphabet with \verb|small| as well.

\begin{reduce}
defid small,name="\eps";%varepsilon;
defid alpha,name=alpha;
defid beta,name=beta;
defid gamma,name=gamma;
defid delta,name=delta;
defid epsilon,name=epsilon;
defid varepsilon,name=varepsilon;
defid zeta,name=zeta;
defid eta,name=eta;
defid theta,name=theta;
defid vartheta,name=vartheta;
defid iota,name=iota;
defid kappa,name=kappa;
defid lambda,name=lambda;
defid mu,name=mu;
defid nu,name=nu;
defid xi,name=xi;
defid pi,name=pi;
defid varpi,name=varpi;
defid rho,name=rho;
defid varrho,name=varrho;
defid sigma,name=sigma;
defid varsigma,name=varsigma;
defid tau,name=tau;
defid upsilon,name=upsilon;
defid phi,name=phi;
defid varphi,name=varphi;
defid chi,name=chi;
defid psi,name=psi;
defid omega,name=omega;
defid Gamma,name=Gamma;
defid Delta,name=Delta;
defid Theta,name=Theta;
defid Lambda,name=Lambda;
defid Xi,name=Xi;
defid Pi,name=Pi;
defid Sigma,name=Sigma;
defid Upsilon,name=Upsilon;
defid Phi,name=Phi;
defid Psi,name=Psi;
defid Omega,name=Omega;
\end{reduce}


\begin{reduce}
defindex e_(down,down);
defid e_,name="e";
defindex d_(arg,down,down);
defid d_,name="D";
defindex u(down);
defid u1,name="u\sb1"; 
defid u2,name="u\sb2"; 
defid u3,name="u\sb3"; 
defid u4,name="u\sb4"; 
defid u5,name="u\sb5"; 
defid u6,name="u\sb6"; 
defid u7,name="u\sb7"; 
defid u8,name="u\sb8"; 
defid u9,name="u\sb9"; 
defindex s(down);
defid cis,name="\cis";
defindex cis(arg);
\end{reduce}


Can we write the system?  
Not in matrices apparently.
So define a dummy array \verb|tmp| that we use to get the correct symbol typeset.

\begin{reduce}
array tmp(n),tmps(m),tmpz(m);
defindex tmp(down);
defindex tmps(down);
defindex tmpz(down);
defid tmp,name="\dot u";
defid tmps,name="\vec e";
defid tmpz,name="\vec z";
rhs_:=rhsfn_$
for k:=1:m do tmps(k):={for j:=1:n collect ee_(j,k),cis(freq(k)*t)};
for k:=1:m do tmpz(k):={for j:=1:n collect zz_(j,k),cis(freq(k)*t)};
\end{reduce}

We have to be shifty here because \verb|rlfi| does not work inside a loop: so write the commands to a file, and then input the file.

\begin{reduce}
out "scratchfile.red";
write "write ""\)
\paragraph{The specified dynamical system}
\("";";
for j:=1:n do write "tmp(",j,"):=coeffn(rhs_,e_(",j,",1),1);";
write "write ""\)
\paragraph{Centre subspace basis vectors}
\("";";
for j:=1:m do write "tmps(",j,"):=tmps(",j,");";
for j:=1:m do write "tmpz(",j,"):=tmpz(",j,");";
write "end;";
shut "scratchfile.red";
\end{reduce}
Now print the dynamical system to the LaTeX sub-file.

\begin{reduce}
on latex$
out "centreManSys.tex"$
in "scratchfile.red"$
shut "centreManSys.tex"$
off latex$
\end{reduce}

Finish the input.
\begin{reduce}
end;
\end{reduce}

\begin{reduce}
in_tex "latexinit2.tex"$
\end{reduce}




\section{Linear approximation to the centre manifold}

But first, and if for the web, open the output file and write out the possibly adjusted nonlinear right-hand side function.
According to the manual, this will append to the earlier output to the file.

\begin{reduce}
if thecase=myweb then out "cmsyso.txt"$
\end{reduce}

\begin{reduce}
write "Analyse ODE/DDE system du/dt = ",ff_;
\end{reduce}

Parametrise the centre manifold in terms of these amplitudes.

\begin{reduce}
operator s; depend s,t;
let df(s(~j),t)=>coeffn(gg_,e_(j,1),1);
\end{reduce}


Invoke the following procedure to substitute whatever the current approximation is into (nonlinear) expressions.

\begin{reduce}
procedure manifold_;
    for j:=1:n collect mkid(u,j)=coeffn(uu_,e_(j,1),1)$
\end{reduce}


The linear approximation to the centre manifold must be the following corresponding to the frequencies down the diagonal (even if zero).
The amplitudes~$s_j$ are slowly evolving as they are either slow modes, or the complex amplitudes of oscillating modes.

\begin{reduce}
uu_:=for j:=1:m sum s(j)*cis(freq(j)*t)
  *(for k:=1:n sum e_(k,1)*ee_(k,j))$
gg_:=0$
\end{reduce}

For some temporary trace printing??
\begin{reduce}
procedure matify(a,m,n)$
  begin matrix z(m,n);
    for i:=1:m do for j:=1:n do z(i,j):=coeffn(a,e_(i,j),1);
    return (z where {cis(0)=>1,small=>s}); 
    end$
\end{reduce}

For the isochron may need to do something different with frequencies, but this should work as the inner product is complex conjugate transpose.
The \verb|pp_| matrix is proposed to place the projection residuals in the range of the isochron. 
\begin{reduce}
zs_:=for j:=1:m sum cis(freq(j)*t)
  *(for k:=1:n sum e_(k,j)*zz_(k,j))$
pp_:=0$
\end{reduce}




\section{Iteratively construct the centre manifold}

But first establish the Taylor series in any delay factors of slow amplitudes.
\begin{reduce}
let d_(s(~k),t,~dt)=>s(k)+(for n:=1:toosmall sum 
        (-dt)^n*df(s(k),t,n)/factorial(n));
\end{reduce}

Truncate expansions to specified order of error, and start the iteration.
\begin{reduce}
for j:=toosmall:toosmall do let small^j=>0;
write "Start iterative construction of centre manifold";
for iter:=1:maxiter_ do begin
if trace_ then write "
ITERATION = ",iter,"
-------------";
\end{reduce}

Compute residual vector (matrix) of the dynamical system \cite{Roberts96a}.
\begin{reduce}
resde_:=-df(uu_,t)+sub(manifold_(),rhsfn_);
if trace_ then write "resde_=",matify(resde_,n,1);
\end{reduce}

Get the local directions of the coordinate system on the curving manifold: store transpose as \(m\times n\)~matrix.
\begin{reduce}
est_:=tpe_(for j:=1:m sum df(uu_,s(j))*e_(1,j),e_);
est_:=conj_(est_);
if trace_ then write "est_=",matify(est_,m,n);
\end{reduce}

Compute residual matrix for the isochron projection \cite{Roberts89b, Roberts97b}. 
But for the moment, only do it if the \verb|freqset| is for slow manifolds.
\begin{reduce}
if itisSlowMan_ then begin
    jacadj_:=conj_(sub(manifold_(),rhsjact_));
if trace_ then write "jacadj_=",matify(jacadj_,n,n);
    resd_:=df(zs_,t)+jacadj_*zs_+zs_*pp_;
if trace_ then write "resd_=",matify(resd_,n,m);
\end{reduce}

Compute residual of the normalisation of the projection.
\begin{reduce}
    resz_:=est_*zs_-eyem_*cis(0);
if trace_ then write "resz_=",matify(resz_,m,m);
end else resd_:=resz_:=0; % for when not slow manifold
\end{reduce}

Write lengths of residuals as a trace print (remember that the expression~$0$ has length one).
\begin{reduce}
write lengthRes:=map(length(~a),{resde_,resd_,resz_});
\end{reduce}

Solve for updates---all the hard work is already encoded in the operators.
\begin{reduce}
uu_:=uu_+linv(resde_,cis);
gg_:=gg_+ginv(resde_,cis);
if trace_ then write "gg_=",matify(gg_,m,1);
if trace_ then write "uu_=",matify(uu_,n,1);
\end{reduce}

Now update the isochron projection, with normalisation.
\begin{reduce}
if itisSlowMan_ then begin
zs_:=zs_+l1inv(resd_,cis)-l2inv(resz_,cis);
pp_:=pp_-g1inv(resd_,cis)+youshouldnotseethis*g2inv(resz_,cis);
if trace_ then write "zs_=",matify(zs_,n,m);
if trace_ then write "pp_=",matify(pp_,m,m);
end;
\end{reduce}


Terminate the loop once residuals are zero.

\begin{reduce}
showtime;
if {resde_,resd_,resz_}={0,0,0} then write iter:=iter+10000;
end;
\end{reduce}

Only proceed to print if terminated successfully.

\begin{reduce}
if {resde_,resd_,resz_}={0,0,0} 
  then write "SUCCESS: converged to an expansion"
  else <<write "FAILED TO CONVERGE; I QUIT";
    if thecase=myweb then <<shut "cmsyso.txt"; 
    quit >> >>;
\end{reduce}


%%%%%%%%%%%%%%%%%%%%%%%%%%%%%%%
\begin{reduce}
%write "Temporarily halt here";end;
\end{reduce}


\section{Output text version of results}

Once construction is finished, simplify \verb|cis(0)|.

\begin{reduce}
let cis(0)=>1;
\end{reduce}

Invoking switch \verb|complex| improves some of the output of the complex numbers, but wrecks other parts of the output.  Best left off.
%\begin{reduce}
%on complex;
%\end{reduce}


Write text results.
\begin{reduce}
write "The centre manifold is (to one order lower)";
for j:=1:n do write "u",j," = ",
  coeffn(small*uu_,e_(j,1),1)/small;
write "The evolution of the real/complex amplitudes";
for j:=1:m do write "ds(",j,")/dt = ",
  coeffn(gg_,e_(j,1),1);
\end{reduce}

Optionally write the projection vectors.
\begin{reduce}
if itisSlowMan_ then begin
  write "The normals to the isochrons at the slow manifold.
Use these vectors: to project initial conditions
onto the slow manifold; to project non-autonomous
forcing onto the slow evolution; to predict the
consequences of modifying the original system; in
uncertainty quantification to quantify effects on
the model of uncertainties in the original system.";
  for j:=1:m do write "z",j," = ",
    for i:=1:n collect coeffn(zs_,e_(i,j),1);
end;
\end{reduce}

Write text results numerically evaluated when expressions are long.
\begin{reduce}
if length(gg_)>30 then begin
on rounded; print_precision 4;
write "Numerically, the centre manifold is (to one order lower)";
for j:=1:n do write "u",j," = ",
  coeffn(small*uu_,e_(j,1),1)/small;
write "Numerically, the evolution of the real/complex amplitudes";
for j:=1:m do write "ds(",j,")/dt = ",
  coeffn(gg_,e_(j,1),1);
if itisSlowMan_ then begin
  write "Numerically, normals to isochrons at slow manifold.";
  for j:=1:m do write "z",j," = ",
    for i:=1:n collect coeffn(zs_,e_(i,j),1);
end;
off rounded;
end;
\end{reduce}

\begin{reduce}
if thecase=myweb then shut "cmsyso.txt"$
\end{reduce}


There is an as yet unresolved problem in the typesetting when the argument of \verb|cis| (frequency) is a rational number instead of integer:
the numerator has an extra pair of parentheses which then makes the typesetting wrong;
maybe we need a pre-\LaTeX\ filter??


\section{Output LaTeX version of results}
% AJR, 24 aug 2012 -- aug 2014

Change the printing of temporary arrays.
\begin{reduce}
array tmpzz(m,n);
defid tmp,name="u";
defid tmps,name="\dot s";
defid tmpz,name="\vec z";
defid tmpzz,name="z";
defindex tmpzz(down,down);
\end{reduce}

Gather complicated result
\begin{reduce}
%for k:=1:m do tmpz(k):=for j:=1:n collect (1*coeffn(zs_,e_(j,k),1));
for k:=1:m do for j:=1:n do tmpzz(k,j):=(1*coeffn(zs_,e_(j,k),1));
\end{reduce}

Write to a file the commands needed to write the LaTeX expressions.
Write the centre manifold to one order lower than computed.
\begin{reduce}
out "scratchfile.red";
write "write ""\)
\paragraph{The centre manifold}
These give the location of the centre manifold in
terms of parameters~\(s\sb j\).
\("";";
for j:=1:n do write "tmp(",j,
  "):=coeffn(small*uu_,e_(",j,",1),1)/small;";
\end{reduce}

Write the commands to write the ODEs on the centre manifold.
\begin{reduce}
write "write ""\)
\paragraph{Centre manifold ODEs}
The system evolves on the centre manifold such
that the parameters evolve according to these ODEs.
\("";";
for j:=1:m do write "tmps(",j,"):=1*coeffn(gg_,e_(",j,",1),1);";
\end{reduce}

Optionally write the commands to write the projection vectors on the slow manifold.
\begin{reduce}
if itisSlowMan_ then begin
  write "write ""\)
\paragraph{Normals to isochrons at the slow manifold}
Use these vectors: to project initial conditions
onto the slow manifold; to project non-autonomous
forcing onto the slow evolution; to predict the
consequences of modifying the original system; in
uncertainty quantification to quantify effects on
the model of uncertainties in the original system.
The normal vector \(\vec z\sb j:=(z\sb{j1},\ldots,z\sb{jn})\)
\("";";
  for i:=1:m do for j:=1:n do 
  write "tmpzz(",i,",",j,"):=tmpzz(",i,",",j,");";
end;
\end{reduce}

Finish the scratchfile.
\begin{reduce}
write "end;";
shut "scratchfile.red";
\end{reduce}

Execute the file with the required commands, with output to the main centre manifold LaTeX file.

\begin{reduce}
out "centreMan.tex";
on latex;
in "scratchfile.red"$
off latex;
shut "centreMan.tex";
\end{reduce}

\begin{reduce}
end;
\end{reduce}

\begin{reduce}
in_tex "latexout2.tex"$
\end{reduce}

\section{Fin}

That's all folks. 

\begin{reduce}
write "Finished constructing centre manifold of ODE/DDE"; 
\end{reduce}

\begin{reduce}
if thecase=myweb then begin
quit;
end;
\end{reduce}

\begin{reduce}
%end;%loop over cases
end;
\end{reduce}



\bibliographystyle{agsm}
\bibliography{bib,ajr}

\end{document}