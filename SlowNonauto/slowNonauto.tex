\documentclass[11pt,a5paper]{article}

\title{Computer algebra coordinate-free construction of slow
manifold in stochastic or non-autonomous multiscale
differential equations}

\author{A.~J. Roberts\thanks{School of Mathematical
Sciences, University of Adelaide, South Australia~5005,
\textsc{Australia}. \url{https://profajroberts.github.io/}}}

\date{31 May 2023 -- \today}



\usepackage{url,natbib,amsmath,defns,reducecode}
\RaisedNamesfalse
\let\harvardurl\url
\IfFileExists{ajr.sty}{\usepackage{ajr}}{}
\usepackage{mycleveref}
%\allowdisplaybreaks
\def\ou\big(#1,#2,#3\big)%
    {{{\rm e}^{\if#31\else#3\fi t}\star}#1\,}
\newcommand{\Z}[1]{{\rm e}^{#1t}{\star}}
\Vec s\Vec u\Vec x\Vec y\Vec z
\Vec X\Vec Y\Vec Z
\newcommand{\sgn}{\operatorname{sgn}}
\newcommand{\res}{\operatorname{Res}}
\def\eps{\ensuremath{\varepsilon}}
\def\sde{\textsc{s/ode}}

\begin{document}
\sloppy

\maketitle

\begin{abstract}
Provides a coded procedure to construct a slow manifold of a
wide class of systems of non-autonomous or stochastic
differential equations (herein abbreviated by \sde{}s).  The
method is coordinate-free.  The methodology is based upon
earlier research~\cite[]{Coullet83, Cox91, Roberts96a,
Chao95, Roberts06k}.  Interpret all \sde{}s in the
Stratonovich sense so the analysis applies to deterministic
differential equations, both non-autonomous and autonomous.
Cater for deterministic autonomous systems by simply
omitting the `noise'.  For generality, this coded procedure
caters for unstable modes, and for differential equation
systems with  a rational right-hand side.  
%This code also underlies an
%interactive web service~\cite[]{Roberts09c}.  
\end{abstract}

\tableofcontents



\section{Introduction}

%\paragraph{Installation}
%Download and install the computer algebra package
%\emph{Reduce} via \url{http://www.reduce-algebra.com}\quad
%Download and unzip the folder
%\url{https://profajroberts.github.io/SlowNonauto.zip} \quad
%Within the folder \verb|SlowNonauto|, start-up \emph{Reduce}
%and load the procedure by executing the command 
%\verb|in_tex "slowNonauto.tex"$| \footnote{This script
%changes many internal settings of \emph{Reduce}, so best to
%do only when needed.} Test your installation by then
%executing \verb|examplesystem();| \quad(see
%\cref{ss:eg}).
%
%\paragraph{Execution}
%Thereafter, construct a specified invariant manifold\slash
%normal form of a specific dynamical system by executing the
%following command with specific values for the input
%parameters.  
%%See \verb|manyExamples.pdf| for many and varied examples.
%\begin{verbatim}
%slownonauto(dudt, slowbasis, fasteval, fastevec, toosmall);
%\end{verbatim}


\paragraph{Inputs}
Write your \sde\ system in terms of variables~\(u_j(t)\). 
You must denote the slow modes by~\verb|u(1)| through
to~\verb|x(nu)|, or the aliases \verb|u1|,\ldots). Each
non-autonomous factor must be denoted by~\verb|w(.)| where
the dot denotes almost any label you care to choose: simple
numbers such as~\verb|w(1)| and/or~\verb|w(2)| are the usual
choices (no aliases); but other labels may be used. Often
\verb|w(.)| is a Stratonovich white noise, a derivative of a
Stratonovich Wiener process. Analyse deterministic,
autonomous, systems by omitting any noise term~\verb|w()| in
the differential equations.  Then, as in the example of the
next \cref{ss:eg}, the input parameters to the procedure are
the following:
\begin{itemize}
\item \verb|dudt|, a comma separated list within
\verb|mat((...))| of the right-hand sides of the \sde{}s for
the variables~\(u_j(t)\);

\item \verb|slowbasis|, a comma separated list within
\verb|mat(...)| of the (desired) basis vectors for the slow
subspace, corresponding to eigenvalues (near) zero;

\item \verb|fasteval|, a comma separated list within
\verb|mat((...))| of the fast eigenvalues of the
system---best if real numbers (not complex), but maybe OK
otherwise, especially by specifying \verb|sign(??)=>-1| or
whatever;

\item \verb|fastevec|, a comma separated list within
\verb|mat(...)| of the corresponding (generalised)
eigenvectors that span the fast subspace;

\item \verb|toosmall|, an integer giving the desired order
of error in the asymptotic approximation that is
constructed.  The procedure embeds the specified system in a
family of systems parametrised by~\(\eps\), and constructs a
slow manifold, and evolution thereon, of the embedding
system to the asymptotic error~\Ord{\eps^{\tt toosmall}} (as
\(\eps\to0\)).  Often the introduced artificial~\(\eps\) has
a useful physical meaning, but strictly you should evaluate
the output at \(\eps=1\) to recover results for the
specified system, and then reinterpret the results, and
errors, in terms of your system's actual `small' parameters.

The code \emph{also} truncates to errors~\Ord{\sigma^3}
where \(\sigma\)~characterises the magnitude of the
non-autonomous\slash stochastic effects~\verb|w(j)|.

\end{itemize}

The right-hand side expressions for the time-derivatives
must be multinomial in variables~$u_i$ and~$w_i$\,.  To
cater for rational function right-hand sides, one also may
include some~\(\dot u_i\) factors in the right-hand sides.


\paragraph{Outputs}  This procedure reports the embedded
system it actually analyses, the number of iterations taken,
the constructed time dependent coordinate transform (the
original variables~\uv\ as a function of the new
variables~\sv), and the corresponding evolution in the new
variables in terms of \sde{}s for~\sv.
\begin{itemize}
\item A plain report to the Terminal window in which
\emph{Reduce} is executing.

\item A \LaTeX\ source report written to the file
\verb|slowReport.tex| (with \verb|slowReportHdr.tex|
and \verb|slowReportSys.tex|). Generate a pdf version 
by executing \verb|pdflatex slowReport|\,.

\item Global array \verb|u_(i)| gives the parametrisation of
the slow manifold~\(u_i(\sv,\sigma,\eps)\).

\item Global array \verb|ff(i)| gives the corresponding
evolution \(\dot U_i=\verb|ff(i)|\)  as functions
of~\((\sv,\sigma,\eps)\).
\end{itemize}
One may change the appearance of the output somewhat. For
example, in order to group terms in output expressions,
execute a \verb|factor| command before executing the
procedure \verb|slownonauto(...)|.


\paragraph{Background}
The theoretical support for the results of the analysis of
this procedure is centre\slash stable\slash unstable
manifold theory \cite[e.g.,][]{Carr81, Knobloch82,
Haragus2011, Roberts2014a}, and an embryonic backwards
theory \cite[]{Roberts2018a}.  This particular procedure is
developed from that for human-efficient computer algebra
\cite[]{Roberts96a}, and extended to stochastic\slash
non-autonomous systems \cite[]{Chao95, Roberts06k}.

We use the computer algebra package \emph{Reduce} 
[\url{http://reduce-algebra.com/}] because it is both free 
and perhaps the fastest general purpose computer algebra 
system~\cite[e.g.]{Fateman2002}.



\subsection{A simple example: \texttt{examplesystem()}} 
\label{ss:eg}
Execute this example by loading the code with command
\verb|in_tex "slowNonauto.tex"$|\quad and then invoking the
command \verb|examplesystem();|\quad  The code here
constructs the non-autonomous slow manifold for the system
\begin{equation}
\dot u_1=-u_1+u_2-u_1^2+\sigma w_1(t), \quad
\dot u_2=+u_1-u_2+u_2^2-\sigma w_1(t).
\label{eqeg}
\end{equation}
The lowercase~$w_j(t)$, called a \emph{noise} within this
document, may denote the formal derivative~$dW_j/dt$ of a
Stratonovich Wiener process~$W_j(t,\omega)$. Alternatively,
\(w_j(t)\)~represent an arbitrary deterministic
time-dependent forcing, or some control, or some `coloured'
random process, or some other extrinsic input to the system.
Parameter~$\sigma$ controls the strength of the so-called
noise.

Use variables~\verb|u1,u2| to denote~\(u_1,u_2\), and use
\verb|w(j)| to denote the non-autonomous effect\slash
noise~$w_j$.  Hence the system~\cref{eqeg} is analysed for
`small'~\((u_1,u_2,\sigma)\) by executing the following
defined procedure:
\begin{reduce}
procedure examplesystem();  begin
    factor small,sigma;
    slownonauto(% first give RHS of S/ODEs
    mat(( -u1+u2-u1^2+w(1), u1-u2+u2^2-w(1) )),
    mat((1,1)), % 2nd list basis vectors for slow subspace
    mat((-2)),  % 3rd list fast eigenvalues
    mat((-1,1)),% 4th list corresponding fast eigenvectors
    3 )$        % last give order of error in small eps
    end;
\end{reduce}
The procedure \verb|slownonauto| automatically multiplies
the noise\slash non-autonomous factors by a
parameter~\verb|sigma| so there is no need include the
parameter~$\sigma$ in the specification of the problem---it
will be done for you.

Further, the procedure uses the parameter~\verb|small|, often
denoted by~\(\eps\), to control truncation in nonlinearity. 
The fourth parameter in the above specifies to construct the
manifold to errors~\Ord{\eps^3}.

Consequently, the procedure embeds the given system as the
\(\eps=1\) version of the following system that it actually
analyses:
\begin{equation}
\dot u_1=-u_1+u_2-\eps u_1^2+\sigma w_1(t), \quad
\dot u_2=+u_1-u_2+\eps u_2^2-\sigma w_1(t).
\label{eqem}
\end{equation}
The construction is actually based about the subspace of
equilibria \(u_1-u_2=\eps=\sigma=0\)\,.  The constructed
time-dependent slow manifold and corresponding \sde{}s are the
following.


\paragraph{Time-dependent slow manifold} (for simplicity
reported to one order lower in both~\(\eps\) and~\(\sigma\)
than actually constructed)
\begin{align*}&
u_1= s_1 -1/2 \eps s_1^{2}
+\sigma  \ou\big(w_1,tt,-2\big)
+\sigma  \eps \ou\big(w_1,tt,-2\big) s_1
+O\big(\varepsilon ^{2},\sigma^{2}\big)
\\&
u_2=s_1 +1/2 \eps s_1^{2}
-\sigma  \ou\big(w_1,tt,-2\big)
+\sigma  \eps \ou\big(w_1,tt,-2\big) s_1
+O\big(\varepsilon ^{2},\sigma^{2}\big)
\end{align*}
On this slow manifold the evolution is governed by the \sde
\begin{equation*}
\dot s_1=+\eps^{2} s_1^{3}
-\sigma  \eps s_1 w_1
+1/2 \sigma ^{2} \eps^{2} \ou\big(w_1,tt,-2\big) s_1 w_1
+O\big(\varepsilon ^{3},\sigma^{3}\big)
\end{equation*}
Since this construction is based upon the subspace of
equilibria \(u_1-u_2=\eps=\sigma=0\), these constructed
expressions for the slow manifold are global in
parameter~\(s_1\) and asymptotic as~\(\eps,\sigma\to0\).
However, evaluation at \(\eps=1\) typically means that the
slow manifold expressions should be reinterpreted as
asymptotic as~\(s_1,\sigma\to0\).






\section{Header of the procedure}

For testing only, the following procedure is for reporting
some errors during development
\begin{reduce}
procedure testy(a,b,ind); begin
    write "TESTY";
    write a:=a;
    write b:=b;
    write indi:=ind;
    rederr "procedure testy caught some failure";
    end$
\end{reduce}


Need a couple of things established before defining the
procedure: the \verb|rlfi| package for a nicer version of
the output via \LaTeX; and operator names for the variables
of the \sde{}s.
\begin{reduce}
load_package rlfi; 
operator u;
operator w;
\end{reduce}
Cater for rational function \sde{}s by allowing time
dependence in~\verb|u| at specification. For rational
function \sde{}s, users must multiply each \sde\ by a common
denominator, and put on the right-hand side the nonlinear
terms involving the time derivative.
\begin{reduce}
depend u,t;
\end{reduce}


Now define the procedure as an operator so we can define
procedures internally, and may be flexible with its
arguments.
\begin{reduce}
operator slownonauto;
for all dudt, slowbasis, fasteval, fastevec, toosmall let
    slownonauto(dudt, slowbasis, fasteval, fastevec, toosmall) 
    = begin 
\end{reduce}


\subsection{Preamble to the procedure}
Operators and arrays are always global, but we can make
variables and matrices local, except for matrices that need
to be declared \verb|matrix|.  So, move to implement all
arrays and operators to have underscores, and almost all
scalars and most matrices to be declared local here (for
some reason \verb|x/y/zrhs| must be global).
\begin{reduce}
scalar maxiter, trace, nu, offdiag, jac, ok, res,
res0, res1, lengthresu;
\end{reduce}

Parameter \verb|maxiter|~is the maximum number of allowed
iterations; this default may be changed.
\begin{reduce}
maxiter:=9$ 
\end{reduce}
For optional trace printing of test cases: comment out
second line when not needed.
\begin{reduce}
trace:=0$
%trace:=1; maxiter:=5; 
\end{reduce}


Write an intro message.
\begin{reduce}
write "Construct non-autonomous slow manifold (version 9 Sep 2023)"$
\end{reduce}
Transpose the defining matrices so that vectors and lists are columns.
\begin{reduce}
urhs_ := tp dudt;
if trace then write urhs_:=urhs_;
sbase_ := tp slowbasis;
if trace then write sbase_:=sbase_;
fevec_ := tp fastevec;
if trace then write fevec_:=fevec_;
feval_ := tp fasteval;
if trace then write feval_:=feval_;
\end{reduce}


In the printed expressions, may factor \verb|small|~(\eps)
and \verb|sigma|~($\sigma$).
\begin{reduce}
if trace then factor small,sigma; 
\end{reduce}

The code cannot yet handle any cubic or higher order in noise amplitude~\verb|sigma|.
\begin{reduce}
let sigma^3=>0;
\end{reduce}

The \verb|rationalize| switch may make code much faster with
complex numbers. The switch \verb|gcd| seems to wreck
convergence, so leave it off.
\begin{reduce}
on div; off allfac; on revpri; 
on rationalize; %on ezgcd;
\end{reduce}







\subsection{Extract the differential equations}

The number of differential equations is the number of rows in the
matrix~\verb|dudt|.
\begin{reduce}
write "no. of DEs ",nu:=part(length(urhs_),1);
write "no. of slow variables ",ns:=part(length(sbase_),2);
write "no. of fast variables ",nf:=part(length(fevec_),2);
if nu=ns+nf then else rederr "Abort: incompatible dimensions";
if nf=part(length(feval_),1) then 
else rederr "different number of eigenvalues and eigenvectors";
if nu=part(length(sbase_),1) then
else rederr "dimension of slow basis vectors do not match the DEs";
if nu=part(length(fevec_),1) then
else rederr "dimension of fast eigenvectors do not match the DEs";
%for i:=1:nf do if numberp(feval_(i,1)) then 
%    else rederr "Need all fast eigenvalues to be simple numbers";
\end{reduce}
Define aliases \verb|ui| map to \verb|u(i)|, except you 
\emph{must} use the \verb|u(i)|-form \emph{within} any 
\verb|df(,t)| on the right-hand side.
\begin{reduce}
for i:=1:nu do set(mkid(u,i),u(i));
if trace then write "finished aliases";
\end{reduce}

Define a zero matrix and zero vectors.
\begin{reduce}
matrix o_(nu,nu),ov_(nu,1),ov_s(ns,1);
\end{reduce}

Check on the basis matrix: putting the slow vectors in the
first~\(n_s\) columns; and the fast vectors into the
remaining~\(n_f\) columns.
\begin{reduce}
pp_:=o_;
for i:=1:nu do for j:=1:ns do pp_(i,j):=sbase_(i,j);
for i:=1:nu do for j:=1:nf do pp_(i,j+ns):=fevec_(i,j);
pp_inv:=1/pp_;
if trace then write "pp_inv := ",pp_inv;
\end{reduce}


Adjust system to suit the linear operator implicitly defined
by the basis vectors and eigenvalues.  Multiply all other
terms, linear or nonlinear, by small.
\begin{reduce}
ll_:=o_;  
for i:=1:nf do ll_(i+ns,i+ns):=feval_(i,1);
ll_:=pp_*ll_*pp_inv;
if trace then write "ll_ :=",ll_;
u_:=ov_;
for i:=1:nu do u_(i,1):=u(i);
urhs_:=ll_*u_+small*(urhs_-ll_*u_);
\end{reduce}

Adjust the non-autonomous terms. Remove the \verb|small|
multiplication of noise terms, and instead multiply
by~\verb|sigma| to empower independent control of the
truncation in noise amplitude.
\begin{reduce}
urhs_:=(urhs_ where w(~j)=>sigma*w(j,xyzzy)/small)$
urhs_:=(urhs_ where w(~j,xyzzy)=>w(j))$
\end{reduce}
\cref{sec:dwti} writes the resulting differential
equations for information.
\begin{reduce}
if trace then for i:=1:nu do 
    write "du(",i,")/dt = ",urhs_(i,1);
\end{reduce}






\section{Setup LaTeX output using rlfi}

Use inline math environment so that long lines, the norm,
get line breaks. The command \verb|\raggedright| in the
\LaTeX\ preamble appears the best option for the line
breaking, but \verb|\sloppy| would also work reasonably.
\begin{reduce}
mathstyle math;
\end{reduce}

\paragraph{Define names for \LaTeX\ formatting}

Define some names I use, so that rlfi translates them to
Greek characters in the \LaTeX.
\begin{reduce}
defid small,name="\eps";%varepsilon;
defid alpha,name=alpha;
defid beta,name=beta;
defid gamma,name=gamma;
defid delta,name=delta;
defid epsilon,name=epsilon;
defid varepsilon,name=varepsilon;
defid zeta,name=zeta;
defid eta,name=eta;
defid theta,name=theta;
defid vartheta,name=vartheta;
defid iota,name=iota;
defid kappa,name=kappa;
defid lambda,name=lambda;
defid mu,name=mu;
defid nu,name=nu;
defid xi,name=xi;
defid pi,name=pi;
defid varpi,name=varpi;
defid rho,name=rho;
defid varrho,name=varrho;
defid sigma,name=sigma;
defid varsigma,name=varsigma;
defid tau,name=tau;
defid upsilon,name=upsilon;
defid phi,name=phi;
defid varphi,name=varphi;
defid chi,name=chi;
defid psi,name=psi;
defid omega,name=omega;
defid Gamma,name=Gamma;
defid Delta,name=Delta;
defid Theta,name=Theta;
defid Lambda,name=Lambda;
defid Xi,name=Xi;
defid Pi,name=Pi;
defid Sigma,name=Sigma;
defid Upsilon,name=Upsilon;
defid Phi,name=Phi;
defid Psi,name=Psi;
defid Omega,name=Omega;
\end{reduce}

For the variables names I use, as operators, define how they
appear in the \LaTeX, and also define that their arguments
appear as subscripts.
\begin{reduce}
defindex w(down);
defindex u(down);
defindex s(down);
defindex ff(down);
\end{reduce}

First use these for the specified dynamical system, later
use them for the normal form equations.
\begin{reduce}
defid ff,name="\dot u";
\end{reduce}

The `Ornstein--Uhlenbeck' operator is to translate into a
\LaTeX\ command, see the preamble, that typesets the
convolution in a reasonable manner. The definition of the
\LaTeX\ command is a bit dodgy as convolutions of
convolutions are not printed in the correct order; however, 
convolutions commute so it does not matter.
\begin{reduce}
defid ou,name="\ou";
defindex ou(arg,arg,arg);
\end{reduce}


\paragraph{Write the \LaTeX\ dynamical system}

Because of the way rfli works, to get good quality output to
the \LaTeX\ document, I need to write the algebraic
expressions to a file, then read them back in again. While
being read back in, I send the output to the \LaTeX\ file.
In this convoluted way I avoid extraneous output lines
polluting the \LaTeX. 

Temporarily use these arrays for the right-hand sides of the
dynamical system.
\begin{reduce}
clear ff;
array ff(nu);
\end{reduce}

Write expressions to the file \verb|scratchfile.red| for
later reading. Prepend the expressions with an instruction
to write a heading, and surround the heading with anti-math
mode to cancel the math environment that \text{rlfi puts
in.}
\begin{reduce}
out "scratchfile.red"$
write "off echo;"$  % do not understand why needed in 2021??
write "write ""\)
\paragraph{Specified dynamical system}
\(""$";
for i:=1:nu do write "ff(",i,"):=urhs_(",i,",1);"; 
write "end;";
shut "scratchfile.red";
\end{reduce}

Then switch on \LaTeX\ output before writing to file as this
\LaTeX\ file is to be input from the main \LaTeX\ file and
hence does not need a header. The header here gets sent to
the `terminal' instead. Then write to
\verb|slowReportSys.tex| the expressions we stored in
\verb|scratchfile.red| as nice \LaTeX.
\begin{reduce}
write "Ignore the following five lines of LaTeX"$
on latex$
out "slowReportSys.tex"$
in "scratchfile.red"$
shut "slowReportSys.tex"$
off latex$
\end{reduce}



\section{Delayed write of text info}
\label{sec:dwti}

Because it is messy to interleave \LaTeX\ and plain output,
I delay writing anything much in plain text until here.

Write the delayed warning message about equations.
\begin{reduce}
write "
===== Inspect these printed equations.  Results 
are rubbish if not appropriate for little small."$
\end{reduce}

Write the plain text versions of the dynamical system.
\begin{reduce}
write "no. of variables ",nu:=nu;
for i:=1:nu do write "du(",i,")/dt = ",urhs_(i,1);
\end{reduce}



\section{Represent the noise}

The `noises'~\verb|w| depend upon time. But we find it
useful to discriminate upon the notionally fast time
fluctuations of a noise/non-autonomous process, and the
notionally ordinary time variations of the dynamic
variables~$u_i$. Thus introduce a notionally fast time
variable~\verb|tt|, which depends upon the ordinary
time~\verb|t|. Equivalently, view~\verb|tt|, a sort of
`partial~$t$', as representing variations in time
independent of those in the variables~$u_i$.
\begin{reduce}
depend w,tt;
depend tt,t,ttf;
\end{reduce}

In the construction, convolutions of the noise arise, both
backwards over history and also forwards in time to
anticipate the noise \cite[]{Roberts06k, Roberts2018a}. For
any non-zero parameter~$\mu$, define the
`Ornstein--Uhlenbeck' convolution
\begin{equation}
    \Z{\mu}\phi=
    \begin{cases}
        \int_{-\infty}^t \exp[\mu(t-\tau)]\phi(\tau)\,d\tau\,,
        &\mu<0\,, \\
        \int_t^{+\infty} \exp[\mu(t-\tau)]\phi(\tau)\,d\tau\,,
        &\mu>0\,,             
    \end{cases}
    \label{eq:zmuf}
\end{equation}
so that the convolution is always with a bounded
exponential. Five useful properties of this convolution are
\begin{align}&
    \Z\mu1=\frac1{|\mu|}\,,
    \label{eq:conv1}
    \\&
    \frac{d\ }{dt}\Z{\mu}\phi=-\sgn\mu\,\phi+\mu\Z{\mu}\phi\,,
    \label{eq:ddtconv}
    \\&
    E[\Z{\mu}\phi]=\Z{\mu}E[\phi]\,,
    \label{eq:exz}
    \\&
    E[(\Z{\mu}\phi)^2]=\frac1{2|\mu|}\,,
    \label{eq:exzz}
    \\&
    \Z\mu\Z\nu=\begin{cases}
    \frac1{|\mu-\nu|}\big[ \Z\mu+\Z\nu \big]\,, &\mu\nu<0\,, \\
    \frac{-\sgn\mu}{\mu-\nu}\big[ \Z\mu-\Z\nu \big]\,, 
    &\mu\nu>0\ \&\ \mu\neq\nu\,.
    \end{cases}
    \label{eq:twoconv}
\end{align}
Also remember that although with $\mu<0$ the
convolution~$\Z\mu$ integrates over the past, with $\mu>0$
the convolution~$\Z\mu$ integrates into the future---both
over a time scale of order~$1/|\mu|$.

The operator~\verb|ou(f,tt,mu)| represents the
convolution~$\Z\mu f$ as defined by~\eqref{eq:zmuf}: called
\verb|ou| because it is an Ornstein--Uhlenbeck process when
\(f\)~is a stochastic white noise. The operator~\verb|ou| is
`linear' over fast time~\verb|tt| as the convolution only
arises from solving \pde{}s in the operator
$\partial_t-\mu$\,. Code its derivative~\eqref{eq:ddtconv}
and its action upon autonomous terms~\eqref{eq:conv1}:
\begin{reduce}
clear ou; operator ou; linear ou;
let { df(ou(~f,tt,~mu),t)=>-sign(mu)*f+mu*ou(f,tt,mu)
    , ou(1,tt,~mu)=>1/abs(mu) when mu neq 0
\end{reduce}
Also code the transform~\eqref{eq:twoconv} that successive
convolutions at different rates may be transformed into
several single convolutions.% modified 2023-08-08
\begin{reduce}
    , ou(ou(~r,tt,~nu),tt,~mu) => 
      (ou(r,tt,mu)+ou(r,tt,nu))/abs(mu-nu) when mu neq 0
      when (sign(mu)+sign(nu)=0)and(sign(mu)neq 0)
    , ou(ou(~r,tt,~nu),tt,~mu) => 
      -sign(mu)*(ou(r,tt,mu)-ou(r,tt,nu))/(mu-nu)
      when (sign(mu)=sign(nu))and(mu neq nu)
    };
\end{reduce}
The above properties are \emph{critical}: they must be
correct for the results to be correct.  Currently, they are
only coded for \emph{real} rates~\(\mu,\nu\).

Second, identify the resonant parts, some of which must go
into the evolution~\verb|gg(i)|, and some into the
transform. It depends upon the exponent of~\verb|yz|
compared to the decay rate of this mode, here~\verb|r|.
\begin{reduce}
clear reso_; operator reso_; linear reso_;
let { reso_(~a,yz,~r)=>1 when df(a,yz)*yz=r*a
    , reso_(~a,yz,~r)=>0 when df(a,yz)*yz neq r*a
    };
\end{reduce}

Lastly, the remaining terms get convolved at the appropriate
rate to solve their respective homological equation by the
operator~\verb|zres_|.
\begin{reduce}
depend yz,ttf;
clear zres_; operator zres_; linear zres_;
let zres_(~a,ttf,~r)=>ou(sign(df(a,yz)*yz/a-r)
    *sub(yz=1,a),tt,df(a,yz)*yz/a-r);
\end{reduce}




\section{Operators to solve noisy homological equation}

When solving homological equations of the form
$F+\xi_t=\res$ (the resonant case $\mu=0$), we separate the
terms in the right-hand side~$\res$ into those that are
integrable in fast time, and hence modify the coordinate
transform by changing~$\xi$, and those that are not, and
hence must remain in the evolution by changing~$F$.  the
operator \verb|zint_| extracts those parts of a term that we
know are integrable; the operator \verb|znon_| extracts
those parts which are not knowably bounded integrable. With
more research, more types of terms may be found to be
integrable; hence what is extracted by \verb|zint_| and what
is left by \verb|zint_| may change with more research, or in
different scenarios.  These transforms are not critical:
changing the transforms may change intermediate
computations, but as long as the iteration converges, the
computer algebra results will be algebraically correct.
\begin{reduce}
clear zint_; operator zint_; linear zint_;
clear znon_; operator znon_; linear znon_;
\end{reduce}


First, avoid obvious secularity.
\begin{reduce}
zrules:= { zint_(w(~i),tt)=>0,    znon_(w(~i),tt)=>w(i)
    , zint_(1,tt)=>0,        znon_(1,tt)=>1
    , zint_(w(~i)*~r,tt)=>0, znon_(w(~i)*~r,tt)=>w(i)*r
\end{reduce}
Second, by~\eqref{eq:ddtconv} a convolution may be split
into an integrable part, and a part in its argument which in
turn may be integrable or not.  Had to change \verb|mu,nu| 
to \verb|mur,nur| in order to avoid inscrutable error.
\begin{reduce}
    , zint_(ou(~r,tt,~mur),tt)
      =>ou(r,tt,mur)/mur+zint_(r,tt)/abs(mur)
    , znon_(ou(~r,tt,~mur),tt)=>znon_(r,tt)/abs(mur)
\end{reduce}
Third, squares of convolutions may be integrated by parts to
an integrable term and a part that may have integrable or
non-integrable parts.
\begin{reduce}
    , zint_(ou(~r,tt,~mur)^2,tt)=>ou(~r,tt,~mur)^2/(2*mur)
                              +zint_(r*ou(r,tt,mur),tt)/abs(mur)
    , znon_(ou(~r,tt,~mur)^2,tt)=>znon_(r*ou(r,tt,mur),tt)/abs(mur)
\end{reduce}
Fourth, different products of convolutions may be similarly
separated using integration by parts. Why is the 2nd one
here not acted upon, unless double tilde is used??
\begin{reduce}
    , zint_(ou(~r,tt,~~mur)*ou(~s,tt,~~nur),tt)
      =>ou(r,tt,mur)*ou(s,tt,nur)/(mur+nur)
      +zint_(sign(mur)*r*ou(s,tt,nur)+sign(nur)*s*ou(r,tt,mur),tt)
      /(mur+nur) when mur+nur neq 0
    , znon_(ou(~r,tt,~~mur)*ou(~s,tt,~~nur),tt)=>
      +znon_(sign(mur)*r*ou(s,tt,nur)+sign(nur)*s*ou(r,tt,mur),tt)
      /(mur+nur) when mur+nur neq 0
\end{reduce}
However, a zero divisor arises when $\mu+\nu=0$ in the
above. Here code rules to cater for such terms by increasing
the depth of convolutions over past history.
\begin{reduce}
    , zint_(ou(~r,tt,~~mur)*ou(~s,tt,~~nur),tt)=>
      ou(ou(r,tt,-nur),tt,-nur)*ou(s,tt,nur)
      +zint_(ou(ou(r,tt,-nur),tt,-nur)*s,tt) 
      when (mur+nur=0)and(sign(nur)>0)
    , znon_(ou(~r,tt,~~mur)*ou(~s,tt,~~nur),tt)=>
      znon_(ou(ou(r,tt,-nur),tt,-nur)*s,tt) 
      when (mur+nur=0)and(sign(nur)>0)
\end{reduce}
The above handles quadratic products of convolutions.
Presumably, if we seek cubic noise effects then we may need
cubic products of convolutions. However, I do not proceed so
far and hence terminate the separation rules.
\begin{reduce}
    }; % let/zrules
\end{reduce}




\section{Initialise approximate transform}

Truncate asymptotic approximation of the coordinate
transform depending upon the parameter~\verb|toosmall|. Use
the `instant evaluation' property of a loop index to define
the truncation so that Reduce omits small terms \text{on the
fly.}
\begin{reduce}
for j:=toosmall:toosmall do let small^j=>0;
\end{reduce}

Variables \verb|u| are operators in
the specification of the equations. We now want them to map
to the approximation to the coordinate transform, so point
them to arrays storing the normal form expressions.  Need to
clear the mapping to the array before exiting.
\begin{reduce}
clear u_;
let u(~j)=>u_(j,1);
\end{reduce}

Express the normal form in terms of new evolving variables
$s_i$, denoted by operators \verb|s(i)|. The expressions for
the normal form \sde{}s are stored in~\verb|ff|.
\begin{reduce}
clear s;
operator s;
depend s,t;
let df(s(~i),t)=>ff(i);
\end{reduce}

The first linear approximation is then $\uv \approx V_s\sv$
such that $\dot s_i \approx 0$\,, in~\verb|ff(i)|. 
\begin{reduce}
ss_:=ov_s;
for j:=1:ns do ss_(j,1):=s(j);
u_:=sbase_*ss_;
if trace then write u_:=u_;
for j:=1:ns do ff(j):=0;
\end{reduce}





\section{Iterative updates}

We iterate to a solution of the governing \sde{}s to
residuals of some order of error. The number of iterations
are limited by a maximum.
\begin{reduce}
write "===== Starting the constructive iteration";
res_:=-df(u_,t)+urhs_;
write lengthres := tp map(length(~q),res_);

for iter:=1:maxiter do begin
  if trace then write "
  ITERATION = ",iter,"
  -------------";
\end{reduce}


Compute the current residual vector.
\begin{reduce}
  if trace then write res_:=res_;
  mode_:=pp_inv*res_;
  if trace then write mode_:=mode_;
\end{reduce}




\subsection{Fast modes}

Modify the slow manifold parametrisation in view of the fast
modes.
\begin{reduce}
  du_:=ov_;
  for j:=1:nf do begin  
    dyi_:=-zres_(mode_(j+ns,1),ttf,-feval_(j,1));
    if trace then write dyi_:=dyi_;
    testing:=(dyi_ where znon_(~a,~b)=>testy(a,b,-1));
    du_(j+ns,1):=dyi_;
  end;
if trace then write "finished fast modes";
\end{reduce}



\subsection{Slow modes}

Update the $s_i$~evolution \verb|ff(i)| and parts of the
\uv~parametrisation. 
\begin{reduce}
  for j:=1:ns do begin
    ff(j):=ff(j)+(znon_(mode_(j,1),tt) where zrules);
    if trace then write "ds",j,"/dt = ",ff(j);
    testing:=(ff(j) where znon_(~a,~b)=>testy(a,b,0));
    dxi_:= (zint_(mode_(j,1),tt) where zrules);
    if trace then write dxi_:=dxi_;
    du_(j,1):=dxi_;
    testing:=(dxi_ where zint_(~a,~b)=>testy(a,b,1));
  end;
\end{reduce}

Update the slow manifold parametrisation
\begin{reduce}
  u_:=u_+pp_*du_;
  if trace then write u_:=u_;
\end{reduce}


Terminate the iteration loop once all residuals are zero,
or the maximum number of iterations has been done.
\begin{reduce}
  res_:=-df(u_,t)+urhs_;
  write lengthres := tp map(length(~q),res_);
  showtime;
  if res_=ov_ then write "Number of iterations ",
    iter:=1000000+iter;
end;
\end{reduce}





\section{Output results}

Only proceed to print if terminated successfully.
\begin{reduce}
if res_=ov_ 
  then write "===== SUCCESS: converged to an expansion"
  else begin
      out "slowReport.tex"; on latex;
      write "Failed-to-converge"; 
      off latex; shut "slowReport.tex";
      rederr "===== FAILED TO CONVERGE: I EXIT";
  end;
\end{reduce}


\subsection{Plain text version}

Print the resultant coordinate transform: but only print to
one lower power in~\verb|small| and~\verb|sigma| in order to
keep output relatively small.
\begin{reduce}
write "Parametrisation of time-dependent slow manifold"$
write "(to one order lower in both small and sigma)"$
if trace then write u_:=u_ 
else begin u_:=sigma*small*u_; 
     write u_:=u_/small/sigma; end;
\end{reduce}

Print the \sde{}s of the parametrisation of the slow manifold:
\begin{reduce}
write "The slow manifold S/ODEs";
for i:=1:ns do write "ds(",i,")/dt = ",ff(i);
\end{reduce}



\subsection{\LaTeX\ version}

Include order of error to make printing more robust. But we
cannot use \verb|small^toosmall| in the following as that is
set to zero (for the asymptotics), so we hard code that
\verb|small| appears as \verb|varepsilon|~\(\varepsilon\). 
Further, to avoid \verb|sigma^3| being replaced by zero,
introduce \verb|sigma_| that maps to~\(\sigma\).
\begin{reduce}
clear order_;  operator order_;
defid order_,name="O";
defindex order_(arg,arg);
defid sigma_,name="\sigma";
\end{reduce}


As before, we have to write expressions to file for later
reading so they get printed without extraneous dross in the
\LaTeX\ source. First open up the temporary file
\verb|scratchfile.red| again.
\begin{reduce}
out "scratchfile.red"$
write "off echo;"$ % do not understand why needed in 2021??
\end{reduce}

Write the time-dependent coordinate transform to file, with a
heading, and with an anti-math environment to cancel the
auto-math of rlfi. For some reason we have to keep these
writes short as otherwise it generates spurious fatal
blank lines in the \LaTeX.
\begin{reduce}
write "write ""\)
\paragraph{Time dependent slow manifold parametrisation}
\("";";
for i:=1:nu do write "u_",i,":=u_(",i,
    ",1) +order_(varepsilon^",toosmall-1+trace,",sigma_^",2+trace,");"; 
\end{reduce}

Write the resultant model differential equations to file,
with a heading, and with an anti-math environment to cancel
the auto-math of rlfi.
\begin{reduce}
write "write ""\)
\paragraph{Result slow manifold DEs}
\("";";
for i:=1:ns do write "ff(",i,"):=ff(",i,
    ") +order_(varepsilon^",toosmall,",sigma_^3);";
write "end;";
\end{reduce}

Shut the temporary output file.
\begin{reduce}
shut "scratchfile.red";
\end{reduce}

Get expressions from file and write the main \LaTeX\ file.
But first \emph{redefine} how these names get printed,
namely as the slow parameters time derivatives. 
\begin{reduce}
defindex s_(down);
defid ff,name="\dot s";
\end{reduce}

Penultimately, write the header information file that is 
to be included in the report via an \verb|\input| in the 
modified \verb|on latex|.
\begin{reduce}
out "slowReportHdr.tex"$
write "\title{A slow manifold of your dynamical system}"$
write "\author{A. J. Roberts, University of Adelaide\\
\texttt{http://orcid.org/0000-0001-8930-1552}}
\date{\now, \today}
\def\ou\big(#1,#2,#3\big)%
    {{{\rm e}^{\if#31\else#3\fi t}\star}\left(#1\right)}
\def\eps{\varepsilon} \def\_{_}
\maketitle
Generally, the lowest order, most important,
terms are near the end of each expression.

off echo;

\(\)
\paragraph{Specified dynamical system}
\(
\)\par
\(\dot u_{1}=\sigma  w_{1}-\eps u_{1}^{2}+u_{2}-u_{1}
\)\par
\(\dot u_{2}=-\sigma  w_{1}+\eps u_{2}^{2}-u_{2}+u_{1}
\)\par

"$
shut "slowReportHdr.tex"$
\end{reduce}


Finally write to the main \LaTeX\ file so switch on latex
after starting to write to the file. Then write expressions
in \verb|scratchfile.red| to \verb|slowReport.tex| as nice
\LaTeX. Switch off latex, to get the end of the document,
and finish writing.
\begin{reduce}
out "slowReport.tex"$
on latex$
in "scratchfile.red"$
off latex$
shut "slowReport.tex"$
\end{reduce}


\section{Fin}
That's all folks, so end the procedure, after clearing the
mapping from operators to the stored expressions. 
\begin{reduce}
clear u(~j);
return Finished_slow_manifold_of_system$ 
end$
\end{reduce}





\section{Override some rlfi procedures}

Now setup the rlfi package to write a \LaTeX\ version of the
output. It is all a bit tricky and underhand. We override
some stuff from \verb|rlfi.red|.\footnote{Find it in
\url{reduce-algebra/trunk/packages/misc/rlfi.red}}  

First, change \verb|name| to get Big delimiters, not
left-right delimiters, so \LaTeX\ can break lines.
\begin{reduce}
deflist('((!( !\!b!i!g!() (!) !\!b!i!g!)) (!P!I !\!p!i! )
     (!p!i !\!p!i! ) (!E !e) (!I !i) (e !e) (i !i)),'name)$
\end{reduce}

Override the procedure that prints annoying messages about
multicharacter symbols. It ends the output of one
expression. This is mainly a copy from \verb|rlfi.red| with
the appropriate if-statement deleted.
\begin{reduce}
symbolic procedure prinlaend;
<<terpri();
  prin2t "\)\par";
  if !*verbatim then
      <<prin2t "\begin{verbatim}";
        prin2t "REDUCE Input:">>;
  ncharspr!*:=0;
  if ofl!* then linelength(car linel!*)
    else laline!*:=cdr linel!*;
  nochar!*:=append(nochar!*,nochar1!*);
  nochar1!*:=nil >>$
\end{reduce}
Similarly, hardcode at the beginning of expression output
that the mathematics is in inline mode.
\begin{reduce}
symbolic procedure prinlabegin;
<<if !*verbatim then
      <<terpri();
        prin2t "\end{verbatim}">>;
  linel!*:=linelength nil . laline!*;
  if ofl!* then linelength(laline!* + 2)
    else laline!*:=car linel!* - 2;
  prin2 "\(" >>$
\end{reduce}

Override the procedure that outputs the \LaTeX\ preamble
upon the command \verb|on latex|.
\begin{reduce}
symbolic procedure latexon;
<<!*!*a2sfn:='texaeval;
  !*raise:=nil;
  prin2t "\documentclass[11pt,a5paper]{article}";
  prin2t "\usepackage[a5paper,margin=13mm]{geometry}";
  prin2t "\usepackage{parskip,time} \raggedright";
  prin2t "\begin{document}
\title{A slow manifold of your dynamical system}


\author{A. J. Roberts, University of Adelaide\\
\texttt{http://orcid.org/0000-0001-8930-1552}}
\date{\now, \today}
\def\ou\big(#1,#2,#3\big)%
    {{{\rm e}^{\if#31\else#3\fi t}\star}\left(#1\right)}
\def\eps{\varepsilon} \def\_{_}
\maketitle
Generally, the lowest order, most important,
terms are near the end of each expression.

off echo;

\(\)
\paragraph{Specified dynamical system}
\(
\)\par
\(\dot u_{1}=\sigma  w_{1}-\eps u_{1}^{2}+u_{2}-u_{1}
\)\par
\(\dot u_{2}=-\sigma  w_{1}+\eps u_{2}^{2}-u_{2}+u_{1}
\)\par

";
  if !*verbatim then
      <<prin2t "\begin{verbatim}";
        prin2t "REDUCE Input:">>;
  put('tex,'rtypefn,'(lambda(x) 'tex)) >>$
\end{reduce}

End the file when input to Reduce
\begin{reduce}
end;
\end{reduce}






\bibliographystyle{agsm}
\bibliography{ajr,bib}


\end{document}
 
