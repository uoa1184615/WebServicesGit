%!TEX root = ../manyExamples.tex
\subsection{\texttt{monahanOne}: Monahan's first is a simple nonlinear example} 
\label{monahanFOne}


\cite{Monahan2011} discuss stochastic averaging and give four examples in an appendix which we also analyse---here we analyse the first example.
They really need this approach as ``a large separation often does not exist in atmosphere or ocean dynamics'' between the fast and slow time scales.

With small scale-separation parameter~\(\tau\), \cite{Monahan2011} first consider the example
\begin{equation*}
\de tx=-x+\Sigma(x)y \qtq{and}
\de ty=-\frac1\tau y+\frac1{\sqrt\tau}\dot W,
\end{equation*}
for general smooth functions~$\Sigma(x)$.
Rescale time, $t=\tau t'$  so that $\de t{}=\frac1\tau\de{t'}{}$ and $\dot W=\frac1{\sqrt{\tau}}\de {t'}W$.
Then, dropping dashes, the \sde\ is
\begin{equation*}
\de tx=-\tau x+\tau\Sigma(x)y
\qtq{and}
\de ty=- y+\dot W.
\end{equation*}

Start by loading the procedure.
\begin{reduce}
in_tex "../stoNormForm.tex"$
\end{reduce}
Execute the construction of a normal form for this system.  
Here let's restrict the general function~\(\Sigma(x)\) to the rational form  $\Sigma(x):=(a_0+a_1x+a_2x^2)/(1+b_1x+b_2x^2)$. Code this form as the following (after multiplying through by the denominator).
\begin{reduce}
factor tau,yy,y,w,ou;
operator a; defindex a(down);
operator b; defindex b(down);
stonormalform(
    { -tau*x(1)*(1+b(1)*x(1)+b(2)*x(1)^2)
      -df(x(1),t)*(b(1)*x(1)+b(2)*x(1)^2)
      +tau*y(1)*(a(0)+a(1)*x(1)+a(2)*x(1)^2) },
    { -y(1)+w(1) },
    {},
    3 )$
end;
\end{reduce}

The procedure reports that it analyses the following family 
\begin{align*}&
\dot x_{1}=\sqrt {\tau } y_{1} \eps a-\eps^{2} \tau  x_{1}
&&
\dot y_{1}=w_{1} \sigma -y_{1}+\sqrt {\tau } \eps x_{1}
\end{align*}
in which we indeed see~\eps\ only in the grouping \(\eps\sqrt\tau\).



\paragraph{Time dependent coordinate transform}  This is linear as the system is linear.
\begin{align*}&
y_{1}=\ParMath{-\ou\big(\ou\big(w_{1},tt,-1\big),tt,-1\big) \sigma  \eps^{2} 
\tau  a-\ou\big(w_{1},tt,-1\big) \sigma  \eps^{2} \tau  a+\ou\big(w_{1},
tt,-1\big) \sigma +Y_{1}+\sqrt {\tau } \eps X_{1}
}
\\&
x_{1}=-\sqrt {\tau } \ou\big(w_{1},tt,-1\big) \sigma  \eps a-\sqrt {
\tau } Y_{1} \eps a+X_{1}
\end{align*}

\paragraph{Result normal form DEs}
The normal form dynamics is linear and decoupled, as per Hartman--Grobman, namely
\begin{align*}&
\dot Y_{1}=-Y_{1} \eps^{2} \tau  a-Y_{1}
\\&
\dot X_{1}=w_{1} \sigma  \eps^{3} \tau\sqrt {\tau }  \big(-2 a^{2}+
 a\big)+\sqrt {\tau } w_{1} \sigma  \eps a+\eps^{2} \tau  
\big(X_{1} a-X_{1}\big)
\end{align*}
\cite{Monahan2011} derive the last two terms in the $X$-equation, but not the first as it is too small for their averaging analysis.
They comment that $a>1$ is some sort of difficulty, presumably because \(X\)~grows when \(a>1\): but here we have no problem with $a>1$, especially as the decay rate to the stochastic slow manifold, the $Y$-\sde, is $(1+\tau a)$ which gets stronger with increasing parameter~$a$.




