\documentclass[11pt,a5paper]{article}
\usepackage[a5paper,margin=13mm]{geometry}
\usepackage{parskip,time} \raggedright
\def\ou\big(#1,#2,#3\big){{{\rm e}^{\if#31\else#3\fi t}\star}#1\,}
\def\eps{\varepsilon}
\title{Normal form of your dynamical system}
\author{A. J. Roberts, University of Adelaide\\
\texttt{http://orcid.org/0000-0001-8930-1552}}
\date{\now, \today}
\begin{document}
\maketitle
Throughout and generally: the lowest order, most
important, terms are near the end of each expression.

off echo;


\(\)
\paragraph{Specified dynamical system}
\(
\)\par

\(\dot x_{1}=-x_{1} y_{1} \eps
\)\par

\(\dot y_{1}=w_{1} \sigma +x_{1}^{2} \eps-2 y_{1}^{2} \eps-y_{1}
\)\par


off echo;


\begin{math}
\end{math}
\paragraph{Time dependent coordinate transform}
\begin{math}
\end{math}\par

\begin{math}
y_{1}=-\ou\big(\ou\big(w_{1},tt,-1\big),tt,-1\big) \sigma  \eps^{2} 
\tau  a-\ou\big(w_{1},tt,-1\big) \sigma  \eps^{2} \tau  a+\ou\big(w_{1},
tt,-1\big) \sigma +Y_{1}+\sqrt {\tau } \eps X_{1}
\end{math}\par

\begin{math}
x_{1}=-\sqrt {\tau } \ou\big(w_{1},tt,-1\big) \sigma  \eps a-\sqrt {
\tau } Y_{1} \eps a+X_{1}
\end{math}\par

\begin{math}
\end{math}
\paragraph{Result normal form DEs}
\begin{math}
\end{math}\par

\begin{math}
\dot Y_{1}=-Y_{1} \eps^{2} \tau  a-Y_{1}
\end{math}\par

\begin{math}
\dot X_{1}=w_{1} \sigma  \eps^{3} \tau  \big(-2 \sqrt {\tau } a^{2}+
\sqrt {\tau } a\big)+\sqrt {\tau } w_{1} \sigma  \eps a+\eps^{2} \tau  
\big(X_{1} a-X_{1}\big)
\end{math}\par
\end{document}
